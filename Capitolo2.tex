\chapter{Model of Anti-de Sitter (n+1)-space}
The aim of this chapter is to construct models of Lorentzian manifolds with constant sectional curvature -1 and maximal isometry group in any dimension. We are also interested in \textit{stressing} the analogies of this manifolds with the models of hyperbolic space in the Riemannian setting. We will show that hyperbolic space is \textit{naturally} embedded in Anti-de Sitter space, and we will later develop this topic in our study of Earthquakes theory. After introducing the models, we are interested in studying the geometry of such manifolds: we will introduce a conformal boundary for Anti-de Sitter space and we will characterize geodesics and totally geodesic subspaces. We will also introduce the notion of polarity in Anti-de Sitter space (in same sense the \textit{correct} Lorentzian correspondence between points and hyperplane, analogous to Euclidean orthogonality) and study its properties. 

\section{The quadric model}
We want to introduce the analogue of the hyperboloid model of hyperbolic space. Denote by $\R^{n,2}$ the real vector space $\R^{n+2}$ equipped with the quadratic form 
\[
    q_{n,2}(x)=x_1^{2}+\dots+x_n^{2}-x_{n+1}^{2}-x_{n+2}^2.   
\]
and by $\langle v,w\rangle_{n,2}$ the associated symmetric form. Finally, let $O(n,2)$ be the group of linear transformation of $\R^{n+2}$ that preserve $q_{n,2}.$
Then we define:
\[
    \H^{n,1}=\{x\in \R^{n,2}\;|\;q_{n,2}(x)=-1\}.
\]

It is immediate to check that $\H^{n,1},$ as the pre-image of a regular value of $q_{n,2}$, is a smooth connected submanifold of $\R^{n,2}$ of dimension $n+1$. The tangent space $T_x\H^{n+1}$, regarded as a subspace of $\R^{n+2}$ coincides with the orthogonal space $x^{\perp}=\{y\in \R^{n+2}\;|\;\langle x,y\rangle_{n,2}=0\}.$ A simple signature argument (and $q_{n,2}(x)=-1$ for every $x\in\H^{n,1}$) shows that the restriction of the symmetric form $\langle .,. \rangle_{n,2}$ to $T\H^{n,1}$ has Lorentzian signature, hence it makes $\H^{n,1}$ a Lorentzian manifold. We remark that this model is the analogue of the hyperboloid model of hyperbolic space, in fact $\H^n$ is isometrically embedded in $\H^{n,1}$ as the submanifold defined by $x_{n+2}=0$, $x_{n+1}>0$. \\
The natural action of $O(n,2)$ on $\R^{n,2}$ preserves $\H^{n,1}$, and in facts $O(n,2)$ acts by isometries on $\H^{n,1}$. We remark that $O(n,2)$ acts transitively on $\H^{n,1}$ and that the stabilizer of a point $x$ acts transitively on the space of orthonormal bases of $T_x\H^{n,1}$. Hence $\H^{n,1}$ has maximal isometry group and Isom($\H^{n,1})\simeq O(n,2)$.\\
By Lemma \ref{maximalisometry}, $\H^{n,1}$ has constant sectional curvature. Let us now check that the sectional curvature is negative (actually, $K=-1$). For this purpose, observe that the normal line in $\R^{n,2}$ to $\H^{n,1}$ at $x$ is identified with the line generated by $x$ itself. It follows that, if $v,w$ are tangent vector fields along $\H^{n,1}$ we have the orthogonal decomposition: 
\[
    (D_{v}w)(x)=(\nabla_{v}w)(x)+\langle v,w\rangle x, 
\]

where $D$ is the flat connection of $\R^{n+2}$ and $\nabla$ is the Levi-Civita connection of $\H^{n,1}$. Using the flatness of $D$ we get: 
\[
    R(u,v)w=\langle u,w\rangle v-\langle v,w\rangle u,
\]

so that 
\[
    \langle R(u,v)v,u \rangle =-(\langle u,u\rangle\langle v,v\rangle-\langle v,u\rangle^2), 
\]

and this shows that $\H^{n,1}$ has constant sectional curvature $-1$. We also remark that $\H^{n,1}$ is not simply connected, being homeomorphic to $\R^{n}\times \mathbb{S}^1$. 

\section{The ``Klein model" and it's boundary.}
Let us introduce a projective model, also knows as the ``Klein model", for Anti-de Sitter geometry. Let us define: 
\[
    \A^{n,1}=\H^{n,1}/\{\pm \text{Id}\}.
\]

Since $\{\pm \text{Id}\}$ is the center of $O(n,2)$ (hence normal), $\A^{n,1}$ (when endowed with the Lorentzian metric induced by the quotient) has maximal isometry group by Proposition \ref{classification} and is therefore a model of constant sectional curvature $-1$. It can also be shown that the center of the isometry group of the Klein model is trivial, hence $\A^{n,1}$ is the \textit{minimal} model of AdS geometry, in the sense that any other model is a covering of $\A^{n,1}$.\\
By definition $\A^{n,1}$ is naturally identified with a subspace of real projective space $\R \text{P}^{n+1}$, more explicitly with the subset of timelike directions of $\R^{n,2}$: 
\[
    \A^{n,1}=\{[x]\in \R \text{P}^{n+1}\;|\;q_{n,2}(x)<0\}.
\]
Like in hyperbolic geometry, the boundary of $\A^{n,1}$ in projective space is a quadric of signature $(n,2)$, that is the projectivization of the of lightlike vectors in $\R^{n,2}$. We denote this quadric by $\partial \A^{n,1}=\{[x]\in \R \text{P}^{n+1}\;|\;q_{n,2}(x)=0\}$.\\ We observe that isometries of $\A^{n,1}$ induce projective transformations which preserve $\partial \A^{n,1}$.  \\

\textit{The conformal Lorentzian structure of the boundary.}
We want to continue to develop analogy with hyperbolic geometry and equip $\partial \A^{n,1}$ with a conformal Lorentzian structure that extends the conformal Lorentzian structure of $\A^{n,1}$, just like we have a conformal \textit{visual boundary} in hyperbolic geometry. \\
A point $\ell \in\R\text{P}^{n+1}$ is identified with $\text{Span}(x)$ for some $x\in \R^{n,2},$ and the tangent space of real projective space has the canonical identification 

%This comes directly from Bonsante-Seppi
%where $S_{[x]}=\mathrm{Span}(x)$, while $Q_{[x]}=\R^{n+2}/S_{[x]}$.
%Indeed given $f\in\mathrm{Hom}(S_{[x]}, \R^{n+2})$ such that  $f(x)\neq -x$,  the point $[x_f]=[f(x)+x]$ does not depend on the choice of the representative $x$.
%In this way we define a natural map  $\Pi:\mathcal U_0\to\RP^{n+1}$, where $\mathcal U_0$ is a neighborhood of $0$ in $\mathrm{Hom}(S_{[x]}, \R^{n+2})$.
%The differential of $\Pi$ at $0$ is a surjetcive and its kernel is $\mathrm{Hom}(S_{[x]}, S_{[x]})$. Thus $d_0\Pi$ yields the stated identification
%$T_{[x]}\RP^{n+1}\cong\mathrm{Hom}(S_{[x]}, \R^{n+2})/\mathrm{Hom}(S_{[x]}, S_{[x]})\cong \mathrm{Hom}(S_{[x]}, Q_{[x]})$.

\[
    T_{\ell}\R\text{P}^{n+1}\simeq \text{Hom}(\ell,\R^{n+2}/\ell). 
\]

Now if $\ell$ is timelike we can identify $\R^{n+2}/\ell$ with $\ell^{\perp}.$ For any given local section $\sigma:\A^{n,1}\to \R^{n,2}$ of the projection $\R^{n,2}\to \A^{n,1}$, we can define a Lorentzian metric on $T\A^{n,1}$ setting:

\[
    \langle \langle f,g \rangle \rangle_{\sigma}=\langle f((\sigma[x])),g(\sigma[x])\rangle_{n,2}. 
\]

for $f,g\in T_{x}\A^{n,1}\simeq \text{Hom}(\ell,\ell^\perp$). If the section $\sigma$ has image in $\H^{n,1}$, then the aforementioned metric coincides with the pull-back of the metric over $\R^{n,2}$, since the differential of $\sigma $ identifies $T_{[x]}\A^{n,1}=T_x\H^{n,1}=x^\perp$. For a general section the identification does not hold, but we can recover a conformal metric via the formula: 
\begin{equation}\label{23}
    \langle \langle f,g \rangle \rangle_{\lambda\sigma}=\lambda^2\langle \langle f,g \rangle \rangle_{\sigma}.
\end{equation}
for any function $\lambda$.\\
We consider now the case that $\ell=\text{Span}(x)$ is lightlike, i.e. $q_{n,2}(x)=0$. In this case we can not induce any natural metric on $\R^{n,2}/\ell$. However, we let 
\[
    \mathbb{L}=\{x\in \R^{n,2}\;|\;q_{n,2}(x)=0\}
\]
be the space of lightlike vectors, then $T_x\mathbb{L}$ is precisely $\ell^{\perp}$ and contains $\ell$ itself. We have a canonical identification: $T_\ell\partial\A^{n,1}\simeq\text{Hom}(\ell,\ell^{\perp }/\ell).$ The bilinear for of $\R^{n,2},$ when restricted to $\ell^{\perp},$ induces a non degenerate bilinear form (of signature $(n-1,1)$) on $\ell^\perp/\ell.$ We will denote such a restriction as $\langle v,w\rangle_{\ell^{\perp}/\ell}.$ \\
We can now define a metric on $\partial\A^{n,1}$ for any section $\sigma:\partial\A^{n,1}\to \mathbb{L}$ of the canonical projection by the formula:
\begin{equation}\label{24}
    ((f,g))_{\sigma}=\langle f(\sigma[x]),g(\sigma[x])\rangle_{\ell^{\perp}/\ell},
\end{equation}
for all $f,g\in \text{Hom}(\ell, \ell^\perp/\ell).$ This metric can be viewed as the pull-back of the metric: 
\begin{equation}\label{25}
    ((f,g))_\sigma=\langle \sigma_*(f), \sigma_{\ast} (g)\rangle_{n,2}, 
\end{equation}
since the degenerate metric on $T_x\mathbb{L}=\ell^\perp$ is, by construction, the pull-back of the metric of $\ell^{\perp}/\ell$ by the projection along the degenerate direction $\ell$.\\ 
The relation valid for the metric on $\A^{n,1}$ also holds for the metric on $\partial\A^{n,1}$, that is: 
\begin{equation}\label{26}
    ((f,g))_{\lambda\sigma}=\lambda^2((f,g))_{\sigma}
\end{equation} 
and therefore the induced conformal class over $T\partial\A^{n,1}$ is well defined and independent of the choice of $\sigma$ and equips the tangent space of the boundary with a conformal Lorentzian metric. To conclude let us show that this conformal metric is naturally the compactification of $\A^{n,1}$. Let $\sigma$ be a section, of the projection $\pi:\R^{n,2}\to \R\text{P}^{n+1},$ defined in a neighborhood $U$ of a point in $\partial\A^{n,1}$, by construction the metric $((\cdot,\cdot))_\sigma$ over $\partial\A^{n,1}\cap U$ is the limit of the conformal metric associated to $\sigma$ defined over $\A^{n,1}\cap U$. % This means that if $(p_n,v_n)$ is a sequence in $T\A^{n,1}$ that converges to $(p_{\infty}, v_\infty)$ then 
\begin{observation}\label{222}
Let's make some observation about the light cone in the case of $\partial\A^{n,1}$. If $[y]\in\partial\A^{n,1}$ Equation \refeq{25} implies that the lightlike vectors in $T_{[y]}\partial\A^{n,1}$ are exactly the projection of vectors $x\in \R^{n,2}$ such that $\langle x,y\rangle_{n,2}=0$ and $q_{n,2}(x)=0$. These vectors are such that $\text{Span}(x,y)$ are totally degenerate planes in $\R^{n,2}$, equivalently are projective lines contained in $\partial\A^{n,1}$. The light cone in $\partial\A^{n,1}$ through $[y]$ is therefore the union of all the projective lines through $[y]$ that are contained in $\partial\A^{n,1}$.

\end{observation}


\subsection{The Poincaré model for the universal cover.}
We have already observed that $\H^{n,1}$, and its double quotient $\A^{n,1}$, are not simply connected. We want to construct a simply connected model for AdS geometry. For this purpose we introduce the universal cover of $\H^{n,1}$ and $\A^{n,1}$.\\
Let $\H^n$ be the hyperboloid model of hyperbolic space. Then: 
\begin{equation}\label{ogcover}
    \pi(y,t)=(y_1,\dots,y_n,y_{n+1}\cos t,y_{n+2}\sin t)
\end{equation}


defines a map $\pi:\H^n\times \R\to \H^{n,1}$ which is a covering with deck transformations of the form $(y,t)\mapsto (y,t+2k\pi)$ for $k\in \Z$. We denote the covering space by $\AS^{n,1}$ and we observe that it is also the universal cover of $\A^{n,1}$, where the covering map it is just the composition of $\pi$ and the double quotient.\\
Pulling back the Lorentzian metric over $\AS^{n,1}$ we get a simply connected Lorentzian manifold of constant curvature -1. The metric on $\AS^{n,1}$ is a warped product of the form: 
\begin{equation}\label{metric}
     \pi^*g_{\H^{n,1}}=g_{\H^n}-y_{n+1}^{2}dt^2.
\end{equation}
   

Moreover $\AS^{n,1}$ has maximal isometry group, hence we have obtained a simply connected model for AdS geometry. More precisely we have a central extension, that is a (non split) short exact sequence:
\[
    0\to \Z\to \textnormal{Isom}(\AS^{n,1})\to O(n,2)\to 1.
\]
It is sometimes convenient to express the metric \ref{metric} using the Poincaré model of hyperbolic space. Recall that the disk model of the hyperbolic space is the unit disk $\D^n$ endowed with the conformal metric: $\frac{4}{(1-r^2)^2}\sum dx_i^2$, where $r^2=\vert x\vert^2$. In our setting the isometry between the disk and the hyperboloid model is given by: 
\begin{equation}
    (x_1,\dots, x_n)\mapsto (y_1=\frac{2x_1}{1-r^2},\dots, y_n=\frac{2x_n}{1-r^2},y_{n+1}=\frac{1+r^2}{1-r^2}) 
\end{equation}
The Poincaré model of AdS geometry is then the cylinder $\D^n\times \R$ endowed with the metric 
\begin{equation}\label{metricdisk}
    \frac{4}{(1-r^2)^2}(dx_1^2+\dots+dx_n^2)-(\frac{1+r^2}{1-r^2})^{2}dt^2 
\end{equation}

It follows from the definition that each slice $\{t=c\}$ is a totally geodesic copy of $\H^n$. The metric defined in \refeq{metricdisk} also shows that the vector field $\partial/\partial t$ is a timelike non vanishing vector field on $\AS^{n,1}$, giving to $\AS^{n,1}$ the structure of a time-orientable manifold. Any choice of time orientation is preserved by the action of the deck transformation of the covering from the Poincaré to the Klein model, hence both $\H^{n,1}$ and $\A$ are time-orientable. \\
\begin{observation}
    Exception made for the central line passing through $x_1=\dots=x_n=t=0,$ vertical lines are not geodesic for the metric defined in \refeq{metricdisk}.  
\end{observation}

\textit{The conformal metric of the boundary.} As we did with $\A^{n,1}$ we would like to give a conformal Lorentzian structure to the boundary of $\AS^{n,1}$. The construction actually works for any covering space of $\A^{n,1}.$ The covering map (we are now considering the projective model of $\H^n$)
\[
    \pi^{\prime} ([y_1:\dots:y_{n}:y_{n+1}],t)=[y_1:\dots:y_n:y_{n+1}\cos t:y_{n+1}\sin t]
\]

extends to a map $\pi^{\prime}:\H^n\cup \partial\H^n\times \R\to \A^{n,1}\cup \partial\A^{n,1}$. In order to compute the conformal structure of the boundary, we consider the map $\tau: \H^n\times \R\to\R^{n+2}$ defined by:
\[
    \tau([y_1:\dots:y_n:y_{n+1}],t)=(y_1/y_{n+1},\dots,y_n/y_{n+1},\cos t,\sin t)
\]  
 that extends to the boundary and induces a (local) section of the projection $\R^{n,2}\to \A^{n,1}$. In fact, let $\eta$ be the generator of the group of deck transformation of the covering $\pi^{\prime}:\AS^{n,1}   \to \A^{n,1}$, then $\tau$ has the equivariance $\tau\circ\eta^i=(-1)^i\tau.$   Using Equation \refeq{26}, the conformal Lorentzian metric on $\partial\AS^{n,1}$ induced by $\sigma$ by means of Equation \refeq{24} is the pull-back of a Lorentzian metric compatible with the natural conformal structure of the boundary $\partial\A^{2,1}$. Using the metric introduced in equation \refeq{metricdisk}, the formula \refeq{23} and by observing that $\tau$ differs by the hyperboloid section by the factor $y_{n+1}=\frac{1+r^2}{1-r^2}$ we obtain the expression 

 \begin{equation}
    \frac{4}{(1+r^2)^2}(dx_1^{2}+\dots+dx_n^2)-dt^2.
 \end{equation}

 This metric extends to $\overline{\D}\times\R$ and thus the metric $g_{\mathbb{S}^{n-1}}-dt^2$ on $\mathbb{S}^{n-1}\times\R,$ where $g_{\mathbb{S}^{n-1}}$ is the round metric over the sphere, is compatible with the conformal Lorentzian structure of $\partial\AS^{n,1}$.\\ 
 This also shows that the conformal structure of $\partial H^{n,1}\simeq\mathbb{S}^{n-1}\times\mathbb{S}^1$ admits the representative $g_{\mathbb{S}^{n-1}}-g_{\mathbb{S}^1}$, and the conformal structure of $\partial\A^{n,1}$ is compatible with the double quotient of the latter, by the involution $(p,q)\mapsto (-p,-q)$ on $\mathbb{S}^{n-1}\times\mathbb{S}^1$.\\
\subsection{Geodesic.}
We have presented our various models as manifolds and now we would like to improve our knowledge of their geometry. As always we start by studying and characterizing geodesics.\\ 

\textit{In the quadric model} Let us start with the exponential map in the hyperboloid model. Given a point $x\in \H^{n,1}$ and a vector $v\in T_x\H^{n,1}$, we want to determine the geodesic through $x$ with speed $v$. We will distinguish several cases according to the sign of $ q_{n,2}(v)$. If $v$ is lightlike, then: 
\[
    \gamma(t)=x+tv
\] is a geodesic of $\R^{n,2}$ and is contained in $\H^{n,1}$, hence $\gamma$ is a geodesic of $\H^{n,1}$. If $v$ is either timelike or spacelike, we claim that the geodesic $\gamma(t)=\exp_x(tv)$ is contained in the linear plane $W=\text{Span}(x,v).$ In fact, the linear transformation $T$ that fixes pointwise $W$ and whose restriction to $W^\perp$ is $-\text{Id}_{W^\perp}$ is in $O(n,2)$. By the uniqueness of the geodesic, $T\circ\gamma=\gamma$ hence $\gamma$ is contained in $\H^{n,1}\cap W$. We can easily derive the expressions\\
\begin{equation}
    \gamma(t)=\cosh(t)x+\sinh(t)v
\end{equation}

if $q_{n,2}(v)=1$ and 
\begin{equation}\label{212}
    \gamma(t)=\cos(t)x+\sin(t)v
\end{equation}
if $q_{n,2}(v)=-1$.\\

\textit{In the Klein model.} In analogy with the Riemannian case, in the Klein model $\A^{n,1}$ geodesics are intersections of projective lines with the domain of $\A^{n,1}\subset\R \text{P}^{n+1}.$ From what we have already said: 
\begin{itemize}
    \item Timelike geodesics correspond to projective lines that are entirely contained in $\A^{n,1}$, are closed non-trivial loops and have length $\pi.$ 
    \item Spacelike geodesics correspond to lines that meet $\partial \A^{n,1}$ transversally in two points. They have infinite length.
    \item Lightlike geodesics correspond to lines tangent to $\partial\A^{n,1}$.   
\end{itemize}

In particular the light cone through a point $[x]\in \A$ coincides with the cone of lines through $[x]$ tangent to $\partial \A$.\\ For instance in the affine chart $\mathbb{A}_{n+2}\neq\{x_{n+2}=0\},$ where in coordinates $(y_1, \dots, y_{n+1})=(x_1/x_{n+2},\dots, x_{n+1}/x_{n+2})$, the intersection $\A\cap\mathbb{A}_{n+2}$ is the interior of a one sheeted hyperboloid, that is: 
\[
    \A\cap\mathbb{A}_{n+2}=\{y_1^2+\dots+y_n^2-y_{n+1}^2<1\},
\]
 while the boundary it's the one-sheeted hyperboloid itself: 
 \[
    \partial\A\cap\mathbb{A}_{n+2}=\{y_1^2+\dots+y_n^2-y_{n+1}^2<1\}.
\]
In an affine chart, timelike geodesic corresponds to affine lines which are entirely contained in the Anti de Sitter space, and which are not asymptotic to its boundary; lightlike geodesics are tangent to the one sheeted hyperboloid, or are asymptotic to it (tangent at infinity).

\textcolor{red}{qua magari inserire una immagine del modello tridimensionale}

%\begin{observation} Any timelike line is the projectivisation of a negative definite plane. As Isom$(\A^{n,1})\simeq \text{PO}(n,2)$ acts transitively on the space of timelike lines, and since the stabilizer of a timelike line is the group $\text{P}(O(n)\times O(2))$ which is the maximal compact subgroup of $\text{PO}(n,2)$, the space of timelike geodesics of $\A^{n,1}$ is naturally identified with the Riemannian symmetric space of PO$(n,2)$. \textcolor{red}{è una curiosità, non viene favvero utilizzato nel seguito, si potrebbe togliere.}
%\end{observation}

\noindent\textit{Totally geodesic subspaces} Totally geodesic subspaces of $\A^{n,1}$ of dimension $k$ are obtained as the intersection of $\A^{n,1}$ with the projectivisation $P(W)$ of a linear subspace $W$ of $\R^{n,2}$ of dimension $k+1.$ The negative index of $W$ can be $1, 2,$ for otherwise the intersection $\A^{n,1}\cap P(W)$ would be empty. We have several cases: 
\begin{itemize}
    \item If $W$ has signature (k-1,2), then $P(W)\cap \A^{n,1}$ is isometric to $\A^{k-1,1}$. 
    \item If $W$ has signature $(k-2, 1)$ , then it is a copy of Minkowski space $\R^{k-2,1}$, hence $P(W)\cap \A^{n,1}$ is a copy of the Klein model of hyperbolic space. 
    \item If $W$ is degenerate, then $P(W)\cap \A^{n,1},$ is a lightlike subspace foliated by lightlike geodesics tangent to the same point of $\partial\A^{n,1}$. 
\end{itemize}  

A particular case of the last point is when $W$ is degenerate and $\text{dim}(W)=n+1.$ Then $P(W)\cap \A^{n,1}$ is a projective hyperplane tangent to $\partial\A$ at a point $[x]$ and $P(W)\cap \partial\A^{n,1}$ is the lightlike cone of $\partial\A^{n,1}$ through $[x]$.\\

\textit{In the universal cover.} In the universal cover $\AS^{n,1},$ geodesics are just the lifts of geodesics in $\A^{n,1}$ or $\H^{n,1}$. Hence every spacelike or lightlike geodesic in $\A^{n,1}$ and $\H^{n,1}$, which is topologically a line, has a countable number of lifts to $\AS^{n,1}$. Timelike geodesics in $\A^{n,1}$ and $\H^{n,1}$ are topologically circles and are in bijections with timelike geodesics in $\AS^{n,1},$ as the covering map restricted to a timelike geodesic, induces a covering map onto the circle. Using the Poincaré model we can give an explicit description of lightlike geodesic. In fact, in Lorentzian geometry not only the nature of a vector is conformally invariant but also unparametrized lightlike geodesics are a conformal property (\cite{Gallot}, Proposition 2.131): 
\begin{theorem}\label{ConformalMetric} If two Lorentzian metrics $g,g^{\prime} $ on a manifold $M$ are conformal, then they have the same unparametrized lightlike geodesics.
\end{theorem}

Because of Theorem \ref{ConformalMetric} we can replace the Poincaré metric by the conformally equivalent metric given by:
\begin{equation}\label{emispherical}
    \frac{4}{(1+r^2)^2}(dx_1^2+\dots+dx_n^2)-dt^2
\end{equation} 
Now we observe that the first term in Equation \ref{emispherical} is exactly the form of the spherical metric on a hemisphere, pulled-back to the unit disk by the stereographic projection. We will call such a metric hemispherical and will denote it by $g_{\mathbb{S}^n}$. Notice that the boundary of $\partial\D$ is an equator for the hemispherical metric, and in fact it's the only equator completely contained in $(\D\cup\partial\D,g_{\mathbb{S}^n})$, a justification to the fact that we will refer to it as \textit{the} equator.\\
As a consequence, unparametrized lightlike geodesic of $\AS^{n,1}$, going through a point $(p_0,t_0)$ are characterized by the conditions that they are mapped to spherical geodesic under the vertical projection $(p,t)\to p$ and moreover: 
\[
    t-t_0=d_{\mathbb{S}^n}(p,p_0)
\] on the geodesic. In particular, these lightlike geodesic meet the boundary of $\AS^{n,1}$ at the point that satisfies the above conditions such that $p$ is on the equator of the hemisphere: as an example, if $p_0$ is the center of the hemisphere, then the points at infinity of the lightcone over $(p_0,t_0)$ are the horizontal slice $t=t_{0}+\pi/2.$ This sphere is also the boundary of a hyperplane dual to $(p_0,t_0)$, in a sense that we will explain in the following section. \\
By an analogous reasoning we can give an explicit description of a lightlike hyperplane in the Poincaré model: the lightlike plane having $(p_0,t_0)$ as a past endpoint, (where now $p_0$ is on the equator) is precisely $\{(p,t)\;|\;t-t_0=d_{\mathbb{S}^n}(p,p_0)\}$ and its future endpoint is $(-p_0,t+\pi.)$ 

\textcolor{red}{anche qua ci sarebbero le figure da inserire, imho queste sono effettivamente utili}

\section{Polarity in Ads}\label{polarsec}
The quadratic form $q_{n,2}$ induces a polarity on the projective space $\R\text{P}^{n+1}$, explicitly the correspondence associates to a projective subspace $P(W)$ the subspace $P(W^\perp)$. In particular, we have an induced duality between spacelike totally geodesic subspaces of $\A^{n,1}$ where the dual of a spacelike $k-$dimensional subspaces is an $n-k+1$ subspace. \\
For instance, if we consider the dual of a point $[x]\in\A^{n,1}$ it will be an $n-$dimensional spacelike hyperplane $P_{[x]}=\text{P}(x^\perp).$ Projectively $P_{[x]}$ is characterized as the hyperplane spanned by the intersection of $\partial\A^{n-1,1}$ with the lightcone from $[x]$. More geometrically, it can be checked that $P_{[x]}$ is the set of antipodal point to $[x]$ along timelike geodesics through $[x].$ Also, every timelike geodesic through $[x]$ meets $P_{[x]}$ orthogonally at time $\pi/2.$ Conversely, given a totally geodesic spacelike hyperplane $H$, all the timelike geodesic that meet $H$ orthogonally intersect in a single point, which is the dual point of $H$.

\subsection*{In the quadric model} We would like to \textit{lift} the idea of duality to coverings of $\A^{n,1}.$ Observe that in $\H^{n,1}$ there are two dual planes associated to any point $x$, namely the sets: 
\[
    P_x^\pm=\{\exp_x(\pm(\pi/2)v)\;|\;q_{n,2}(v)=-1,\; v\;\text{future-directed}\}.
\] 

Now the points $P_x^+$ and $P_x^-$ are antipodal and $P_{-x}^\pm=P_x^\mp$. The planes $P_x^\pm$ disconnect $\H^{n,1}$ in two regions $U_x$ and $U_{-x},$ where $U_x$ is the connected component of $x$. They can be characterized as: 
\[
    U_x=\{y\in\H^{n,1}\;|\;\langle x,y\rangle_{n,1}<0\}.
\]

Spacelike and lightlike geodesic through $x$ do not exit $U_x$, while all the timelike geodesics through $x$ meet orthogonally $P_x^\pm$ and all pass through the point $-x$. More precisely a point $y\neq x$ is connected to $x$: 
\begin{itemize}
    \item by a spacelike geodesic if and only if $\langle x,y\rangle_{n,1}<-1,$
    \item by a lightlike geodesic if and only if $\langle x,y\rangle_{n,1}=-1,$
    \item by a timelike geodesic if and only if $|\langle x,y\rangle_{n,1}|<1.$ 
\end{itemize}

An immediate consequence is that if $y$ is connected to $x$ by a spacelike geodesic, there is no geodesic joining $y$ to $-x$. Hence the exponential map of $\H^{n,1}$ is not surjective. But as any point $y\in\H^{n,1}$ can be connected through a geodesic either to $x$ or $-x$ hence the exponential over $\A^{2,1}$ is surjective.
\subsection{In the universal cover.}
Recall that $\AS^{n,1}\simeq \H^n\times\R$ with the group of deck transformations being isomorphic to $\Z$, where a generator acts by translation of $2\pi$ in the $\R$ factor. It follows that the preimage of a spacelike plane $P\subset\A^{n,1}$ is the disjoint union of spacelike planes $(P^k)_{k\in\Z},$ enumerated so that the generator $\eta$ of $\Z$ acts by sending $P^k$ to $P^{k+1}.$ \\
Each connected component of $\AS^{2,1}\setminus\bigcup_{k\in\Z} P^k$ is a fundamental domain for the action of deck transformation of the covering $\AS^{n,1}\to\A^{n,1}.$ Given a point $x$, let us apply the previous construction to the plane $P_x=P_{\pi^{\prime}(x)}$ which is the dual of the image $\pi^{\prime}(x)$ in $\A^{n,1}$, and let $V_x$ be the connected component which contains $x$. We will refer to $V_x$ as the \textit{Dirichlet domain} in $\AS^{2,1}$ centered at $x$, since the construction of $V_x$ is the analogue of a Dirichlet domain to our context. Then the restricted covering map: $\restr{\pi^{\prime}}{V_x}:V_x\to\A^{2,1}\setminus P_x$ is an isometry. Therefore lightlike and spacelike geodesic through $x$ are entirely contained in $V_x$.\\ 

\textcolor{red}{Anche qua le immagini sono utili }