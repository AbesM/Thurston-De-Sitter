\chapter{Model of Anti-de Sitter (n+1)-space}
\section{Anti-de Sitter geometry}
We want to construct models of Lorentian manifolds with constant sectional curvature -1 and maximal isometry group in any dimension. We are also interested in \textit{stressing} the analogies with the models of hyperbolic space in the Riemannian setting. 

\subsection{The quadric model}
We want to introduce the analogue of the hyperboloid model of hyperbolic space. Denote by $\R^{n,2}$ the real vector space $\R^{n+2}$ equipped with the quadratic form 
\[
    q_{n,2}(x)=x_1^{2}+\dots+x_n^{2}-x_{n+1}^{2}-x_{n+2}^2.   
\]
and by $<v,w>_{n,2}$ the associated symmetric form. Finally let $O(n,2)$ be the group of linear transformation of $\R^{n+2}$ that preserve $q_{n,2}.$
Then we define: 
\[
    \H^{n,1}=\{x\in \R^{n,2}\;|\;q_{n,2}(x)=-1\}.
\]

It is immediate to check that $\H^{n,1}$ is a smooth connected submanifold of $\R^{n,2}$ of dimensione $n+1$. The tangent space $T_x\H^{n+1}$, regarded as a subspace of $\R^{n+2}$ coincides with the orthogonal space $x^{\perp}=\{y\in \R^{n+2}\;|\;<x,y>_{n,2}=0\}.$ A simple signature argument shows that the restricion of the symmetric form $\langle .,. \rangle_{n,2}$ to $T\H^{n+1}$ has Lorentzian signature, hence it makes $\H^{n,1}$ a Lorentzian manifold. We remark that this model is the analogue of the hyperboloid model of hyperbolic space, in fact $\H^n$ is isometrically embedded in $\H^{n,1}$ as the submanifold defined by $x_{n+2}=0$, $x_{n+1}>0$. \\
The natural action of $O(n,2)$ on $\R^{n,2}$ preserves $\H^{n,1}$, and in facts $O(n,2)$ acts by isometries on $\H^{n,1}$. We remark that $O(n,2)$ acts transitively on $\H^{n,1}$ and that the stabilizer of a point $x$ acts transitively on the space of orthonormal bases of $T_x\H^{n,1}$. Hence $\H^{n,1}$ has maximal isometry group and it is identified with $O(n,2)$.\\
By \ref{maximalisometry}, $\H^{n,1}$ has constant sectional curvature. Let us now check taht the sectional curvature is negative (actually, $K=-1$). For this purpose, observe that the nromal line in $\R^{n,2}$ to $\H^{n,1}$ at $x$ is identified with the line generated by $x$ itself. It follows that, if $v,w$ are tangent vector fields along $\H^{n,1}$ we have the orthogonal decomposition: 
\[
    (D_{v}w)(x)=(\nabla_{v}w)(x)+\langle v,w\rangle x, 
\]

where $D$ is the flat connection of $R^{n+2}$ and $\nabla$ is the Levi-Civita connection of $\H^{n,1}$. Using the flatness of $D$ we get: 
\[
    R(u,v)w=\langle u,w\rangle v-\langle v,w\rangle u,
\]

so that 
\[
    <R(u,v)v,u\rangle =-(\langle u,u\rangle\langle v,v\rangle-\langle v,u\rangle^2), 
\]

and this shows that $\H^{n,1}$ has constant sectional curvature $-1$. We also remark that $H^{n,1}$ is not simply connected, being homeomorphic to $\R^{n}\times S^1$. \todo{It's a rotation manifold (circa)}

\subsection{The "Klein model" and it's boundary.}
Let us introduce a projective mode, or "Klein model", for anti-De Sitter geometry. Let us define: 
\[
    \A^{n,1}=\H^{n,1}/\{\pm \mathds{1}\}.
\]

Since $\mathds{1}$ is the center of $O(n,2)$ (hence normal), $\A^{n,1}$ (endowed with the Lorentzian metric induce by the quotient) has maximal isometry group by \ref{classification} and is therefore a model of constant sectional curvature $-1$. It can also be shown that the center of the isometri group of the Klein model is trivial, hence $\A^{n,1}$ is the \textit{minimal} model of AdS geometry, in the sense that any other model is a covering of $\A^{n,1}$.\\
By definition $\A^{n,1}$ is naturally identified with a subspace of real projective space $\R P^{n+1}$, more explicitely with the subset of timelike directions of $\R^{n,2}$: 
\[
    \A^{n,1}=\{[x]\in \R P^{n+1}\;|\;q_{n,2}(x)<0\}.
\]
Like in hyperbolic geometry, the boundary of $\A^{n,1}$ in projective space is a quadric of signature $(n,2)$, that is the projectivization of the of lightlike vectors in $\R^{n,2}$. We denote this quadric by $\partial \A^{n,1}=\{[x]\in \R P^{n+1}\;|\;q_{n,2}(x)=0\}$.\\ We observe that isometries of $\A^{n,1}$ induce projective transformations which preserve $\partial \A^{n,1}$.  \\

\textit{The conformal Lorentzian structure of the boundary.}

\subsection{The Poincaré model for the universal cover.}

\subsection{Geodesic.}

\section{Anti-de Sitter space in dimension (2+1)}