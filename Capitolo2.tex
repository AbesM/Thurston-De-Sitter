\chapter{Model of Anti-de Sitter (n+1)-space}
\section{Anti-de Sitter geometry}
We want to construct models of Lorentian manifolds with constant sectional curvature -1 and maximal isometry group in any dimension. We are also interested in \textit{stressing} the analogies with the models of hyperbolic space in the Riemannian setting. 

\subsection{The quadric model}
We want to introduce the analogue of the hyperboloid model of hyperbolic space. Denote by $\R^{n,2}$ the real vector space $\R^{n+2}$ equipped with the quadratic form 
\[
    q_{n,2}(x)=x_1^{2}+\dots+x_n^{2}-x_{n+1}^{2}-x_{n+2}^2.   
\]
and by $<v,w>_{n,2}$ the associated symmetric form. Finally let $O(n,2)$ be the group of linear transformation of $\R^{n+2}$ that preserve $q_{n,2}.$
Then we define: 
\[
    \H^{n,1}=\{x\in \R^{n,2}\;|\;q_{n,2}(x)=-1\}.
\]

It is immediate to check that $\H^{n,1}$ is a smooth connected submanifold of $\R^{n,2}$ of dimensione $n+1$. The tangent space $T_x\H^{n+1}$, regarded as a subspace of $\R^{n+2}$ coincides with the orthogonal space $x^{\perp}=\{y\in \R^{n+2}\;|\;<x,y>_{n,2}=0\}.$ A simple signature argument shows that the restricion of the symmetric form $\langle .,. \rangle_{n,2}$ to $T\H^{n+1}$ has Lorentzian signature, hence it makes $\H^{n,1}$ a Lorentzian manifold. We remark that this model is the analogue of the hyperboloid model of hyperbolic space, in fact $\H^n$ is isometrically embedded in $\H^{n,1}$ as the submanifold defined by $x_{n+2}=0$, $x_{n+1}>0$. \\
The natural action of $O(n,2)$ on $\R^{n,2}$ preserves $\H^{n,1}$, and in facts $O(n,2)$ acts by isometries on $\H^{n,1}$. We remark that $O(n,2)$ acts transitively on $\H^{n,1}$ and that the stabilizer of a point $x$ acts transitively on the space of orthonormal bases of $T_x\H^{n,1}$. Hence $\H^{n,1}$ has maximal isometry group and it is identified with $O(n,2)$.\\
By \ref{maximalisometry}, $\H^{n,1}$ has constant sectional curvature. Let us now check taht the sectional curvature is negative (actually, $K=-1$). For this purpose, observe that the nromal line in $\R^{n,2}$ to $\H^{n,1}$ at $x$ is identified with the line generated by $x$ itself. It follows that, if $v,w$ are tangent vector fields along $\H^{n,1}$ we have the orthogonal decomposition: 
\[
    (D_{v}w)(x)=(\nabla_{v}w)(x)+\langle v,w\rangle x, 
\]

where $D$ is the flat connection of $R^{n+2}$ and $\nabla$ is the Levi-Civita connection of $\H^{n,1}$. Using the flatness of $D$ we get: 
\[
    R(u,v)w=\langle u,w\rangle v-\langle v,w\rangle u,
\]

so that 
\[
    <R(u,v)v,u\rangle =-(\langle u,u\rangle\langle v,v\rangle-\langle v,u\rangle^2), 
\]

and this shows that $\H^{n,1}$ has constant sectional curvature $-1$. We also remark that $H^{n,1}$ is not simply connected, being homeomorphic to $\R^{n}\times S^1$. \todo{It's a rotation manifold (circa)}

\subsection{The "Klein model" and it's boundary.}
Let us introduce a projective mode, or "Klein model", for anti-De Sitter geometry. Let us define: 
\[
    \A^{n,1}=\H^{n,1}/\{\pm \mathds{1}\}.
\]

Since $\mathds{1}$ is the center of $O(n,2)$ (hence normal), $\A^{n,1}$ (endowed with the Lorentzian metric induce by the quotient) has maximal isometry group by \ref{classification} and is therefore a model of constant sectional curvature $-1$. It can also be shown that the center of the isometri group of the Klein model is trivial, hence $\A^{n,1}$ is the \textit{minimal} model of AdS geometry, in the sense that any other model is a covering of $\A^{n,1}$.\\
By definition $\A^{n,1}$ is naturally identified with a subspace of real projective space $\R P^{n+1}$, more explicitely with the subset of timelike directions of $\R^{n,2}$: 
\[
    \A^{n,1}=\{[x]\in \R P^{n+1}\;|\;q_{n,2}(x)<0\}.
\]
Like in hyperbolic geometry, the boundary of $\A^{n,1}$ in projective space is a quadric of signature $(n,2)$, that is the projectivization of the of lightlike vectors in $\R^{n,2}$. We denote this quadric by $\partial \A^{n,1}=\{[x]\in \R P^{n+1}\;|\;q_{n,2}(x)=0\}$.\\ We observe that isometries of $\A^{n,1}$ induce projective transformations which preserve $\partial \A^{n,1}$.  \\

\textit{The conformal Lorentzian structure of the boundary.}

\subsection{The Poincaré model for the universal cover.}
We have already observed that $\H^{n,1}$, and its double quotient $\A^{n,1}$, are not simply connected. We want to construct a simply connected model for AdS geometry. For this purpose, we introduce the universal cover of $\H^{n,1}$ and $\A^{n,1}$.\\
Let $\H^n$ be the hyperboloid model of hyperbolic space. Then: 
\[
    \pi(y,t)=(y_1,\dots,y_n,y_{n+1}\cos t,y_{n+2}\sin t)
\]

defines a map $\pi:\H^n\times \R\to \H^{n,1}$ which is a covering with deck transformation of the form $(y,t)\mapsto (y,t+2k\pi)$ for $k\in \Z$. We denote the space by $\AS^{n,1}$ and we observe that it is also the universal cover of $\A^{n,1}$, where the covering map it's just the composition of $\pi$ and the double quotient.\\
Pulling back the Lorentzian metric over $\AS^{n,1}$ we get a simply connected Lorentzian manifold of constant curvature -1. The metric on $\A{n,1}$ is a \textcolor{red}{warped product} of the form: 
\begin{equation}\label{metric}
     \pi^*g_{\H^{n,1}}=g_{\H^n}-y_{n+1}^{2}dt^2.
\end{equation}
   

Moreover $\AS^{n,1}$ has maximal isometry group, hence we have obtained a simply connected model for AdS geometry. More precisely we have a central extension, that is a (non split) short exact sequence:
\[
    0\to \Z\to \textnormal{Isom}(\AS^{n,1})\to O(n,2)\to 1.
\]
It is convient to express the metric \ref{metric} using the Poincaré model of the hyperbolic space. Recall that the disk model of the hyperbolic space is the unit disk $\D^n$ endowed with the conformal metric: $\frac{4}{(1-r^2)^2}\sum dx_i^2$, where $r^2=\vert x\vert^2$. \todo{inserire l'isometria disco piano}
The Poincaré model of the AdS geometry is then the cylinder $\D^n\times \R$ endowed with the metric \ref{metricdisk}\todo{mettere}.\\
\subsection{Geodesic.}

\textit{In the quadric model} Let us start with the exponential map in the hyperbloid model. Given a point $x\in \H^{n,1}$ and a vector $v\in T_x\H^{n,1}$, we want to determine the geodesic through $x$ with speed $v$. We will distinguish several cases according to the sign of $ q_{n,2}(v)$. If $v$ is lightlike, then: 
\[
    \gamma(t)=x+tv
\] is a geodesic of $\R^{n,2}$ and is cointained in $\H^{n,1}$, hence $\gamma$ is a geodesic of $\H^{n,1}$. If $v$ is etiher timelike or spacelike, we claim that the geodesic $\gamma(t)=\exp_x(tv)$ is contained in the linear plane $W=\text{Span}(x,v).$ In fact, the linear transformation $T$ that fixes pointwise $W$ and whose restriction to $W^\perp$ is $-\1_{W^\perp}$ is in $O(n,2)$. By the uniqueness of the geodesic, $T\circ\gamma=\gamma$ hence $\gamma$ is contained in $\H^{n,1}\cap W$. We can easily derive the expressions\\
\begin{equation}
    \gamma(t)=\cosh(t)x+\sinh(t)v
\end{equation}

if $q_{n,2}(v)=1$ and 
\[
    \gamma(t)=\cos(t)x+\sin(t)v
\]
if $q_{n,2}(v)=-1$.\\

\textit{In the Klein model.} In analogy with the Riemannian case, in the Klein model $\A$ geodesics are intersections of projective lines with the domain of $\A\subset\R P^{n+1}.$ From what we have already said: 
\begin{itemize}
    \item Timelike geodesics correspond to projective lines that are entirely contained in $\A$, are closed non-trivial loops and have lenght $\pi.$ 
    \item Spacelike geodesics correspond to lines that meet $\partial \A$ transversally in two points. They have infinite lenght.
    \item Lightlike geodesics correspond to lines tangent to $\partial\A$.   
\end{itemize}

In particular the light cone through a point $[x]\in \A$ coincides with the cone of lines through $[x]$ tangent to $\partial \A$.\\ For instance in the affine chart $\mathbb{A}_{n+2}\neq\{x_{n+2}=0\},$ where in coordinates $(y_1, \dots, y_{n+1})=(x_1/x_{n+2},\dots, x_{n+1}/x_{n+2})$, the intersection $\A\cap\mathbb{A}_{n+2}$ is the interior of a one sheeted hyperboloid, that is: 
\[
    \A\cap\mathbb{A}_{n+2}=\{y_1^2+\dots+y_n^2-y_{n+1}^2<1\},
\]
 while the boundary it's the one-sheeted hyperboloid itself: 
 \[
    \partial\A\cap\mathbb{A}_{n+2}=\{y_1^2+\dots+y_n^2-y_{n+1}^2<1\}.
\]+
In an affine chart, timelike geodesic corresponds to affine lines which are entirely contained in the Anti de Sitter space, and which are not asymptotic to its boundary; lightlike geodesics are tangent to the one sheeted hyperboloid, or are asymptotic to it (tangent at infinity).

\begin{observation} An important observation concerns the space of timelike geodesics. Any timelike line is the projectivisation of a negative definite plane. As Isom($\A^{n+1})\simeq \text{PO}(n,2)$ acts transitevely on the space of timelike lines, and since the sabiliser of a timelike line is the group $P(O(n)\times O(2))$ which is the maximal compact subgroup of $\text{PO}(n,2)$, the space of timelike geodesics of $\A$^{n,1}$ is naturally identified with the Riemannian symmetric space of PO$(n,2)$.
\end{observation}

\noindent\textit{Totally geodesic subspaces}

\textit{In the universal cover.} In the universal cover $\AS^{n,1},$ geodesics are just the lifts of geodesics in $\A^{n,1}$ or $\H^{n,1}$. Hence every spacelike or lightlike geodesic in $\A^{n,1}$ and $\H^{n,1}$, which is topologically a line, has a countable number of lifts to $\AS^{n,1}$. Timelike geodesics in $\A^{n,1}$ and $\H^{n,1}$ are topologically circles and are in bijections with timelike geodesics in $\AS^{n,1},$ as the covering map restricted to a timelike geodesic, induces a covering map onto the circle. Using the Poincaré model we can give an explicit description of lightlike geodesic.\todo{non chiarissimo cosa stia succedendo qua} In fact, in Lorentzian geometry not only the nature of a vector is conformally invariant but also unparametrized lightlike are a conformal invariant: 
\begin{theorem}\label{ConformalMetric} If two Lorentianz metrics $g,g^{\prime} $ on a manifold $M$ are conformal, then they have the same unparametrized lightlike geodesics.
\end{theorem}
\begin{proof}
    [GHL04, Proposition 2.131]
\end{proof}
Because of \ref{ConformalMetric} we can replace the Poincaré metric by the conformally equivalent metric given by:
\begin{equation}\label{emispherical}
    \frac{4}{(1+r^2)^2}(dx_1^2+\dots+dx_n^2)-dt^2
\end{equation} 
Now we observe that the first term in \ref{emispherical} is exactly the form of the spherical metric on a hemispere, pulled-back to the unit disk by the stereographic projection. We will call such a metric hemispherical and will denote it by $g_{\S^n}$. Observe that the boundary of $\partial\D$ is an equator for the hemispherical metric, and in fact it's the only equator completely cointained in $(\D\cup\partial\D,g_{\S^n})$, a justification to the fact that we will refere to it as \textit{the} equator.\\
As a consequence, unparametrized lightlike geodesic of $\AS^{n,1}$, goint trough a point $(p_0,t_0)$ are characterized by the coditions that they are mapped to spherical geodesic under the vertical projection $(p,t)\to p$ and moreover: 
\[
    t-t_0=d_{\S^n}(p,p_0)
\] on the geodesic. \todo{what does it mean?} In particular, these lightlike geodesic meet the boundary of $\AS^{n,1}$ at the point that satisfies the above conditions such that $p$ is on the equator of the hemisphere: as an example, if $p_0$ is the center of the hemisphere, then the points at infinity of the lightcone over $(p_0,t_0)$ are the horizontal slice $t=t_{0}+\pi/2.$ This sphere is also the boundary of a hyperplane dual to $(p_0,t_0)$, in a sense that we will explain in the following section. 
\textcolor{red}{pagina 16 Bonsante-Seppi}   
