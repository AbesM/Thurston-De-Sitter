\chapter{Model of Anti-de Sitter (n+1)-space}

We want to construct models of Lorentian manifolds with constant sectional curvature -1 and maximal isometry group in any dimension. We are also interested in \textit{stressing} the analogies with the models of hyperbolic space in the Riemannian setting. 

\subsection{The quadric model}
We want to introduce the analogue of the hyperboloid model of hyperbolic space. Denote by $\R^{n,2}$ the real vector space $\R^{n+2}$ equipped with the quadratic form 
\[
    q_{n,2}(x)=x_1^{2}+\dots+x_n^{2}-x_{n+1}^{2}-x_{n+2}^2.   
\]
and by $<v,w>_{n,2}$ the associated symmetric form. Finally let $O(n,2)$ be the group of linear transformation of $\R^{n+2}$ that preserve $q_{n,2}.$
Then we define: 
\[
    \H^{n,1}=\{x\in \R^{n,2}\;|\;q_{n,2}(x)=-1\}.
\]

It is immediate to check that $\H^{n,1}$ is a smooth connected submanifold of $\R^{n,2}$ of dimensione $n+1$. The tangent space $T_x\H^{n+1}$, regarded as a subspace of $\R^{n+2}$ coincides with the orthogonal space $x^{\perp}=\{y\in \R^{n+2}\;|\;<x,y>_{n,2}=0\}.$ A simple signature argument shows that the restricion of the symmetric form $\langle .,. \rangle_{n,2}$ to $T\H^{n+1}$ has Lorentzian signature, hence it makes $\H^{n,1}$ a Lorentzian manifold. We remark that this model is the analogue of the hyperboloid model of hyperbolic space, in fact $\H^n$ is isometrically embedded in $\H^{n,1}$ as the submanifold defined by $x_{n+2}=0$, $x_{n+1}>0$. \\
The natural action of $O(n,2)$ on $\R^{n,2}$ preserves \textcolor{red}{pagina 9}