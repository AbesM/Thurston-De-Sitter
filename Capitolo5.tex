\chapter{Thurston's earthquake theorem}

\begin{definition}
    A geodesic lamination $\lambda$ of $\H^2$ is a collection of disjoint geodesic that foliate a closed subset $X$ of $\H^2$. The set $X$ is called the \textit{support} of $\lambda.$ The geodesic in $\lambda$ are called \textit{leaves} (as in classic foliation terminology). The connected components of $\H^2\;\setminus\;X$ are called \textit{gaps}. The \textit{strata} of $\lambda$ are the leaves and the gaps.
\end{definition}

Consider $\gamma$ an isometry of $\H^2$, the \textit{axis} of the isometry is the geodesic $\ell$ of $\H^2$ connecting the two fixed points of $\gamma$ in $\partial\H^2$. As a consequence we have that the geodesic is preserved by the isometry, when restricted to such a curve $\gamma_\ell:\ell\to\ell$ acts as a translation with respect to any constant speed parametrization of $\ell$.\\
Given $A,B$ subsets of $\H^2,$ we say that a geodesic $\ell$ \textit{weakly separated} $A$ and $B$ if $A$ and $B$ are contained in the closure of different connected components of $\H^2\;\setminus\;\ell.$ 

\begin{definition}
    A \textit{left earthquake} of $\H^2$ is a bijective map $E:\H^2\to\H^2$ such that there exists a geodesic lamination $\lambda$ for which the restriction $E_S$ to any stratum $S$ of $\lambda$ is equal to the restriction of an isometry of $\H^2,$ and for any two strata $S$ and $S^{\prime}$ of $\lambda$ the comparison isometry 
    
    \[
        \text{Comp}(S,S^{\prime})\coloneqq (E_S)^{-1}\circ E_S^{\prime}
    \]

    is the restriction of an isometry of $\H^2$ such that:
    \begin{itemize}
        \item $\gamma$ is different from the identity, unless possibly where one of the two strata $S$ and $S^{\prime}$ is contained in the closure of the other;
        \item when it's note the identity, $\gamma$ is a hyperbolic transformation whose axis $\ell$ weakly separetes $S$ and $S^{\prime};$
        \item moreover, $\gamma$ translates to the left, seen from $S$ to $S^{\prime}$. 
    \end{itemize}
\end{definition}

Let's explain more carefully what we mean with the last condition. Suppose $f:[0,1]\to\H^2$ is a smooth path such that $f(0)\in S,\;f(1)\in S^{\prime}$ and the image of $f$ intersects $\ell$ transversally and exactly at one point $z_0=f(t_0)\in\ell.$ Let $v=f^{\prime}(t_0)\in T_{z_0}\H^2$ be the tangent vector at the intersection point. Let $w\in T_{z_0}\H^2$ be a vector tangent to the geodesic $\ell$ pointing towards $\gamma(z_0).$ Then we say that $\gamma$ translates to the left seen from $S$ to $S^{\prime} $ if $v,w$ is a positive basis of $T_{z_0}\H^2$, for the standard orientation of $\H_2.$\\
We observe that such a condition is independent of the order we order in which we choose $S$ and $S^{\prime},$  if $\text{Comp}(S,S^{\prime})$ translates to the left seen from $S$ to $S^{\prime},$ then $\text{Comp}(S^{\prime} ,S)$ translates to the left seen from $S^{\prime} $ to $S.$ 

%ok ma cosa vuol dire davvero questa cosa?

\begin{observation} The earthquake $E$ is not required to be (and in fact in some case won't be) continuous. For instance this happens when the lamination $\lambda$ is finite, meaning that $\lambda$ is a collection of a finite number of geodesics. 
\end{observation}

Let's consider a first basic example: 

\textit{Esempio}\todo{creare un ambiente ah-hoc} The map 
\[
    E:\H^2\to\H^2
\]

defined by:
\[
  E(z)= \begin{cases}
    z & \text{if Re(z)}<0 \\
    az & \text{if Re(z)}=0 \\
    bz & \text{if Re(z)}>0 \\    
\end{cases}
\]

is a left earthquake if $1<a<b,$ and a right earthquake if $0<b<a<1$. The lamination $\lambda$ that satisfies the definition is composed of a unique geodesic, namely the geodesic $\ell$ corresponding to the imaginary axis. \\
Such a map is clearly not continuous along $\ell$.
Thurston proved (\textcolor{red}{dove?}) that any earthquake map extends continuously to an orientation-preserving homeomorphism of $\partial\H^2$ meaning that there exist a (unique) orientation-preserving homeomorphism $f:\partial\H^2\to\partial\H^2$ such that the map: 
\[
    \begin{cases}
        E(z), &\text{ if z }\in \H^2\\
        f(z), &\text{ if z}\in \partial\H^2  ;\\
        
    \end{cases}
\]
is continuous in any point $\partial(\H^2.)$\\
Then Thurston proved the following theorem, that he called \textit{"geology is transitive}:

\begin{theorem}
    Given any orientation-preserving homeomorphism $f:\partial\H^2\to\partial\H^2,$ there exists a left earthquake map of $\H^2,$ and a right earthquake map, that extends continuously to $f$ on $\partial\H^2.$
\end{theorem}


\textcolor{red}{Da qua metto la convessità per Diaf-Seppi, da vedere come le due nozioni coincidano (se coincidano)}
\section{Affine space and Convexity Notions}
The initial step in our proof is to consider the graph of an orientation-preserving homeomorphism $f:\R\text{P}^1\to\R\text{P}^1$ as a subset of $\partial\A^{2,1}$, and taking its convex hull. However, the convex hull of a set in projective space can be defined in affine chart, but $\overline{\A^{2,1}}$ is not contained in any affine chart. The following lemma has the purpose to show that the convex hull of the graph of $f$ is well-defined: 
\begin{lemma}\label{bdconvex}
    Let $f:\R\text{P}^1\to\R\text{P}^1$ be an orientation-preserving homeomorphism. Then: 
    \begin{enumerate}
        \item There exists a spacelike plane $\text{P}_\gamma$ in $\A^{2,1}$ such that $\partial\A^{2,1}\cap\text{graph}(f)=\emptyset.$
        \item Given any point $(x_0,y_0)\notin\text{graph}(f)$, there exists a spacelike plane $\text{P}_\gamma$ such that $\partial\text{P}_\gamma=\emptyset$ and $(x_0,y_0)\in\partial\text{P}_\gamma.$ 
    \end{enumerate}
\end{lemma}

\begin{proof}
    \begin{enumerate}
        \item We recall that $\PSL$ acts transitively on pairs of distinct points of $\R\text{P}^1\simeq\R\cup\{\infty\}$ (actually more is true, as it acts simply transitively on \textit{positively oriented triples}). Hence we may assume, up to the action of the isometry group of $\A^{2,1}$ by post-composition on $f$ (remark something that I have to cite), that $f(0)=0$ and $f(\infty)=\infty.$ The $f$ induces a monotone increasing homeomorphism from $R\to\R$. Since $f(0)=0,$ the map $f$ preserves the two intervals: $(-\infty,0)$ and $(0,\infty).$ Let now $\gamma=\mathcal{R}_i$ be the order-two elliptic isometry fixing $i$. Clearly $\gamma$ is an involution that maps $0\to\infty,$ \todo{show this}, and switches the two intervals $(-\infty,0)$ and $(0,\infty)$. Hence $f(x)\neq\gamma(x)$ for all $x\in\R\cup\{\infty\}$, that is $\text{graph}(f)\cap\text{graph}(\gamma)=\emptyset.$ Now by lemma \textcolor{red}{cite} and the fact that $\gamma$ is an involution, $\text{graph}(f)\cap\partial\text{P}_\gamma=\emptyset.$
        \item In proving this point we will use the aforementioned $\PSL$-transitivy on triples, and we will apply both pre and post-composition of an element of $\PSL$. We observe that if $(x_0,y_0)\notin\text{graph}(f)$, then we can find points $x,x^{\prime}$ such that $f$ maps the unoriented arc of $\R\text{P}^1$ connecting $x,x^{\prime}$ containing $x_0$ to the unoriented arc connecting $f(x),f(x^{\prime})$ \textit{not} containing $f(y_0),$ the proof is just figure \textcolor{red}{inserire la figura}. Now, since $f$ preserve the orientation of $\R\text{P}^1,$ up to switching $x,x^{\prime}$, we have that $(x, x_0, x^{\prime})$ is a positive triple in $\S$, while $(f(x),y_0,f(x^{\prime}))$ is a negative triple. \\
        Following the observation, and using simply transitivity on oriented triples, we can assume $(x,x_0,x^{\prime})=(0,1,\infty)$ and $(f(x),y_0,f(x^{\prime}))=(0,-1,\infty).$ At this point choosing $\gamma=\mathcal{R}_i$ as in the first point of the proof satisfies the condition in the second item as well, since $\gamma(1)=-1$.  
    \end{enumerate}
\end{proof}

Now, given a spacelike plane $\text{P}_\gamma$ in $\A^{2,1}$, let $\mathcal{P}_\gamma$ be the unique projective subspaces in $\text{P}\mathcal{M}(2,\R)$ that contains $\text{P}_\gamma$, which is identified by the equation \refeq{spacelike} (where now $[A]=\gamma$). Let us denote by $\mathcal{A}_\gamma$ the complement of $\mathcal{P}_\gamma$, which we will call a (\textit{spacelike}) \textit{affine chart}. In this setting the first item of the lemma \ref{bdconvex} can be riformulated as: 
\begin{corollary}\label{affinechart}
    Let $f:\S\to\S$ be an oriented-preserving homeomorphism. There exists a spacelike affine chart $\mathcal{A}_\gamma$ containing $\text{graph}(f)$.
\end{corollary} 

Resolved any matter with the good definition of convex hull of $\text{graph}(f)$ we want to understand the matter with an example: \\
\textit{Example:} Given $\sigma \in \PSL,$ the convex hull of $\text{graph}(\sigma)$ is the closure of the totally geodesic spacelike plane $\text{P}_{\sigma^{-1}}$ in $\A^{2,1}.$ Actually, because of lemma \textcolor{red}{u know the drill}, the boundary at inifinity of $P_{\sigma^{-1}}$ equals $\text{graph}(\sigma),$ and moreover $\text{P}_{\sigma^{-1}}$ is convex, since spacelike geodesics of $\A^{2,1}$ (which are the intersection of two transverse spacelike planes) are lines in an affine chart, and any two points of $\partial\H^2$ are connected by a geodesic. Hence $\text{P}_{\sigma^{-1}}$ is clearly the smallest convex set containing $\text{graph}(\sigma).$\\
This is the only case where $\text{graph}(\sigma)$ is contained in a plane, and therefore it's convex hull as empty interior. If $f$ is not the restriction to $\S$ of an element of $\PSL$, then the convex hull of $\text{graph}(f)$ is a convex body in the affine chart $\mathcal{A}_\gamma.$

To continue in our exploration of convexity in $\A^{2,1}$ we will need one last technical Lemma similar in spirit to \ref{bdconvex}: 
\begin{lemma}\label{convI}
    Let $f:\S\to\S$ be and orientation-preserving homeomorphism and let $\text{P}_\gamma$ in $\A^{2,1}$ be a spacelike plane such that $\partial\text{P}_\gamma\cap\text{graph}(f)=\emptyset.$ Given any two distinct points $(x,f(x)), (x^{\prime} ,f(x^{\prime} ))$ in $\text{graph}(f)$, there exists a spacelike plane, disjoint from P$_\gamma$, containing them in its boundary at infinity. 
\end{lemma}
\begin{proof}
    Acting via $\PSL\times\PSL$ we can assume that $\gamma=\text{Id}$. The hypothesis $\partial\text{P}_{\text{Id}}\cap\text{graph(f)}=\emptyset$ can be rephrased as saying that $f$ has no fixed point. We are looking for a $\sigma\in\PSL$ such that:
    \begin{itemize}
        \item P$_{\text{Id}}\cap\text{P}_\sigma=\emptyset$ 
        \item $(x,f(x)),(x^{\prime} ,f(x^{\prime}))\in\partial\text{P}_{\sigma^{-1}}=\text{graph}(\sigma).$
    \end{itemize}
    For the first condition to hold it is suffices that the boundary of $\text{P}_{\text{Id}},\text{P}_{\sigma^{-1}}$ do not intersect, that is to say, $\sigma(y)\neq y$ for every $y\in\S$. We are then asking for a $\sigma$ with no fixed points on $\S$, namely $\sigma$ has to be an elliptic isometry. The second condition can be restated as $\sigma(x)=f(x)$ and $\sigma(x^{\prime})=f(x^{\prime}).$\\
Now, since $f$ has no fixed points, $f(x)\neq x$ and $f(x^{\prime})\neq x^{\prime}$. There are various case to distinguish (refer to figure \textcolor{red}{inserire} for a visual aid). First we suppose for $(x,f(x),x^{\prime})$ to be a positive triple. Then either $(x,f(x^{\prime}),f(x),x^{\prime})$ or $(x,f(x),x^{\prime},f(x^{\prime}))$ are in cyclic order, because the remaining possibility, namely that $(x,f(x),f(x^{\prime}) ,x^{\prime})$ are in cyclic order, would imply that $f$ has a fix point. \todo{why so?} If $(x,f(x^{\prime}),f(x),x^{\prime})$ are in cyclic order, then the hyperbolic geodesic $\ell$ connecting $x$ to $f(x)$ and $\ell^{\prime}$ conntecting $x^{\prime} $ to $f(x^{\prime})$ must intersect, and the order two elliptic isometry $\sigma$ fixing $\ell\cap\ell^{\prime}$ maps $x\to f(x)$ and $x^{\prime} \to f(x^{\prime})$.\\
If $(x,f(x),x^{\prime},f(x^{\prime}))$ are in cyclic order, then the geodesic $\ell_1$ connecting $x$ to $x^{\prime}$ and the geodesic $\ell_2$ connecting $f(x)$ to $f(x^{\prime})$ must intersect and one can find an elliptic element $\sigma$ fixing $\ell_1\cap\ell_2$ sending $x\to f(x)$ and $x^{\prime} \to f(x^{\prime}).$ Second, if $(x,f(x),x^{\prime} )$ is a negative triple, then the argument works \textit{verbatim}. Finally, there is the possibility that $f(x)=x^{\prime}$. If $f(x^{\prime})\neq f(x)$, the $\sigma$ we are looking for is an order-three elliptic isometry with fixed point in the barycenter of the triangle of vertices $x,f(x)=x^{\prime},f(x^{\prime})$. If $f(x^{\prime})=x$, we run out of cases by taking as $\sigma$ the order-two elliptic isometry with fixed point on the geodesic $\ell$ from $x$ to $x^{\prime}$. \\
\end{proof}

In particular, as a consequence of Lemma \ref{convI}, we have that given any spacelike affine chart $\mathcal{A}_\gamma$ containing $\text{graph}(f)$ and any two distinct points in $\text{graph}(f)$ the line connecting them is contained in $\A^{2,1}\cap\mathcal{A}_\gamma$ (excepts for the endpoints, which are in $\partial\A^{2,1}$), and it is a spacelike geodesic of $\A^{2,1}$.\\
We are now ready to prove the following: 
\begin{proposition}\label{convII}
    Let $f:\S\to\S$ be an orientation-preserving homeomorphism, let $\text{P}_\gamma$ in $\A^{2,1}$ be a spacelike plane such that $\partial\text{P}_\gamma\cap\text{graph}(f)=\emptyset,$ and let $K$ be the convex hull of $\text{graph}(f)$ in the affine chart $\mathcal{A}_\gamma$. Then: 
    \begin{itemize}
        \item The interior of $K$ is contained in $\A^{2,1}$
        \item The intersection of $K$ with $\partial\A^{2,1}$ is equal to $\text{graph}(f)$.
    \end{itemize}
    In particular, $K$ is a convex body that is a subset of $\overline{\A^{2,1}}$.
\end{proposition}
\begin{proof}
    Given a point in $\partial\A^{2,1}\setminus\text{graph}(f)$, by the second item of Lemma \ref{bdconvex} there exists a spacelike plane $\text{P}_\eta$ passing through $p$ that does not intersect $\text{graph}(f).$ This implies $\text{P}_\eta\cap K=\emptyset,$ hence $K\cap\partial\A^{2,1}=\text{graph}(f).$ Since $K$ is connected, it is contained in the closure of one component of the complement of $\partial\A^{2,1}$ in $\mathcal{A}_\gamma.$ But $K$ is connected and intersects $\A^{2,1}\setminus\text{P}_\gamma$ because, by \ref{convI}, the line segment connecting any two point of $\text{graph}(f)$ in the affine chart $\mathcal{A}_\gamma$ is contained in $\A^{2,1}\cap\mathcal{A}_\gamma.$ Hence $K$ is contained in $\overline{\A^{2,1}}$ with interior in $\A^{2,1}$. \\
\end{proof}

In light of Corollary \ref{affinechart} and Proposition \ref{convII}, we give the following definition: 
\begin{definition}
    Let $f:\S\to\S$ be an orientation-preserving homeomorphism, we define $\mathcal{C}(f)$ to be the subset of $\overline{\A^{2,1}}$ that is the convex hull of $\text{graph}(f)$ in any spacelike affine chart $\mathcal{A}_\gamma$ such that $\partial\text{P}_\gamma\cap\text{graph}(f)=\emptyset.$
\end{definition}

The definition is well-posed because lines and planes are well-defined in projective space, hence the change of coordinates from an affine chart to another preserves convex sets. Because of this when referring to convexity notion in the following, we will implicitly assume that we have chosen a spacelike affine chart $\mathcal{A}_\gamma$ that contains $\text{graph}(f)$.

\section{Support planes.}
We still need to borrow some notions and notations from convex analysis. Given a convex body $K$ in affine space of dimension three, a \textit{support plane} of $K$ is an affine plane $Q$ such that $K$ is contained in a closed half-space bounded by $Q$, and $\partial K\cap Q\neq\emptyset.$ If $p$ is in the intersection of $K$ with $Q$ we will say that $Q$ is a \textit{support plane} at $p$. As a consequence of the Hahn-Banach theorem there exixst a support plane at every point $p\in\partial K.$ \\
We will adopt the terminology to the Anti-de Sitter setting, given a convex hull $\mathcal{C}(f)$ in $\A^{2,1}$, we say that a totally geodesic plane $P$ is a support plane of $\mathcal{C}(f)$ (at $p\in\partial\mathcal{C}(f))$ is $p\in\mathcal{C}(f)\cap \overline{P}\subset\overline{\A^{2,1}}$ and, in an affine chart containing $\text{graph}(f)$, $\mathcal{C}(f)$ lies in a closed half-space bounded by the affine plane that contains $P$. Even this definition does not depend on the choice of the affine chart as long this one contains $\text{graph}(f)$.\\ 

    

\section{The fundamental example.} We want to now finally link Ads geometry, in particular pleated surfaces, with the earthquake theory. Before stating the precise results we explore what we can consider the \textit{fundamental example.} Let $S$ be a piecewise totally geodesic surfaces consisting in the union of two half-planes in $\A^{2,1}$ meeting along a spacelike geodesic as in \textcolor{red}{inserire figura}

We want to understand the left and right projection for this surface $S$. Observe that these are well-defined in the complement of spacelike geodesic which constitutes the bending locus of $S.$ The projections $\Pi_l$ and $\Pi_r$ are isometric on each (totally geodesic) connected component of the complement of such bendind geodesic in $S$ [refernece]. Let us call the two components $S_1,S_2$. Without loss of generality we can assum that $S_1$ is contained in the plane $P_{Id}$ composed of order-two elliptic elements in $\PSL$. Therefore the bending locus is a spacelike geodesic contained in $P_{Id}$, namely the set of order-two elliptic elements having a fixed point in a geodesic $\ell $ of $\H^2$.  