\chapter{Thurston's earthquake theorem}

We pause for the moment with our exploration of the Anti-de Sitter realm. We want to recall basic definitions of geodesic lamination and earthquakes, with the introduction of a first \textit{basic} example. We will then give an outline of the deep underlying relation between pleated surfaces and earthquakes discovered by Mess and then finally move on the proof of the earthquake theorem. 

\begin{definition}
    A geodesic lamination $\lambda$ of $\H^2$ is a collection of disjoint geodesic that foliate a closed subset $X$ of $\H^2$. The set $X$ is called the \textit{support} of $\lambda.$ The geodesic in $\lambda$ are called \textit{leaves} (as in classic foliation terminology). The connected components of $\H^2\;\setminus\;X$ are called \textit{gaps}. The \textit{strata} of $\lambda$ are the leaves and the gaps.
\end{definition}

\noindent Consider $\gamma$ a loxodromic isometry of $\H^2$, the \textit{axis} of the isometry is the geodesic $\ell$ of $\H^2$ connecting the two fixed points of $\gamma$ in $\partial\H^2$. As a consequence we have that the geodesic is preserved by the isometry, when restricted to such a curve $\restr{\gamma}{\ell}:\ell\to\ell$ acts as a translation with respect to any constant speed parametrization of $\ell$.\\
Given $A,B$ subsets of $\H^2,$ we say that a geodesic $\ell$ \textit{weakly separated} $A$ and $B$ if $A$ and $B$ are contained in the closure of different connected components of $\H^2\;\setminus\;\ell.$ 

\begin{definition}
    A \textit{left earthquake} (resp. \textit{right}) of $\H^2$ is a bijective map $E:\H^2\to\H^2$ such that there exists a geodesic lamination $\lambda$ for which the restriction $\restr{E}{S}$ to any stratum $S$ of $\lambda$ is equal to the restriction of an isometry of $\H^2,$ and for any two strata $S$ and $S^{\prime}$ of $\lambda$ the comparison isometry 
    
    \[
        \text{Comp}(S,S^{\prime})\coloneqq (\restr{E}{S})^{-1}\circ \restr{E}{S^{\prime} }
    \]

    is the restriction of an isometry of $\H^2$ such that:
    \begin{itemize}
        \item $\gamma$ is different from the identity, unless possibly where one of the two strata $S$ and $S^{\prime}$ is contained in the closure of the other;
        \item when it is not the identity, $\gamma$ is a loxodromic transformation whose axis $\ell$ weakly separates $S$ and $S^{\prime};$
        \item moreover, $\gamma$ translates to the left (resp right), seen from $S$ to $S^{\prime}$. 
    \end{itemize}
\end{definition}

Let's explain more carefully what we mean with the last condition. Suppose $f:[0,1]\to\H^2$ is a smooth path such that $f(0)\in S,\;f(1)\in S^{\prime}$ and the image of $f$ intersects $\ell$ transversally and exactly at one point $z_0=f(t_0)\in\ell.$ Let $v=f^{\prime}(t_0)\in T_{z_0}\H^2$ be the tangent vector at the intersection point. Let $w\in T_{z_0}\H^2$ be a vector tangent to the geodesic $\ell$ pointing towards $\gamma(z_0).$ Then we say that $\gamma$ translates to the left seen from $S$ to $S^{\prime} $ if $v,w$ is a positive basis of $T_{z_0}\H^2$, for the standard orientation of $\H^2.$\\
We observe that such a condition is independent of the order in which we choose $S$ and $S^{\prime},$  if $\text{Comp}(S,S^{\prime})$ translates to the left seen from $S$ to $S^{\prime},$ then $\text{Comp}(S^{\prime} ,S)$ translates to the left seen from $S^{\prime} $ to $S.$ 

%ok ma cosa vuol dire davvero questa cosa?

\begin{observation} The earthquake $E$ is not required to be (and in fact in some case will not be) continuous. For instance this happens when the lamination $\lambda$ is finite, meaning that $\lambda$ is a collection of a finite number of geodesics. 
\end{observation}

Let's consider a first basic example: 

\begin{example}\label{simplequake}
The map 
\[
    E:\H^2\to\H^2
\]
defined by:
\[
  E(z)= \begin{cases}
    z & \text{if Re(z)}<0 \\
    az & \text{if Re(z)}=0 \\
    bz & \text{if Re(z)}>0 \\    
\end{cases}
\]

is a left earthquake if $1<a<b,$ and a right earthquake if $0<b<a<1$. The lamination $\lambda$ that satisfies the definition is composed of a unique geodesic, namely the geodesic $\ell$ corresponding to the imaginary axis. \\
Such a map is clearly not continuous along $\ell$.
Thurston proved in \cite{thurston1986earthquakes} that any earthquake map extends continuously to an orientation-preserving homeomorphism of $\partial\H^2$ meaning that there exist a (unique) orientation-preserving homeomorphism $\varphi:\partial\H^2\to\partial\H^2$ such that the map:
\[
    \begin{cases}
        E(z), &\text{ if z }\in \H^2\\
        \varphi(z), &\text{ if z}\in \partial\H^2  ;\\
        
    \end{cases}
\]
is continuous in any point in $\partial\H^2.$\\ \textcolor{blue}{la proof è abbastanza facile, avrebbe senso rifarla}
Then Thurston proved the following (in some sense \textit{dual}) theorem, that he called \textit{``geology is transitive"}:

\begin{theorem}[``Geology is transitive"]
    Given any orientation-preserving homeomorphism $\varphi:\partial\H^2\to\partial\H^2,$ there exists a left earthquake map of $\H^2,$ and a right earthquake map, that extends continuously to $\varphi$ on $\partial\H^2.$
\end{theorem}
\end{example}

We would like to give a different proof of the statement using the tools of Anti-de Sitter geometry developed in the previous chapters. More in detail, the key observation that we will use is due to Mess's work \cite{Mess}, that highlighted the relation between pleated surfaces and earthquake maps. Recall that given an achronal meridian $\Lambda\subset\A^{2,1},$ the upper and lower boundary components $\partial_{\pm}\CF$ of the convex hull of $\Lambda$ are a convex and a concave pleated surface, see Proposition \ref{466} and Remark \ref{467}.\\
In general the pleated surfaces $\partial_\pm(\Lambda)$ may contain lightlike triangles, which happens exactly in correspondence of a sawtooth, as in \ref{465}. In this case, the Gauss map is not defined on these lightlike triangles.\\
Let's be more precise about the detail of what is going on. Left and right projections from $\partial_+\CF$ to $\H^2$ are (right and left respectively) earthquake maps with image a straight convex set, and the earthquake \textcolor{blue}{measured} lamination coincide with the bending measured lamination. We can consider now the composition $\Pi_r\circ\Pi_l^{-1},$ that is a left earthquake map defined in the complement of the simplicial leaves of the lamination, and its earthquake measured lamination is identified to the bending measured lamination of $\partial_{+}\CF$. A completely analogous statement holds for $\partial_{-}\CF$ by reversing the roles of left and right.\\  
Now, when the curve $\Lambda$ is the graph of an orientation-preserving homeomorphism of $\S$, one obtains as a result earthquakes maps of $\H^2$. When moreover $\varphi$ is the homeomorphism which conjugates left and right representations $\rho_l,\rho_r:\pi_1(\Sigma)\to\PSL$ of the holonomy of a MGH Cauchy compact manifold, the naturality of the construction implies that the earthquake map descends to an earthquake map from the left to the right hyperbolic surfaces, namely $\H^2/\rho_l(\pi_1(\Sigma))$ and $\H^2/\rho_r(\pi_1(\Sigma))$.\\
Given the previous discussion, we can now start with the details.


% \section{Affine space and Convexity Notions}
% The initial step in our proof is to consider the graph of an orientation-preserving homeomorphism $f:\R\text{P}^1\to\R\text{P}^1$ as a subset of $\partial\A^{2,1}$, and taking its convex hull. However, the convex hull of a set in projective space can be defined in affine chart, but $\overline{\A^{2,1}}$ is not contained in any affine chart. The following lemma has the purpose to show that the convex hull of the graph of $f$ is well-defined: 
% \begin{lemma}\label{bdconvex}
%     Let $f:\R\text{P}^1\to\R\text{P}^1$ be an orientation-preserving homeomorphism. Then: 
%     \begin{enumerate}
%         \item There exists a spacelike plane $\text{P}_\gamma$ in $\A^{2,1}$ such that $\partial\A^{2,1}\cap\text{graph}(f)=\emptyset.$
%         \item Given any point $(x_0,y_0)\notin\text{graph}(f)$, there exists a spacelike plane $\text{P}_\gamma$ such that $\partial\text{P}_\gamma=\emptyset$ and $(x_0,y_0)\in\partial\text{P}_\gamma.$ 
%     \end{enumerate}
% \end{lemma}

% \begin{proof}
%     \begin{enumerate}
%         \item We recall that $\PSL$ acts transitively on pairs of distinct points of $\R\text{P}^1\simeq\R\cup\{\infty\}$ (actually more is true, as it acts simply transitively on \textit{positively oriented triples}). Hence we may assume, up to the action of the isometry group of $\A^{2,1}$ by post-composition on $f$ (we observe that because of equivariance given $(\alpha,\beta)\in\T$ it holds $(\alpha,\beta)\cdot\text{graph}(f)=\text{graph}(\beta f\alpha^{-1})$), that $f(0)=0$ and $f(\infty)=\infty.$ The $f$ induces a monotone increasing homeomorphism from $R\to\R$. Since $f(0)=0,$ the map $f$ preserves the two intervals: $(-\infty,0)$ and $(0,\infty).$ Let now $\gamma=\mathcal{R}_i$ be the order-two elliptic isometry fixing $i$. Clearly $\gamma$ is an involution that maps $0\to\infty,$ \todo{show this}, and switches the two intervals $(-\infty,0)$ and $(0,\infty)$. Hence $f(x)\neq\gamma(x)$ for all $x\in\R\cup\{\infty\}$, that is $\text{graph}(f)\cap\text{graph}(\gamma)=\emptyset.$ Now by lemma \textcolor{red}{cite} and the fact that $\gamma$ is an involution, $\text{graph}(f)\cap\partial\text{P}_\gamma=\emptyset.$
%         \item In proving this point we will use the aforementioned $\PSL$-transitivy on triples, and we will apply both pre and post-composition of an element of $\PSL$. We observe that if $(x_0,y_0)\notin\text{graph}(f)$, then we can find points $x,x^{\prime}$ such that $f$ maps the unoriented arc of $\R\text{P}^1$ connecting $x,x^{\prime}$ containing $x_0$ to the unoriented arc connecting $f(x),f(x^{\prime})$ \textit{not} containing $f(y_0),$ the proof is just figure \textcolor{red}{inserire la figura}. Now, since $f$ preserve the orientation of $\R\text{P}^1,$ up to switching $x,x^{\prime}$, we have that $(x, x_0, x^{\prime})$ is a positive triple in $\S$, while $(f(x),y_0,f(x^{\prime}))$ is a negative triple. \\
%         Following the observation, and using simply transitivity on oriented triples, we can assume $(x,x_0,x^{\prime})=(0,1,\infty)$ and $(f(x),y_0,f(x^{\prime}))=(0,-1,\infty).$ At this point choosing $\gamma=\mathcal{R}_i$ as in the first point of the proof satisfies the condition in the second item as well, since $\gamma(1)=-1$.  
%     \end{enumerate}
% \end{proof}

% Now, given a spacelike plane $\text{P}_\gamma$ in $\A^{2,1}$, let $\mathcal{P}_\gamma$ be the unique projective subspaces in $\text{P}\mathcal{M}(2,\R)$ that contains $\text{P}_\gamma$, which is identified by the equation \refeq{spacelike} (where now $[A]=\gamma$). Let us denote by $\mathcal{A}_\gamma$ the complement of $\mathcal{P}_\gamma$, which we will call a (\textit{spacelike}) \textit{affine chart}. In this setting the first item of the lemma \ref{bdconvex} can be riformulated as: 
% \begin{corollary}\label{affinechart}
%     Let $f:\S\to\S$ be an oriented-preserving homeomorphism. There exists a spacelike affine chart $\mathcal{A}_\gamma$ containing $\text{graph}(f)$.
% \end{corollary} 

% Resolved any matter with the good definition of convex hull of $\text{graph}(f)$ we want to understand the matter with an example: \\


% To continue in our exploration of convexity in $\A^{2,1}$ we will need one last technical Lemma similar in spirit to \ref{bdconvex}: 
% \begin{lemma}\label{convI}
%     Let $f:\S\to\S$ be and orientation-preserving homeomorphism and let $\text{P}_\gamma$ in $\A^{2,1}$ be a spacelike plane such that $\partial\text{P}_\gamma\cap\text{graph}(f)=\emptyset.$ Given any two distinct points $(x,f(x)), (x^{\prime} ,f(x^{\prime} ))$ in $\text{graph}(f)$, there exists a spacelike plane, disjoint from P$_\gamma$, containing them in its boundary at infinity. 
% \end{lemma}
% \begin{proof}
%     Acting via $\PSL\times\PSL$ we can assume that $\gamma=\text{Id}$. The hypothesis $\partial\text{P}_{\text{Id}}\cap\text{graph(f)}=\emptyset$ can be rephrased as saying that $f$ has no fixed point. We are looking for a $\sigma\in\PSL$ such that:
%     \begin{itemize}
%         \item P$_{\text{Id}}\cap\text{P}_\sigma=\emptyset$ 
%         \item $(x,f(x)),(x^{\prime} ,f(x^{\prime}))\in\partial\text{P}_{\sigma^{-1}}=\text{graph}(\sigma).$
%     \end{itemize}
%     For the first condition to hold it is suffices that the boundary of $\text{P}_{\text{Id}},\text{P}_{\sigma^{-1}}$ do not intersect, that is to say, $\sigma(y)\neq y$ for every $y\in\S$. We are then asking for a $\sigma$ with no fixed points on $\S$, namely $\sigma$ has to be an elliptic isometry. The second condition can be restated as $\sigma(x)=f(x)$ and $\sigma(x^{\prime})=f(x^{\prime}).$\\
% Now, since $f$ has no fixed points, $f(x)\neq x$ and $f(x^{\prime})\neq x^{\prime}$. There are various case to distinguish (refer to figure \textcolor{red}{inserire} for a visual aid). First we suppose for $(x,f(x),x^{\prime})$ to be a positive triple. Then either $(x,f(x^{\prime}),f(x),x^{\prime})$ or $(x,f(x),x^{\prime},f(x^{\prime}))$ are in cyclic order, because the remaining possibility, namely that $(x,f(x),f(x^{\prime}) ,x^{\prime})$ are in cyclic order, would imply that $f$ has a fix point. \todo{why so?} If $(x,f(x^{\prime}),f(x),x^{\prime})$ are in cyclic order, then the hyperbolic geodesic $\ell$ connecting $x$ to $f(x)$ and $\ell^{\prime}$ conntecting $x^{\prime} $ to $f(x^{\prime})$ must intersect, and the order two elliptic isometry $\sigma$ fixing $\ell\cap\ell^{\prime}$ maps $x\to f(x)$ and $x^{\prime} \to f(x^{\prime})$.\\
% If $(x,f(x),x^{\prime},f(x^{\prime}))$ are in cyclic order, then the geodesic $\ell_1$ connecting $x$ to $x^{\prime}$ and the geodesic $\ell_2$ connecting $f(x)$ to $f(x^{\prime})$ must intersect and one can find an elliptic element $\sigma$ fixing $\ell_1\cap\ell_2$ sending $x\to f(x)$ and $x^{\prime} \to f(x^{\prime}).$ Second, if $(x,f(x),x^{\prime} )$ is a negative triple, then the argument works \textit{verbatim}. Finally, there is the possibility that $f(x)=x^{\prime}$. If $f(x^{\prime})\neq f(x)$, the $\sigma$ we are looking for is an order-three elliptic isometry with fixed point in the barycenter of the triangle of vertices $x,f(x)=x^{\prime},f(x^{\prime})$. If $f(x^{\prime})=x$, we run out of cases by taking as $\sigma$ the order-two elliptic isometry with fixed point on the geodesic $\ell$ from $x$ to $x^{\prime}$. \\
% \end{proof}

% In particular, as a consequence of Lemma \ref{convI}, we have that given any spacelike affine chart $\mathcal{A}_\gamma$ containing $\text{graph}(f)$ and any two distinct points in $\text{graph}(f)$ the line connecting them is contained in $\A^{2,1}\cap\mathcal{A}_\gamma$ (excepts for the endpoints, which are in $\partial\A^{2,1}$), and it is a spacelike geodesic of $\A^{2,1}$.\\
% We are now ready to prove the following: 
% \begin{proposition}\label{convII}
%     Let $f:\S\to\S$ be an orientation-preserving homeomorphism, let $\text{P}_\gamma$ in $\A^{2,1}$ be a spacelike plane such that $\partial\text{P}_\gamma\cap\text{graph}(f)=\emptyset,$ and let $K$ be the convex hull of $\text{graph}(f)$ in the affine chart $\mathcal{A}_\gamma$. Then: 
%     \begin{itemize}
%         \item The interior of $K$ is contained in $\A^{2,1}$
%         \item The intersection of $K$ with $\partial\A^{2,1}$ is equal to $\text{graph}(f)$.
%     \end{itemize}
%     In particular, $K$ is a convex body that is a subset of $\overline{\A^{2,1}}$.
% \end{proposition}
% \begin{proof}
%     Given a point in $\partial\A^{2,1}\setminus\text{graph}(f)$, by the second item of Lemma \ref{bdconvex} there exists a spacelike plane $\text{P}_\eta$ passing through $p$ that does not intersect $\text{graph}(f).$ This implies $\text{P}_\eta\cap K=\emptyset,$ hence $K\cap\partial\A^{2,1}=\text{graph}(f).$ Since $K$ is connected, it is contained in the closure of one component of the complement of $\partial\A^{2,1}$ in $\mathcal{A}_\gamma.$ But $K$ is connected and intersects $\A^{2,1}\setminus\text{P}_\gamma$ because, by \ref{convI}, the line segment connecting any two point of $\text{graph}(f)$ in the affine chart $\mathcal{A}_\gamma$ is contained in $\A^{2,1}\cap\mathcal{A}_\gamma.$ Hence $K$ is contained in $\overline{\A^{2,1}}$ with interior in $\A^{2,1}$. \\
% \end{proof}

% In light of Corollary \ref{affinechart} and Proposition \ref{convII}, we give the following definition: 
% \begin{definition}
%     Let $f:\S\to\S$ be an orientation-preserving homeomorphism, we define $\mathcal{C}(f)$ to be the subset of $\overline{\A^{2,1}}$ that is the convex hull of $\text{graph}(f)$ in any spacelike affine chart $\mathcal{A}_\gamma$ such that $\partial\text{P}_\gamma\cap\text{graph}(f)=\emptyset.$
% \end{definition}

% The definition is well-posed because lines and planes are well-defined in projective space, hence the change of coordinates from an affine chart to another preserves convex sets. Because of this when referring to convexity notion in the following, we will implicitly assume that we have chosen a spacelike affine chart $\mathcal{A}_\gamma$ that contains $\text{graph}(f)$.

\section{Earthquake Theorem}
We consider $\varphi:\S\to\S$ an orientation-preserving homeomorphism of the circle, and by $\Lambda_\varphi$ we denote its graph as a subset of $\T$. We recall that with the notation of the previous chapter, $\Lambda_\varphi$ is a proper achronal meridian. By means of the Gauss map we had defined left and right projections; 

\[
    \Pi_l^{\pm:}\partial_\pm\CF\to\H^2 \;\;\; \Pi_r^{\pm:}\partial_\pm\CF\to\H^2
\]

we want now to extend everything even when we have weaker regularity condition. Consider a point $\pi\in \partial_\pm\CF$ and let $P$ be a support plane of $\CF$ at $\pi$. By Proposition \ref{supportplane} the support plane is necessarily spacelike, hence of the form $P=P_\gamma$ for some $\gamma\in\PSL$. What happens when $\partial_\pm\CF$ is not $C^1$ at $p$ is that we do not have a unique support plane. Hence we \textit{choose} a support plane $P_\gamma$ at $p$, requiring that the choice of support planes is made so that the support plane is constant of any connected component of the subset of $\partial_\pm\CF$ consisting of those points that admit more than one support plane. The definition of the Gauss map then \textit{depends} on the choice of $P_\gamma$ (see Corollary \ref{multipleplanes} for more details on hoe the choice influences the map). Once we have chosen the support planes we can just follow \textit{verbatim} the construction of the Gauss map (in what follow we will consider it as given by the construction described in Remark \ref{DiafGauss})

% \section{Left and right projections.} We want to introduce two maps, \textit{left and right projections,} which will have a key role in the proof of the earthquake theorem. These are the maps: 
% \[
%     \Pi_l^\pm:\partial_\pm\CF\to\H^2 \;\;\; \Pi_r^\pm:\partial_\pm\CF\to\H^2
% \] 

% defined on the past or future of $\partial\CF$ constructed as follows. Given a point $p\in\partial_\pm\CF$, let $P$ be a support plane of $\CF$ at $p$. By Proposition \textcolor{red}{ref}, the support plane is necessarily spacelike, hence of the form $P=P_\gamma$ for some $\gamma\in\PSL$.

 %\begin{observation}
 %We have shown the existance of such a $P_\gamma$ but in general it might not be unique, if $\partial_{\pm}\CF$ is not $C^1$ at $p.$ Hence we \textit{choose} a support plane $P_\gamma$ at $p$. We also require that the choice of support plane is made so that the support plane is constant on any connected components of the subsets of $\partial_\pm\CF$ consistings of points that admit more than one support plane. The definitions of the projection then \textit{depends} (we will shortly characterize how in \textcolor{red}{add}) on the choice of $P_\gamma$.
 %\end{observation}

% Now, we choose a support plane $P_\gamma$ at $p$, left or right multiplication by $\gamma^{-1}$ maps $\gamma$ to the identity, and therefore it maps $P_\gamma$ to $P_{\text{Id}},$ which we have already studied in \textcolor{red}{reference} it is the space of order-two elliptic elements and is therefore naturally identified with $\H^2$ via the map: $\text{Fix}:P_{\text{Id}}\to\H^2.$ \\
% In classic Lie group theory fashion we denote by $L_{\gamma^{-1}}:\PSL\to\PSL$ and $\R_{\gamma^{-1}}:\PSL\to\PSL$ the left and right multiplication by $\gamma^{-1}$; these can also be viewed as the action of the elements $(\gamma,\text{Id})$ and $(\text{Id},\gamma^{-1})$ of $\PSL\times\PSL$. We now have that both $L_{\gamma^{-1}}(p)$ and $R_{\gamma^{-1}}(p)$ are elements of $P_{\text{Id}}$ and $L_{\gamma^{-1}}(p)$ (resp. $R_{\gamma^{-1}}(p)$) is a bijection between $P_\gamma$ and $P_\text{Id}.$ We can now define: 
% \begin{equation}\label{defproj}
%     \Pi_l^\pm(p)=Fix(R_{\gamma^{-1}}(p))\;\ \Pi_r^\pm(p)=Fix(L_{\gamma^{-1}}(p)).
% \end{equation}
    


% The choice of define left and right projection in this apparently counterintuitive way is motivated by the following:

% \begin{lemma}
%     Let $f:\S\to\S$ be an orientation-preserving homeomorphism, and let $(\alpha, \beta)\in \PSL\times\PSL$. Let us denote $K=\CF$ and $\hat{K}=(\alpha,\beta)\cdot\CF$ and let, $\Pi_l^{\pm},\Pi_r^{\pm}:\partial_{\pm}K\to\H^2$ and $\hat{\pi}_l^\pm,\hat{\pi}_r^\pm:\hat{K}\to\H^2$ be the left and right projection of $K$ and $\hat{K}$ respectively. Then:
%     \begin{equation}\label{equivariance}
%         \hat{\pi}_l^\pm\circ(\alpha,\beta)=\alpha\circ\Pi_l^\pm\;\;\;\ \hat{\pi}_r^\pm\circ(\alpha,\beta)=\beta\circ\Pi_r^\pm.
%     \end{equation}
% \end{lemma}
% Let's clarify the statement. The isometry $(\alpha,\beta)$ maps a point $p\in K$ to a point $\hat{p}\in\hat{K}$, and maps support planes at $p\in K$ to support planes at $\hat{p}$. Hence the relation introduced in \refeq{equivariance} holds when we consider the projections $\hat{\pi}_l^\pm$ and $\Pi_r^\pm$ defined with the choice of support planes of $\hat{K}$ given by the images $\hat{P}$ chosen in the definitions of $\Pi_l^\pm$ and $\Pi_r^\pm$.

% \begin{proof}
%     We recall that for any $\pi \in \partial^\pm K$, we consider $\hat{p}\coloneqq(\alpha,\beta)\cdot p\in\hat{K},$ and for a chosen support plane $P=P_\gamma$ for $K$ at $p,$ $(\alpha,\beta)\cdot P=P_{\hat{\gamma}}$ is the chosen support plane for $\hat{K}$ at $\hat{p}.$\\
%     By the duality, $\hat{\gamma}=(\alpha,\beta)\cdot\gamma=\alpha\gamma\beta^{-1}.$ It follows that: 
%     \begin{align*}
%         \hat{\pi}_l^\pm(\hat{p}) &= \text{Fix}(R_{\hat{\gamma}^{-1}}(\hat{p}))=\text{Fix}(R_{(\beta\gamma^{-1}\alpha^{-1})}(\alpha p\beta^{-1}))\\
%         &=\text{Fix}(R_{(\gamma^{-1}\alpha^{-1})}(\alpha p))=\text{Fix}(\alpha\circ R_{\gamma^{-1}}(p)\circ\alpha^{-1}) \\
%         &=\alpha(\text{Fix}(R_\gamma^{-1})(p))=\alpha\circ\Pi_l^\pm(p).
%     \end{align*}
%     And a completely analogous argument holds for the right projection.
% \end{proof}

\begin{example}\label{413} Let us consider a toy case where $\varphi\in\PSL$ so that $\CF=P_{\varphi^{-1}}$ as in the previous Example \ref{43}. This is in some sense a degenerate case, as $\CF$ has empty interior, hence Corollary \ref{nametag} does not apply and it does not \textit{really} makes sense to talk about the future and the past component boundary. However, we can still define a left and right projections. Since $P_{\varphi^{-1}}$ itself is the unique support plane at any of its points, from the definition of the Gauss map we have the following expressions for the left and right projections $\Pi_l,\Pi_r:P_{\varphi^{-1}}\to\H^2:$
\begin{equation}
    \Pi_l(p)=\text{Fix}(p\circ\varphi)\;\;\Pi_r(p)=\text{Fix}(\varphi\circ p).
\end{equation}   
We can also extend the two map to the boundary of $P_{\varphi^{-1}}$: recalling that its boundary coincides with the graph of $\varphi$ (Lemma \ref{32}) we have: 
\begin{equation}\label{simproj}
    \Pi_l(x,\varphi(x))=x \;\;\;\Pi_r(x,\varphi(x))=\varphi(x). 
\end{equation}

Equation \refeq{simproj} is immediately checked when $\varphi=\text{Id},$ because in that case we have that $\Pi_l,\Pi_r$ simply coincide with the fixed point map $\text{Fix}:P_1\to\H^2$, and we observed previously that $\text{Fix}$ extends to the map $(x,x)\to x$ from $\partial P_{\text{Id}}$ to $\partial\H^2$. The general case of Equation \refeq{simproj} is then consequences of the equivariance of the Gauss map, with the additional observation that the isometry $(\text{Id},\varphi)$ maps $\text{graph}(\text{Id})$ to $\text{graph}(\varphi)$ and $P_\text{Id}$ to $P_{\varphi^{-1}}$. \\
We can now compute the map of $\H^2$ obtained by composing the inverse of the left projection with the right projection. Indeed, this is induced by the map $P_{\text{Id}}\to P_\text{Id}$ sending an order-two elliptic element $\mathcal{R}=p\circ\varphi\in P_{\text{Id}}$ to $\varphi\circ p=\varphi\circ\mathcal{R}\circ\varphi^{-1}$. Hence we have   
\begin{equation}\label{composition}
    \Pi_r\circ\Pi_l^{-1}=\varphi:\H^2\to\H^2.
\end{equation}
In conclusion we have that the composition of the maps $\Pi_r\circ\Pi_l^{-1}$ is an isometry and its extension to the boundary of $\H^2$ is precisely the map $f=\varphi$ of which $\partial P_{\sigma^{-1}}$ is the graph. In what follows we will observe that this is what happens in the general case, that is, given an orientation-preserving homeomorphism of the circle $\varphi$, the composition $\Pi_r^\pm\circ(\Pi_l^\pm)^{-1}$ associated with $\partial_\pm\CF$ will be the left and right earthquake extending $\varphi$.
\end{example}

\section{The fundamental example.} We want to move one more intermediate step towards the final theorem. After the simple case, this time we will describe what we can consider \textit{the fundamental example}. Consider a piecewise totally geodesic surfaces in $\A^{2,1},$ which are obtained as the union of two connected subsets, each contained in a totally geodesic spacelike plane, meeting along a common geodesic. \\
Let's formalize this idea in a more precise way. Consider the union of two half-planes, each contained in a totally geodesic spacelike plane $P_{\gamma_1},P_{\gamma_2}$. The first key fact is the following:

\begin{lemma}\label{Mati}
    Let $\gamma_1\neq\gamma_2\in\PSL$. Then $P_{\gamma_1}$ and $P_{\gamma_2}$ intersect in $\A^{2,1}$ if and only if $\gamma_2\circ{\gamma_1^{-1}}$ is a loxodromic isometry. 
\end{lemma}
\begin{proof}
    As in Example \ref{43} $P_{\gamma_i}$ is the convex hull of $\partial P_{\gamma_i}=\text{graph}(\gamma_i^{-1}),$ the closures $\overline{P}_{\gamma_i}$  intersect in $\overline{\A^{2,1}}$ if and only if $\text{graph}(\gamma_1)\cap\text{graph}(\gamma_2)\neq\emptyset$. Moreover, by \refeq{suplane}, $P_{\gamma_1}$ and $P_{\gamma_2}$ intersect in $\A^{2,1}$ if and only if $\text{graph}(\gamma_1)\cap\text{graph}(\gamma_2)$ contains at least two different points. \\
    We know that $(x,y)\in\T$ is in $\text{graph}(\gamma_1)\cap\text{graph}(\gamma_2)$ if and only if $y=\gamma_{-1}(x)=\gamma_2^{-1}(x)$, which is equivalent as asking $x\in\text{Fix}(\gamma_2\circ\gamma_1^{-1})$. But the composition $\gamma_2\circ\gamma_1^{-1}$ is an element of $\PSL$, hence it has two fixed point in $\partial\H^2\simeq\S$ if and only if it is a loxodromic isometry.
\end{proof}

Now consider $S=I_1\cup I_2$ where $I_1,I_2$ are two closed intervals such that $I_1\cap I_2$ consist exactly of the fixed points of $\gamma_2\circ\gamma_1^{-1}.$ Clearly there are two possibilities to produce a homeomorphism of $\S$ by composing the restriction of $\gamma_1^{-1}$ and $\gamma_2^{-1}$ to the intervals $I_j$'s, namely: 

\begin{equation}
    \varphi_{\gamma_1,\gamma_2}(x) = \begin{dcases}
        \gamma_1^{-1}, & \text{if } x \in I_1; \\
        \gamma_2^{-1}, & \text{if } x \in I_2;
    \end{dcases}
    \;\text{and}\;
    \varphi^-_{\gamma_1,\gamma_2}(x) = \begin{dcases}
        \gamma_2^{-1}, & \text{if } x \in I_1 ;\\
        \gamma_1^{-1} , & \text{if } x \in I_2.
    \end{dcases}
    \end{equation}
    
Both $\varphi_{\gamma_1,\gamma_2}^{\pm}$ are orientation-preserving homeomorphism, since $\gamma_1^{-1}$ and $\gamma_2^{-1}$ map homeomorphically the intervals $I_1$ and $I_2$ to the same intervals $J_1\coloneqq\gamma_1^{-1}(I_1)=\gamma_2^{-1}(I_1)$ and $J_2\coloneqq\gamma_1^{-1}(I_2)=\gamma_2^{-1}(I_2)$ which intersects only at their endpoints. \\
We denote by $D_i$ the convex hull of $I_i$ in $\H^2,$ and by $\ell=D_1\cap D_2$ the axis of $\gamma_2\circ\gamma_1^{-1}$. Then with these notations we can state: 

\begin{proposition}\label{gettinthere}
    Suppose that $\gamma_2\circ\gamma_1^{-1}$ is a loxodromic isometry that translates along $\ell$ to the left, as seen from $D_1$ to $D_2$. Then:
    \begin{itemize}
        \item The future boundary component $\partial_+\mathcal{C}(\varphi_{\gamma^+,\gamma_2}^{+})$ coincides with the union of the convex envelope of $\text{graph}(\restr{\gamma_1^{-1}}{I_1})$ and the convex envelope of $\text{graph}(\restr{\gamma_2^{-1}}{I_2})$
        \item The past boundary component $\partial_-\mathcal{C}(\varphi^-_{\gamma_1,\gamma_2})$ coincides with the union of the convex envelope of $\text{graph}(\restr{\gamma_1^{-1}}{I_2})$ and of the convex envelope of $\text{graph}(\restr{\gamma_2^{-1}}{I_1}).$
    \end{itemize}
    If instead $\gamma_2\circ\gamma_1^{-1}$ translates along $\ell$ to the right as seen from $D_1$ to $D_2$, then:
    \begin{itemize}
        \item The past boundary component of $\partial_-\mathcal{C}(\varphi_{\gamma_1,\gamma_2}^+)$ coincides with the union of the convex envelope of $\text{graph}(\restr{\gamma_1^{-1}}{I_1})$ and of the convex envelope of $\text{graph}(\restr{\gamma_2^{-1}}{I_2})$. 
        \item The future boundary component $\partial_+\mathcal{C}(\varphi_{\gamma_1,\gamma_2}^-)$ is the union of the convex envelope of $\text{graph}(\restr{\gamma_1^{-1}}{I_2})$ and of the convex envelope of $\text{graph}(\restr{\gamma_2^{-1}}{I_1}).$
    \end{itemize}
\end{proposition}

\begin{proof}
    Let us consider the case where $\gamma_2\circ\gamma_1^{-1}$ translates to the left along $\ell$, and let us prove the first item. Let $x,x^{\prime}$ be the fixed points of $\gamma_2\circ\gamma_1^{-1},$ let $y=\gamma_1^{-1}(x)=\gamma_2^{-1}(x)$ and $y^{\prime}=\gamma_1^{-1}(x^{\prime})=\gamma_2^{-1}(x^{\prime})$. Then the convex envelope of graph($\restr{\gamma_i^{-1}}{I_i}$) is a half-plane $A_i$ in $P_{\gamma_i}$ bounded by the geodesic $P_{\gamma_1}\cap P_{\gamma_2}$, which has endpoints $(x,y)$ and $(x^{\prime} ,y^{\prime})$, Clearly both the convex envelope of graph($\restr{\gamma_i^{-1}}{I_i}$) are contained in $\mathcal{C}(\varphi_{\gamma_1,\gamma_2}^{+}).$\\
    We could be even more precise. We claim that $P_{\gamma_1}$ and $P_{\gamma_2}$ are future support planes for $\mathcal{C}(\varphi_{\gamma_1,\gamma_2}^{+})$. The claim will imply that the union of $A_1$ and $A_2$ is contained in the future boundary component $\partial_+\mathcal{C}(\varphi_{\gamma_1,\gamma_2}^{+})$, because every point $p\in A_1\cup A_2$ admits a future support plane through $p$ which is either $P_{\gamma_1}$ or $P_{\gamma_2}$. However $A_1\cup A_2$ is a topological disc in $\partial_+\mathcal{C}(\varphi_{\gamma_1,\gamma_2}^{+}),$ whose boundary is precisely the curve $\text{graph}(\varphi_{\gamma_1,\gamma_2}^{+})$ by construction. Hence the claim will imply that $A_1\cup A_2=\partial_+(\varphi_{\gamma_1,\gamma_2}^{+}).$\\
    We prove the claim for $P_{\gamma_1}$, proof for $P_{\gamma_2}$ is analogous. For convenience, we set $\gamma_1=\text{Id}$ and $\gamma_2=\gamma$ is a loxodromic isometry with fixed points $x,x^{\prime}$, translating to the left as seen from $D_1$ to $D_2$. Indeed, we can apply $(\text{Id},\gamma_1),$ which sends $P_{\gamma_1}$ to $P_1$, $P_{\gamma_2}$ to $P_{\gamma_2\gamma_1^{-1}}$, and graph($\varphi_{\gamma_1,\gamma_2}^+$) to graph($\varphi_{\text{Id},\gamma_2\gamma_1^{-1}}^+$). \\
    We can now consider a path $\sigma_t,$ for $t\in[0,\epsilon]$ of elliptic elements fixing a given point $z_0\in\H^2,$ that rotate clockwise by an angle $t.$ As in the proof of Lemma \ref{Mati} the planes $P_{\sigma_t}$ are pairwise disjoint in $\overline{\A^{2,1}}$, because $\sigma_{t_2}\circ\sigma_{t_1}^{-1}$ is still an elliptic element fixing $z_0$ for $t_1\neq t_2$, hence it has no fixed points in $\S$. Moreover recall that $\gamma^{-1}$ has a fixed axis $\ell$ and translates along $\ell$ to the right as seen from $D_1$ to $D_2$. Now $\varphi_{\text{Id},\gamma}^+$ equals the identity on $I_1$ and $\gamma_2$ on $I_2,$ it fixes $I_1$ pointwise and moves points on $I_2$ clockwise. It follows that the equation $\varphi_{\text{Id},\gamma}^+(x)=\gamma_t^{-1}(x)$ has no solution for $t>0$, because $\sigma_{t}^{-1}=\sigma_{-t}$ moves all the points counterclockwise if $t$ is positive. This shows that $P_{\sigma_t}\cap\mathcal{C}(\varphi^+_{\text{Id},\gamma})=\emptyset$ for $t>0,$ and thus $P_{\text{Id}}$ is a support plane for $\mathcal{C}(\varphi^+_{\text{Id},\gamma})$ by Remark \ref{diffsupp}.\\
    Moreover if it is a future support plane: indeed one can check that $\sigma_{t+\pi/2}=\mathcal{R}_{z_0}\circ\sigma_t\in P_{\sigma_t}$, and the path $t\mapsto\sigma_t$ is future-directed because, from the discussion after \refeq{tangent}, its tangent vector is future-directed, hence $\mathcal{C}(\varphi_{\text{Id},\gamma}^+)$ is locally in the past of $P_{\text{Id}}$. \\
    We have shown the first item of the proposition, all the others follow with completely analogous arguments.
\end{proof}



From \ref{gettinthere} we get the following consequence that we state as a:

\begin{corollary}\ref{53}
    Suppose that $\gamma_2\circ\gamma_1^{-1}$ is a hyperbolic isometry that translates along $\ell$ to the left (resp. right), as seen from $D_1$ to $D_2$, and write $\gamma_2\circ\gamma_1^{-1}=\exp(\mathfrak{a})$ for some $\mathfrak{a}\in\mathfrak{sl}(2,\R)$. Let $p$ be a point in the future (resp. past) boundary components of $\mathcal{C}(\varphi_{\gamma_1,\gamma_2}^{+}).$ Then:
    \begin{itemize}
        \item If $p\in\text{int}(A_1)$, then $P_{\gamma_1}$ is the unique support plane of $\mathcal{C}(\varphi_{\gamma_1,\gamma_2}^{+})$ at $p$. 
        \item If $p\in\text{int}(A_2)$, then $P_{\gamma_2}$ is the unique support plane of $\mathcal{C}(\varphi_{\gamma_1,\gamma_2}^{+})$ at $p$. 
        \item If $p\in A_1\cap A_2= P_{\gamma_1}\cap P_{\gamma_2}$, the the support planes of $\mathcal{C}(\varphi_{\gamma_1,\gamma_2}^{+})$ at $p$ are precisely those of the form $P_{\sigma\gamma_1}$ where $\sigma=\text{exp}(t\mathfrak{a})$ for $t\in [0,1].$  
    \end{itemize}

    
\end{corollary}

Where we are still using the notation introduced in the proof of \ref{gettinthere}: $A_i\subset P_{\gamma_i}$ is the convex envelope of $\text{graph}(\restr{\gamma_i^{-1}}{I_i}$), an half-plane bounded by the geodesic $P_{\gamma_1}\cap P_{\gamma_2}.$ As expected a completely analogous statement could be formulated for $\mathcal{C}(\varphi_{\gamma_1,\gamma_2}^{-})$ but we restrict to the study of $\varphi_{\gamma_1,\gamma_2}^{+}$ for simplicity. 

\begin{proof}
    From Proposition \ref{gettinthere}, the pleated surface that we obtained as the union of $A_1\subset P_{\gamma_1}$ and $A_2\subset P_{\gamma_2}$ coincides with $\partial_+\mathcal{C}(\varphi_{\gamma_1,\gamma_2}^+)$ if $\gamma_2\circ\gamma_1^{-1}$ is a hyperbolic isometry that translates along $\ell$ to the left, and with $\partial_-\mathcal{C}(\varphi_{\gamma_1,\gamma_2}^+)$ if it translates to the right, by Proposition \ref{gettinthere}. \\
    The firs two items are then clear, since $P_{\gamma_i}$ are smooth surfaces, hence $A_i$ is smooth at any interior point, and therefore has a unique support plane at the point. The last item can be proved in the same spirit as Proposition \ref{gettinthere}. First, we can assume $\gamma_1=\text{Id}$ and $\gamma_2=\gamma$ is a hyperbolic isometry translating on the left (resp. right) along $\ell.$ By \refeq{suplane}, if $P_\sigma$ is a support plane at $p,$ then $p$ is in the convex hull of the pairs $(y,\sigma^{-1}(y))$ where $y$ satisfies the relation $\sigma^{-1}(y)=\varphi_{\text{Id},\gamma}^{\pm}(y).$ The only possibility is then for $p$ to lie in the geodesic connecting the points $(x,x)$ and $(x^{\prime},x^{\prime})$ in $\T,$ where $x,x^{\prime}$ are the fixed points of $\gamma$. Hence $\sigma$ must have the same fixed point as $\gamma$. It follows that $\sigma$ is then a loxodromic isometry with axis $\ell$ (or the identity). Moreover, following the same reasoning as in \ref{gettinthere}, $P_\sigma$ is in the future (resp. past) of $\mathcal{C}(\varphi_{\gamma_1,\gamma_2}^+)$ if and only if $\sigma$ translates on the left (resp. right), and its translation length is less than that of $\gamma.$ Hence $\gamma$ is of the form $\exp(t\mathfrak{a})$ for $t\in [0,1].$
\end{proof}

\section{Simple earthquake.}
We are finally arrived to the case of considering orientation-preserving homeomorphism obtained by combining two elements of $\PSL$. We want to show that in a similar setting the composition of the projections $\Pi_l^{\pm}$ and $\Pi_r^\pm$ provide the earthquake map as in Example \ref{simplequake}. In first instance this does not seem like a huge achievement as we are just recovering a simple earthquake map that we were already able to define explicitly. Nevertheless, the following proposition will be a key step to complete the proof of the earthquake theorem. 

\begin{proposition}\label{estensione}
    Let $\gamma_1,\gamma_2 \in \PSL$ be such that $\gamma_2\circ\gamma_1^{-1}$ is a hyperbolic isometry, and let $\Pi_l^\pm, \Pi_r^\pm$ be the projections associated with the convex envelope of $\varphi_{\gamma_1,\gamma_2}^+.$ Then:
    \begin{enumerate}
        \item $\Pi_l^\pm, \Pi_r^\pm:\partial_\pm\mathcal{C}(\varphi_{\gamma_1,\gamma_2}^+)\to\H^2$ are bijections.
        \item Assume that $\gamma_2\circ\gamma_1^{-1}$ translates along $\ell$ to the right (resp. left), as seen from $D_1$ to $D_2.$ Then the composition $\Pi_r^-\circ(\Pi_l^-)^{-1}:\H^2\to\H^2$ (resp. $\Pi_r^+\circ(\Pi_l^+))^{-1}:\H^2\to\H^2)$ is a left (resp. right) earthquake map extending $\varphi_{\gamma_1,\gamma_2}^+.$
    \end{enumerate} 
\end{proposition}

We remark that we are limiting to the case of $\varphi_{\gamma_1,\gamma_2}^+$ for simplicity and completely analogous results could be formulated in terms of $\varphi_{\gamma_1,\gamma_2}^-.$ Before proving it we remark that proposition \ref{estensione} holds for \textit{any choice}  of support planes that is needed to define the projections.

\begin{proof}
    For the first point, recall that $A_i\subset P_{\gamma_i}$, and that the union $A_1\cup A_2$ is the past (resp. future) boundary component for $\varphi_{\gamma_1,\gamma_2}^+$ if $\gamma_2\circ\gamma_1^{-1}$ translates along $\ell$ to the right (resp. left). \\
    Hence $\restr{(\Pi_l^\pm)}{\text{int}(A_i)}$ and $\restr{(\Pi_r^\pm)}{\text{int}(A_i)}$ are the restrictions of the projections associate with the totally geodesic plane $P_{\gamma_i}$ just as seen in Example \ref{413}. In particular, $\restr{(\Pi_l^\pm)}{\text{int}(A_i)}$ and $\restr{(\Pi_r^\pm)}{\text{int}(A_i)}$ are the restriction to int($A_i$) of global isometries of $\A^{2,1}$ (those defined by multiplication on the left or on the right by $\gamma_i^{-1}$) sending $P_{\gamma_i}$ to $P_{\text{Id}}$, post-composed with the usual isometry $\text{Fix}:P_{\text{Id}}\to\H^2$. It follows that the restrictions of projection map geodesic of $P_{\gamma_i}$ to geodesic of $\H^2$. More is true, due to Equation \refeq{simproj}, $\restr{(\Pi_l^\pm)}{\text{int}(A_i)}$ maps int($\partial(A_i))=\text{graph(\restr{\gamma_i^{-1}}{\text{int}(I_i)})}$ to $\text{int}(I_i).$ Hence $\Pi_l^\pm(\text{int}(A_i))=\text{int}(D_i)$. In similar fashion, $\Pi_r^\pm(\text{int}(A_i))=\gamma_1^{-1}(\text{int}(D_1))=\gamma_2^{-1}(\text{int}(D_2)).$\\
 We want to show that the projections are bijective. To do so we will show that the image of the geodesic $A_1\cap A_2=P_{\gamma_1}\cap P_{\gamma_2}$, via $\Pi_l^\pm$ is the geodesic $\ell=D_{1}\cap D_2$, while the image via $\Pi_r^\pm$ is the geodesic $\gamma_1^{-1}(\ell)=\gamma_2^{-1}(\ell).$ The definition of $\Pi_l^\pm$ and $\Pi_r^\pm$ on $A_1\cap A_2$ \textit{depends} on the choice of a support plane. We recall that we must choose the \textit{same} support plane at any point $p\in A_1\cap A_2$. Now, because of Corollary \ref{53}, the possible choices of support planes at $p$ are all of the form $P_{\sigma\gamma_1}$, for some $\sigma$ that has the same fixed points as $\gamma_2\circ\gamma_1^{-1}$, which are precisely the common endpoint of $I_1$ and $I_2$. \\
 We stay consistent with the notation of Lemma \ref{Mati}, thus the endpoints at infinity of $A_1\cap A_2$ are the points $(x,y)$ and $(x^{\prime},y^{\prime})$ where $x,x^{\prime} $ are the fices point of $\gamma_2\circ\gamma_1^{-1}$ (and of $\sigma)$. Again from Equation \refeq{simproj} we have (for any choice of $\sigma$ as in the third item of Corollary \ref{53}) $\Pi_l^\pm(x,y)=x$ and $\Pi_l^\pm(x^{\prime} ,y^{\prime} )?y^{\prime},$ from which it follows that $\Pi_l^\pm(A_1\cap A_2)=\gamma_1^{-1}(\ell)=\gamma_2^{-1}(\ell).$ \\
 We move now on item number two. Define $E\coloneqq\Pi_r^-\circ(\Pi_l^-)^{-1}$, which is a bijection of $\H^2$. Consider the geodesic lamination of $\H^2$ composed by the sole geodesic $\ell.$ Hence the strata of $\ell$ are $\text{int}{D_1},$ $\text{int}(D_2)$ and $\ell$. We will show that the comparison isometries $\text{Comp}(S,S^{\prime})\coloneqq(\restr{E}{S})^{-1}\circ\restr{E}{S^{\prime}}$ translate to the right or to the left seen from one stratum to another, according to as $ \gamma_2\circ\gamma_1^{-1}$ translates to the left or to the right seen from $D_1$ to $D_2$.\\
 Let us consider $S=\text{int}(D_1)$ and $S^{\prime}=\text{int}(D_2)$. Then, by Example \ref{413}, $E$ equals $\gamma_i^{-1}$ on int($D_i)$, because $(\Pi_l^\pm)^{-1}(\text{int}(D_i))=\text{int}(A_i)\subset P_{\gamma_i^{-1}}$. \\
 Hence the comparison isometry $\text{Comp}(\text{int}(D_1),\text{int}(D_2))$ equals $\gamma_1\circ\gamma_2^{-1}$, and it translates to the left (resp. right) seen from $\text{int}(D_1)$ to $\text{int}(D_2)$ exactly when $\gamma_2\circ\gamma_1^{-1}$, which is its inverse, translates to the right (resp. left).  
 \textcolor{blue}{The proof when one of the two strata} $S$ or $S^{\prime}$ is $\ell$ is completely analogous, by using the third item of Corollary \ref{53}. Indeed (via Remark 4.11), by any choice of $\sigma$ of the form $\sigma=\exp(t\mathfrak{a})$ with $t\in(0,1)$, Comp($\ell,\text{int}(D_2))=\sigma\circ\gamma_2^{-1}$ translates to the left (resp. right) seen from $\ell$ to $\text{int}(D_2)$, and Comp(int($D_1$),$\ell)=\gamma_1\circ\sigma^{-1}$ translates to the left (resp. right) seen from int($D_1$) to $\ell.$ If instead $\sigma=\exp(t\mathfrak{a})$ with $t\in\{0,1\}$, then $\sigma$ coincides with $\gamma_1$ or with $\gamma_2,$ hence on of the comparison isometries Comp(int($D_1),\ell$) and Comp(int($D,2),\ell$) translates to the left, while the other it the identity, which is still \textit{allowed} in the definition of earthquake because $\ell$ is in the boundary on $\text{int}(D_i)$.
\end{proof}

\section{The example is prototypical.} We have just treated what seems to be a very \textit{special and convenient} simple earthquake. What we want to show now is that it is actually the prototypical example, that will serve to treat the general case of the earthquake theorem. The following lemma explains \textit{how} the situation of two intersecting planes is actually pretty common. 

\begin{lemma}\label{condor}
    Let $\varphi:\S\to\S$ be an orientation-preserving homeomorphism which is not in $\PSL$. Then: 
    \begin{itemize}
        \item Any two support panes of $\mathcal{C}(\Lambda_\varphi)$ at points of $\partial_+\CF$ intersects in $\A^{2,1}$. Analogously, any two past support planes of $\CF$ at points of $\partial_-\CF$ intersects in $\A^{2,1}$.
        \item Given a point $p\in\partial_\pm\CF,$ if there exist two support planes at $p$, then their intersection (which is a spacelike geodesic) is contained in $\partial_\pm\CF$. As a consequence, any other support plane at $p$ contains this spacelike geodesic.
    \end{itemize}
\end{lemma}

\begin{proof}
    Let us consider future support planes, the other case being analogous, For the first item, let $P$ and $Q$ be support planes intersecting $\partial_+\CF$, which are spacelike by Proposition \ref{supportinho}, and suppose by contradiction that $P$ and $Q$ are disjoint. Then we can slightly move them in the future to get spacelike planes, $P^{\prime}, Q^{\prime}$ such that $P,Q,P^{\prime}$ and $Q^{\prime}$ are mutually disjoint and $P^{\prime}\cap \partial_+\CF=Q^{\prime}\cap\partial_+\CF=\emptyset.$ (For example, if $P=P_{\gamma_1}$ and $Q=P_{\gamma_2}$ then we can use Lemma \ref{Mati} and consider $P^{\prime}=P_{\sigma\gamma_1}$) for $\sigma$ an elliptic element of small clockwise angle of rotation.)\\
    Now notice that the complement of $P^{\prime}\cup Q^{\prime}$ in $\A^{2,1}$ is the disjoint union of two cylinders and $P$ and $Q$ are lie in different connected components of this complement. See \textcolor{blue}{Figura}. However, $\partial_+\CF$ is connected, and has empty intersection with $P$ and $Q$, leading to a contradiction. \\
    Let's move on to the second item. Let $P=P_{\gamma_1}$ and $Q=P_{\gamma_2}$ be support planes such that $p\in\partial_+\CF\cap P\cap Q.$ By Lemma \ref{Mati}, $\gamma_2\circ\gamma_1^{-1}$ is loxodromic.  Up to switching the roles of $\gamma_1$ and $\gamma_2$ we can assume that $\gamma_2\circ\gamma_1^{-1}$ translates to the left seen from $D_1$ and $D_2$, where as usual $D_i$ is the convex hull of the interval $I_i,$ and the common endpoints $x,x^{\prime}$ of $I_1$ and $I_2$ are the fixed points of $\gamma_2\circ\gamma_1^{-1}.$ Hence $\partial P_{\gamma_1}\cap\partial P_{\gamma_2}=\{(x,y),(x^{\prime}, y^{\prime})\}$ where $y=\gamma_1^{-1}(x)=\gamma_2^{-1}(x)$ and $y^{\prime}=\gamma_1^{-1}(x^{\prime})=\gamma_2^{-1}(x^{\prime})$. \\
    Now, via \refeq{suplane}, $P_{\gamma_i}\cap\text{graph}(f)$ consist of at least two points for $i=1,2$. We claim that the aforementioned interections contains at least $(x,y)$ and $(x^{\prime},y^{\prime})$. Indeed, since $P_{\gamma_2}$ is a support plane, $\CF\cap P_{\gamma_1}$ is contained in the half-plane $A_1 \subset P_{\gamma_1}.$ If $\text{graph}(f)\cap P_{\gamma_1}$ had not contained $(x,y)$ and $(x^{\prime} ,y^{\prime}),$ then $\CF\cap P_{\gamma_1}$ would not contain the boundary geodesic $A_1 \cap A_2$, and thus would not contain $p$. A \textit{verbatim} argument holds also for $P_{\gamma_2}$. This shows that both $(x,y)$ and $(x^{\prime},y^{\prime})$ are in $\text{graph}$, and therefore the spacelike geodesic $P_{\gamma_1}\cap P_{\gamma_2}$ is in $\partial_+\CF$. \\ 
\end{proof}

\begin{observation}
In the first item of \ref{condor}, the hypothesis that $P$ and $Q$ are support planes at points of $\partial_\pm\CF$ (and not at points of $\Lambda_\varphi$ $\subset\partial\A^{2,1}$) is necessary. In light of \ref{supportinho} we know support planes of $\CF$ are either spacelike or lightlike, and they are necessarily spacelike if they intersect $\CF$ at points of $\partial_\pm\CF$. \\
Now, if one of the two plane $P$ and $Q$ must intersect in $\overline{\A^{2,1}}$, but not necessarily in the interior. It can happen that two future (or past) support planes (one of which possibly lightlike) at a point $(x,\varphi(x))$ intersect at $(x,\varphi(x))$ but not in the interior of $\A^{2,1}$.
\end{observation}
 
The Lemma \ref{condor} allows us to notice the following. Recall that we have defined the left and right projections $\Pi_l^\pm, \Pi_r^\pm$, and they depended on the choice of a support plane at all points $p$ that admit more than one support plane. Moreover, we require that this support plane is chosen to be constant on any connected component of the subset of $\partial_\pm\CF$ consisting of points that admit more than one support plane. We want to show:

\begin{corollary}\label{multipleplanes}
    Let $\varphi:\S\to\S$ be and orientation-preversing homeomorphism which is not in $\PSL$, and suppose $p\in \partial_\pm\CF$ has at least two support planes. Then there exist $\gamma_1,\gamma_2\in \PSL,$ and suppose $p\in\partial_\pm\CF$ has at least two support planes. Then there exist $\gamma_1,\gamma_2\in\PSL$ such that $\gamma_2\circ\gamma_1^{-1}=\exp(\mathfrak{a})$ is a loxodromic element, such that all support planes at $p$ are precisely those of the form $P_{\sigma\gamma_1}$ where $\sigma=\exp(t\mathfrak{a})$ for $t\in [0,1].$ The same conclusion holds for all other point $p^{\prime}\in P_{\gamma_1}\cap P_{\gamma_2}$. \\
    In particular, the image of the spacelike geodesic $P_{\gamma_1}\cap P_{\gamma_2}$ under the projections $\Pi_l^\pm$ and $\Pi_r^\pm$ is a geodesic in $\H^2$ that does not depend on the choice of the support plane as in the definition of the projections.  
\end{corollary}

\begin{proof}
Suppose $P_{\widetilde{\gamma}_1}$ and $P_{\widetilde{\gamma}_2}$ are (say, future) distinct support planes at $p$. Write $ @\widetilde{\gamma_2}\circ\widetilde{\gamma_1}^{-1}=\exp(\widetilde{\mathfrak{a}}),$ which is a loxodromic element by Lemma \ref{Mati} and the first item of Lemma \ref{condor}. By the second item of Lemma \ref{condor}, any other support plane at $p$ must be of the form $P_{\sigma\widetilde{\gamma}_1}$ for $\sigma$ an element having the same fixed points as $\widetilde{\gamma_2}\circ\widetilde{\gamma_1}^{-1}$. That is, $\gamma$ is the form $\exp(s\widetilde{\mathfrak{a}})$ for some $s\in\R$.\\
We claim now that the set: 
\[
    I=\{s\in\R\;|\;\exp(s\widetilde{\mathfrak{a}})\;\text{is a support plane of}\;\CF\;\text{at}\;p\}
\]
is a compact interval. This would conclude the proof, up to applying an affine change of variable mapping to the interval $I=[s_1,s_2]$ to $[0,1]$, and defining $\gamma_i=\exp(s_i\widetilde{\mathfrak{a}}).$\\
Let's prove the compactness of $I$, suppose $s,s^{\prime} \in I$. Now $\CF$ is contained in the past of a pleated surface obtained as the union of two half-spaces, one contained in $P_{\exp(s\widetilde{\mathfrak{a}})\widetilde{\gamma}_1}$ and the other in $P_{\exp(s^{\prime}\widetilde{\mathfrak{a}})\widetilde{\gamma}_1}$, meeting along the spacelike geodesic $P_{\widetilde{\gamma}_1}\cap P_{\widetilde{\gamma}_2}$. Then every support plane for this pleated surface is a support plane for $\CF$ as well. That is, by the last item of Corollary \ref{53}, $[s,s^{\prime}]\subset I$. This shows that $I$ is an interval. It is compact by Lemma \ref{49}, applied to the constant sequence $p_n=p$ and to $\gamma_n=\exp(s_n\widetilde{\mathfrak{a}})$, showing that $s_n$ must by converging (up to subsequences) and its limit is in $I$. This concludes the proof.  

\end{proof}

\section{Proof of the earthquake theorem.}

We have now all the tools that are required for the proof of the earthquake theorem. We outline the strategy that we will follow: given an orientation-preserving homeomorphism $\varphi:\S\to\S$ (we can assume that it is not in $\PSL$), we consider the projections $\Pi_l^\pm,\Pi_r^\mp:\partial_\pm\CF\to\H^2$, and we want tho show that the composition $\Pi_r^\pm\circ(\Pi_l^\pm)^{-1}$  is well-defined and is a (left or right) earthquake map extending $f$. \textcolor{red}{outline the strategy}

\section{Extension to the boundary}
We study the extension of the projections $\Pi_l^\pm, \Pi_r^\pm$ to the boundary of $\CF$. 

\begin{proposition}\label{seiuno}
    The projections $\Pi_l^\pm, \Pi_r^\pm$ extend to $\Lambda_\varphi$. More precisely, if $p_n\in\partial_\pm\CF\to(x,y)\in\Lambda_\varphi,$ then $\Pi_l^\pm(p_n)\to x$ and $\Pi_r^\pm(p_n)\to y$.
\end{proposition}

We want to remark that the conclusion of \ref{seiuno} hold for \textit{any choice} of the projections, regardless of the chosen support planes where several choices are available. \textcolor{pink}{magari metterlo dopo la proof?}

\begin{proof}
    Let $p_n\in\partial_\pm\CF$ be a sequence converging to $(x,y)\in\Lambda_\varphi,$ and let $P_{\gamma_n}$ be a sequence of support planes of $\CF$ at $p_n$, which are necessarily spacelike because of the results of Proposition \ref{supportinho}. By Lemma \ref{49}, up to extracting a subsequence, there are two possibilities: either $\gamma_n\to\gamma$ and $P_{\gamma_n}$ converges to the spacelike support plane $P_\gamma$, or $\gamma_n$ diverges in $\PSL$ and $P_{\gamma_n}$ converges to the lightlike plane whose boundary is $(\{x\}\times\S)\cup(\S\times\{y\})$. We treat the two cases separately, and we remand to \ref{convergenza} as a reminder of our characterization of convergence to the boundary.
    We start by supposing the convergence $\gamma_n\to\gamma.$ By hypothesis: 
    \begin{equation}\label{18}
        p_n(z_0)\to x \;\;\; p_n^{-1}(z_0)\to y
    \end{equation}
    for any $z_0\in\H^2.$ It also follows from the definition of the projections: 
    \begin{equation}\label{19}
        \Pi_l^\pm(p_n)=\text{Fix}(p_n\gamma_n^{-1}) \;\; \text{and}\;\; \Pi_r^\pm(p_n)=\text{Fix}(\gamma_n^{-1}p_n).
    \end{equation}
  We consider the identification of $P_\text{Id}$ with $\S$ via $(x,x)\to x,$ we thus have show (choosing for instance the point $z_0=i$) that: $p_n\gamma_n^{-1}(i)\to x$ and $\gamma_n^{-1}p_n(i)\to y.$ \\
  However, since $\gamma_n\to\gamma$, $p_n\gamma_n^{-1}(i)$ is at bounded distance from $p_n\gamma^{-1}(i)$. Applying the hypothesis, namely \refeq{18}, to $z_0=\gamma^{-1}(i)$, we have $p_n\gamma^{-1}(i)\to x$ and therefore $p_n\gamma_n^{-1}(i)\to x$ as well. \\
  The argument is analogous to show that $\gamma_n^{-1}p_n(i)\to y$, except that it it useful to observe that $\gamma_n^{-1}p_n=p_n^{-1}\gamma_n$ since it is an order-two isometry. Now $p_n^{-1}\gamma_n(i)$ is at bounded distance from $p_n^{-1}\gamma(i)$, which converges to $y$ by hypothesis. Hence $p_n^{-1}\gamma_n(i)\to y$ as well. \\
  Let us move to the latter case, namely $\gamma_n$ diverges in $\PSL.$ Here we will use not only the assumption of \refeq{18}, but also: 
  \begin{equation}\refeq{20}
    \gamma_n(z_0)\;\text{and}\;\gamma_n^{-1}(z_0)\to y,
  \end{equation} 
  for any $z_0\in\H^2$. The condition of \refeq{20} holds because $\gamma_n$ converges to the projective class of a rank one matrix $A$, such that $P_{[A]}$ is a lightlike support plane; we have already observed that the boundary at infinity of $P_[A]$ must be equal to $(\{x\}\times\S)\cup(\S\times\{y\})$. Combining ,\ref{convergenza} and Lemma \ref{}, we deduce that $\gamma_n(z_0)\to x$ and $\gamma_n^{-1}(z_0)\to y$ as claimed. \\
  With this observation, we can rewrite \refeq{19} as the identities: 
  \begin{equation}\label{21}
    p_n=\mathcal{R}_{\Pi_l^\pm(p_n)}\circ\gamma_n \;\;\; \text{and}\;\;\; p_n^{-1}=\mathcal{R}_{\Pi_r^\pm(p_n)}\circ\gamma_n^{-1}, 
  \end{equation} 
  where we recall that for us $\mathcal{R}_w$ denotes the orderd two elliptic isometry with fixed point $w\in\H^2.$ \\
  Up to extracting a subsequence, we can assume that $\Pi_l^{\pm}(p_n)\to\widehat{x}_\pm$ and $\Pi_r^\pm{p_n}\to\widehat{y}_\pm$, for some points $\widehat{x}_\pm, \widehat{y}_\pm\in\H^2\cup\partial\H^2$. We need tho show that $\widehat{x}_\pm=x$ and $\widehat{y}_\pm=y$. \\
  In this direction, suppose by contradiction $\widehat{x}_\pm\neq x$. Suppose first that $\widehat{x}_\pm\in\H_2.$ We will use the fact (\textcolor{blue}{appendix})
  that if $w_n\to w\in \H^2,$ then $\mathcal{R}_{w_n}$ converges to $\mathcal{R}_{w_n}$ uniformly on $\H^2\cup\partial\H^2$. From \refeq{21}, and the fact that, from \refeq{18} and \refeq{20}, both $p_n(z_0)$ and $\gamma_n(z_0)$ converge to $x$, we would then have: 
  \[
    x=\lim_n p_n(z_0)=\lim_n(\mathcal{R}_{\Pi_l^\pm(p_n)}(\gamma_n(z_0))=\mathcal{R}_{\widehat{x}_\pm}(x)\neq x
  \]
  since $\mathcal{R}_{\widehat{x}_\pm}$ does not have fixed points on $\partial\H^2$, thus giving a contradiction. If $\widehat{y}_\pm\in\H^2$, the same argument works flawless. \\
  Lastly suppose $\widehat{x}_\pm\in\partial\H^2$, in this case we can find a neighbourhood $U$ of $\widehat{x}_\pm$ not containing $x$, such that for $n$ large $\mathcal{R}_{\Pi_l^\pm(p_n)}$ maps the complement of $U$ inside $U$ (again lemma in the appendix). This gives rise to a contradiction with the condition in \refeq{21} because $p_n(z_0)$ and $\gamma_n(z_0)$ are in the complement of $U$ for $n$ large, but at the same time $\mathcal{R}_{\Pi_l^{\pm}(p_n)}(\gamma_n(z_0))$ should be in $U$ for $n$ large. The argument for $\widehat{y}$ is the same \textit{verbatim}.
\end{proof}

\begin{observation}
In the proof we have not used the full hypothesis of a surface on which the projections are defined in a boundary component of $\Lambda_\varphi$, but only the property that whenever a sequence $P_{\gamma_n}$ of spacelike support planes converges to a lightlike plane, then this limit is a support plane too, which is true for any general convex surface.
\end{observation}



\begin{proposition}\label{seitre}
    The projections $\Pi_l^\pm, \Pi_r^\pm:\partial_\pm\CF\to\H^2$ are bijections. 
\end{proposition}

As a direct consequence of Proposition \ref{seitre}, the composition $\Pi_r^\pm\circ(\Pi_l^\pm)^{-1}$ is well-defined and is a bijection of $\H^2$ to itself. This and Proposition \ref{seiuno} let us show the following: 

\begin{corollary}
    The composition $\Pi_r^\pm\circ(\Pi_l^\pm)^{-1}$ extends to a bijection from $\H^2\cup\partial\H^2$ to itself, which equals $\varphi$ on $\partial\H^2$ and is continuous at any point of $\partial\H^2$.
\end{corollary}
\begin{proof}
    Since $\Pi_l^\pm$ and $\Pi_r^\pm$ are bijective and extend to the bijections from $\Lambda_\varphi$ to $\partial\H^2$ sending $x,y\to x$ and $y\to\varphi(x)$ respectively, the composition $\Pi_r^\pm\circ(\Pi_l^\pm)^{-1}$ extends to a bijection of $\H^2\cup\partial\H^2$ to itself sending $x\to \phi(x).$ \\
    We need to check continuity. Proposition \ref{seiuno} shows that the extensions $\Pi_l^\pm$ and $\Pi_r^\pm$ to $\partial_\pm\CF\cup\Lambda_\varphi$ are continuos at any point of $\Lambda_\varphi$. Hence it remains to check that $(\Pi_l^\pm)^{-1}$ is continuos at any point of $\partial\H^{2}$.\\
    This is pretty straightforward: let $z_n$ be a sequence in $\H^2\cup\partial\H^2$ converging to $x\in\partial\H^2$, and let $p_n=(\Pi_l^\pm)^{-1}(z_n)$. Up to extracting a subsequence, $p_n\to p$. The limit $p$ must be in $\Lambda_\varphi$ because if $p\in\partial_\pm\CF$, altough $\Pi_l^\pm$ might not be continuos there, we have already seen in Proposition \ref{seitre} (in the passage about closeness of the image of $\Pi_l^\pm$) that $\lim_n\Pi_l^\pm(p_n)=\lim_n z_n$ is a point of $\H^2$, thus giving a contradiction with $\lim_n z_n=x\in\partial\H^2$. \\ 
    If $p\in\Lambda_\varphi$, then we can use the continuity and injectivity of $\Pi_l^\pm$ on $\Lambda_\varphi$ to infer that $p=(\Pi_l^\pm)^{-1}(x)$.

\end{proof}

We are only left with verification that $\Pi_r^\pm\circ(\Pi_l^\pm)^{-1}$ satisfies the defining properties of an earthquake map. 

\begin{proposition}
    The composition $\Pi_r^-\circ(\Pi_l^-)^{-1}:\H^2\to\H^{2}$ is a left earthquake map. Analogously, $\Pi_r^+\circ(\Pi_l^+)^{-1}:\H^2\to\H^2$ is a right earthquake map.
\end{proposition}
\begin{proof}
    Let us start by defining a geodesic lamination $\lambda.$ Let us consider all the support planes $P_\gamma$ of $\CF$ at points of $\partial_\pm\CF$ (which we recall must be spacelike by Proposition \ref{supportinho}). Define $\mathcal{L}$ to be the collection of all the connected components of $(P_\gamma\cap\partial_\pm\CF)\setminus\text{int}(P_\gamma\cap\partial_\pm\CF),$ as $P_\gamma$ varies over all support planes. As observed before, by \textcolor{red}{questa va messa ancora}, $P_\gamma\cap \partial_\pm\CF$ is the convex hull in $P_\gamma$ of $\partial P_\gamma\cap\Lambda_\varphi,$ which consist of at least two points. If it consists of exactly two points, then $P_\gamma\cap \partial_\pm\CF$ is a spacelike geodesic $L$; otherwise $P_\gamma\cap\partial_\pm\CF$ has nonempty interior and each connected component of its boundary is a spacelike geodesic. Now, $\Pi_l^\pm$ is an isometry onto its image when restricted to any $L\in\mathcal{L}$ (wich might depend on the choice of a support plane if there are several support planes at point of $L$, but the image does not depend on this choice by Corollary \refeq{57}). Hence we define $\lambda$ to be the collection of alle the $\Pi_l^\pm(L)$ as $L$ varies over $\mathcal{L}$. \\
    To show that $\lambda$ is a geodesic lamination, we first observe that the geodesic $\ell\in\lambda$ are pairwise disjoint, because the spacelike geodesics $L$ in $\mathcal{L}$ are pairwise disjoint and $\Pi_l^\pm$ is injective. The in remains to show that their union is a closed subset of $\H^2$. But this follows immediately from the proof of Proposition \ref{seitre}. Indeed, suppose that $\ell_n=\Pi_l^\pm(L_n)$ converges to $\ell=\Pi_l^\pm(L)$, and let $z_n=\Pi_l^\pm(p_n)\in\ell_n$ be a sequence converging to $z\in\ell$. Since $\text{Im}(\Pi_l^\pm)$ is closed, $z\in\text{Im}(\Pi_l^\pm)$, and given the injectivity of $\Pi_l^\pm$, $z=\Pi_l^\pm(p)$ for some $p\in L$. The in the last part of the proof of Proposition \ref{seitre} we have shown that in this situation $\ell_n$ converges to $\ell$. \\
    Having shown that $\lambda$ is a geodesic lamination, we are ready to check that $@\pi_r^-\circ(\pi_l^-)^{-1}$ is an earthquake map. Observe that the gaps of $\lambda$ are precisely the images under $\Pi_l^\pm$ of the interior of the sets $P_\gamma\cap\partial_\pm\CF$ (when this intersection is not reduced to a geodesic), as $P_\gamma$ varies among all the support planes. \\
    Let $S_1$ and $S_2$ be two srata of $\lambda$, and let $\Sigma_i=(\Pi_l^\pm)^{-1})(S_i).$ Hence $\Sigma_i\subset P_{\gamma_i}\cap\partial_\pm\CF,$ where $P_{\gamma_i}$ is a support plane. As usual, there could be several support planes at points of $\Sigma_i$, and this can occour only if $\Sigma_i$ is reduced to a geodesic by Lemma \ref{condor}. Recalling from Remark \textcolor{blue}{aggiungere}  that the chose support plane is assumed to be constant alont $\Sigma_i$, we can suppose that $P_{\gamma_i}$ is the support plane chosen in the definition of $@\Pi_l^\pm$ and $\Pi_r^\pm$.\\
    Now we proceed as in the proof of injectivity in Proposition \ref{seitre}. Consider first the case that $\gamma_1\neq\gamma_2$. By Lemma \ref{51} and Lemma \ref{55}, $\gamma_2\circ\gamma_1^{-1}$ is a loxodromic isometry; let $D_1$ and $D_2$ be the convex envelopes in $\H^2$ of the two intervals $I_1$ and $I_2$ with endpoints the fixed points of $\gamma_2\circ\gamma_1^{-1}.$ Up to switching $\gamma_1$ with $\gamma_2$, we assume that $\gamma_2\circ\gamma_1^{-1}$ translates to the left seen from $D_1$ to $D_2$. Then $\restr{\Pi_l^\pm}{\Sigma_i}=\restr{(\widehat{\Pi})_l^\pm}{\Sigma_i}$ and $\restr{\Pi_r^\pm}{\Sigma_i}=\restr{(\widehat{\Pi})_r^\pm}{\Sigma_i}$, where $\widehat{\Pi}_l^\pm$ and $\widehat{\Pi}_r^\pm$ are the left and right projections associated with $\mathcal{C}(\Lambda_{\varphi_{\gamma_1,\gamma_2}^+})$ and moreover $S_i\subset D_i$.\\
    By the second part of Proposition \ref{54}, the comparison isometry $\widehat{\text{Comp}}(D_1,D_2)$ of the map $\widehat{\Pi}_r^\pm\circ(\widehat{\Pi}_l^\pm)^{-1}$ translates to the left (for $\Pi_r^-$ and $\Pi_l^-$) or right (for $\Pi_r^+$ and $\Pi_l^+$) seen from $D_1$ to $D_2$. Then $\text{Comp}(S_1,S_2)$, which is indeed equal to $\widehat{\text{Comp}}(D_1,D_2)$, translates to the left (or right) seen from $S_1$ to $S_2.$\\
    Finally, we consider the cae $\gamma_1=\gamma_2$, which can only happen either if $\Sigma_1=\Sigma_2$ (hence $S_1=S_2)$ or if $\Sigma_1$ has nonempty interior and $\Sigma_2$ is one of its boundary components (or vice versa exchanging $\Sigma_1$ with $\Sigma_2$). In this case we already have that $\text{Comp}(S_1,S_2)=\text{Id}$. But the comparison isometry is allowed to be the identity, when one of the two strata is contained in the closure of the other. This concludes the proof. 
\end{proof}

\subsection{Revovering earthquakes of closed surfaces.}
We have proved Thurston's theorem. We would like to recover the \textit{original} earthquake theorem due to Nielsen and recover the existence of earthquake maps between two homeomorphic closed hyperbolic surfaces.\\
We recall briefly the definition of equivariance given in \textcolor{red}{given where}. 

\begin{corollary}
    Let $S$ be a closed oriented surface and let $\rho,\varrho:\pi_1(S)\to\PSL$ be two Fuchsian representations. Then there exists a $(\rho,\varrho)-$equivariant left earthquake map of $\H^2$, and a $(\rho,\varrho)-$equivariant right earthquake map. 
\end{corollary}

\begin{proof}
    It's just a corollary of the classification of genus $g\geq 2$ MGHC Hyperbolic spaces.
\end{proof}