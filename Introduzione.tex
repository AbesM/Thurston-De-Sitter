\chapter{Introduction}

As in Riemannian geometry the \textit{most interesting} geometry of constant sectional curvature is the hyperbolic one, in Lorentzian geometry the analogus (as in with constant negative curvature) is the \textit{Anti-de Sitter} geometry. In the first chapter we will introduce the basic concepts of Lorentzian geometry and explore the relationships between general spaces with constant sectional curvature and spaces with maximal isometry group and constant sectional curvature.
We would then like to expand the study of Anti-de Sitter geometry in chapter two with a brief exposition of the most common models of AdS spaces and the study of geodesics in this manifolds. 
In chapter three we would like to specialize to $2+1$ Anti-de Sitter geometry fixing the dimension and working with explicit models, the main result of this section is the identification of AdS space with groups of matrices. 
In chapter four, with this "newly introduced" concepts, we would like to recover a celebrated theorem of Thurston related to hyperbolic surfaces following the pioneeristic work of G. Mess \cite{Mess}.