\chapter{Mess Work}

In his 1990 paper "Lorentz Spacetimes of Constant Curvature" \cite{Mess}, Geoffrey Mess offered a completely new approach to the study of spacetimes in 2+1-dimension by employing tools and techniques from low-dimensional geometry and topology. The aim of this chapter is to give a brief introduction to Mess ideas, with a special attention to AdS geometry. 

\section{Causality and Convexity properties}
We begin with some definitions:
\begin{definition}
    A subset $X$ of $\AS^{2,1}\cup \partial\AS^{2,1}$ is achronal (respectively acausal) if no pair of points in $X$ is connected by timelike (resp. causal) lines in $\AS^{2,1}$.
\end{definition}
Since acausality and achronality are conformally invariant notions, it will be often convenient to consider the metric $g_{\S^2}-dt^2$ on $\D\times\R$ introduced \todo{aggiustare.} which is confromal to the Poincaré model. We give now a first useful carachterization of achronal and acausal sets:
\begin{lemma}
    A subset $X$ of $\AS^{2,1}\cup \partial\AS^{2,1}$ is achronal (respectively acausal) if and only if it is the graph of a function $f:D\to\R$ which is 1-Lipschitz, (resp. strictly 1-Lipschitz) with respect to the distance induced by the hemispherical metric $g_{\S^2}$.\\
    Where we have denoted $D=\pi_{\D}(X).$ 
\end{lemma}
\begin{proof}
     Let's assume $X$ is achronal. Now, since vertical lines in the Poincaré model are of timelike type, the restrictions of the projections $\pi_\D:\D\times\R\to\D$ to $X$ are injective. But then, $X$ can be interpreted as the graph of a function $f:D\to\R$. By imposing that $(x,f(x))$ and $(y,f(y))$ are not connected by a timelike curve we deduce that: 
     \begin{equation}\label{soloqua}
        |f(x)-f(y)|\leq d_{\S^2}(x,y)
     \end{equation}
     where $d_{\S^2}$ is the distance induced by the hemispherical metric. By the same reasonment we show that a 1-lipschit< graph over $\D$ is achronal. Moreover two points $(x,t)$ and $(y,s)$ are on the same lightlike geodesic if and only if $d_{S^2}(x,y)=|t-s|$. Hence $X$ is acausal if and only if the inequality \refeq{soloqua} is strict. \
\end{proof}

Now a 1-Lipschitz function on a region $D\subset \D$ extends uniquely to the boundary of $D$. As a simple consequence to the previous lemma, we thus have: 

\begin{lemma}\label{achronalgraph}
    An achronal subset $X$ in $\AS^{2,1}$ is properly embedded if and only if it is a global graph over $\D$, and in this case it extends uniquely to the global graph of a 1-Lipschitz function over $\D\cup\partial\D$.
\end{lemma}

Because of \ref{achronalgraph} we will often refer to an \textit{achronal surface} as an achronal subset $X\subset\AS^{2,1}$ which is the graph of a 1-Lipschitz function defined in a domain in $\D.$ Before moving over to the study of propertis we shall remark how achronality and acasuality are global conditions.

\begin{definition} Given a surface $S$ and a Lorentzian manifold $(M,g),$ a $\mathcal{C}^1$ immersion $\sigma:S\to M$ is \textit{spacelike} if the pull-back metric $\sigma^*g$ is Riemannian. If $\sigma$ is an embedding, we refer to its image as a \textit{spacelike surface}.
\end{definition}

A spacelike surface $S$ is locally acasual, but there are examples of spacelike surfaces which are not achronal (hence a fortiori not acasual), a fact that highlights the global character of the achronality condition. On the other hand the following is true: 

\begin{lemma}
    Any properly embedded spacelike surface in $\AS^{2,1}$ is acasual. 
\end{lemma}
\begin{proof}
    By \ref{achronalgraph}, any properly embedded spacelike surfaces $S$ in $\AS^{2,1}$ disconnects the spaces in two regions $U,V$ whose common boundary is $S$, and we can assume that the outward pointing normal from $U$ (resp. $V$) is past-directed (resp. future directed). It then turns out that any future oriented causal path that meets $S$ passes from $V$ towards $U$. This implies that any causal path meets $S$ at most once. \todo{are we trying to model the fact that this is the present event?}
\end{proof}

\textcolor{red}{saltata un bel po' di roba}

\subsection{Domain of dependece}
We want to define a study properties of Cauchy surfaces and domains of dependece, a priori concept of Lorentianz geometry, in $\AS^{2,1}$

\begin{definition}
    Given an achronal subset $X$ in a Lorentzian manifold $(M,g)$ the \textit{domain of dependence} of $X$ is the set: 
    \[
        \mathcal{D}(X)=\{p\in M | \text{every inextensibile causal curve through} p \text{meets} X \}.
    \]
    We say that $X$ is a \textit{Cauchy surface} of $M$ if $\mathcal{D}(X)=M$. A spacetime $M$ is said \textit{globally hyperbolic} if it admits a Cauchy surface.
\end{definition}

\begin{section}
    We will now introduce the \textit{Gauss Map} associated to a spacelike surface in Anti de Sitter space, this tool will reveal useful in the study of the relation of Anti-de Sitter geometry and Teichmueller theory and hyperbolic surfaces.

    We will start by fixing notation and recalling some general facts about immersed spacelike surfaces in Anti de Sitter space. We will assume that all our immersed surfaces are of class $C^1$. \\
    Given a regular immersion $\sigma: S\to\A^{2,1}$ we will say that it is \textit{spacelike} if the pull-back metric  


\textcolor{red}{maybe here something is repeated lol}

The tangent bundle $TS$ is naturally identified to a subbundle of $T\A^{2,1}$ by means of $d\sigma$. The normal bundle $N_\sigma$ is then defined as the $g_{\A}$-orthogonal complement of $TS$ in $T\A^{2,1}$, The restricion of $g_{\A}$ to $N_\sigma$ is negative definite. Using the $g_\A$ induced decomposition: 
\[
\sigma^*T\A^{2,1}=TS\oplus N_\sigma,    
\]

the pull-back of the ambient Levi-Civita connection $\nabla$, restricted to sections tangent to $S$ splits as the sum of the Levi-Civita connection $\nabla^I$ of the first fundamental form $I$, and a symmetric 2-form with value in $N_\sigma$. As a consequence of time-orientability of $\A^{2,1}$, the normal bundle admits a natural trivialization, which is the same as a choice of a continuos unit normal vector field for $\sigma.$ We will denote by $\nu:S\to N_\sigma$ the future-directed unit normal section, and consider the decomposition for all vector field $X,Y$ tangent to $S$: 
\[
    \nabla_V W=\nabla^I_V W+II(V,W)\nu,
\]
where the symmetric $(2,0)$ tensor $II$ is called \textit{second fundamental form}. It will reveal useful to consider the $I-$symmetric $(1,1)$-tensor $B\in (TS)^*\otimes TS$ defined by $II(V,W)=I(B(V),W)$ which is called the \textit{shaper operator} of $\sigma.$ As in the Riemannian case, it turns out that $\sigma_*(B(v))=\nabla_v\nu.$\\  
The first and second fundamental form of an immersion $\sigma$ satisfy contrainst equations, known as $\textit{Gauss-Codazzi equations}$, More preciselly the Gauss equation consists of the identity: 
\[K_I=-1-\det_I II\] where $K_I$ is the curvature of $I$ and $\det_I II$ is $\det B$ by definition. On the other hand the Codazzi equation states that $\nabla^III$ is a totally symmetric $3-$form. We have: 
\begin{equation}
    
\end{equation}  