\chapter{Mess Work}

In his 1990 paper "Lorentz Spacetimes of Constant Curvature" \cite{Mess}, Geoffrey Mess offered a completely new approach to the study of spacetimes in 2+1-dimension by employing tools and techniques from low-dimensional geometry and topology. The aim of this chapter is to give a brief introduction to Mess ideas, with a special attention to AdS geometry. 

\section{Causality and Convexity properties}
We begin with some definitions:
\begin{definition}
    A subset $X$ of $\AS^{2,1}\cup \partial\AS^{2,1}$ is achronal (respectively acausal) if no pair of points in $X$ is connected by timelike (resp. causal) lines in $\AS^{2,1}$.
\end{definition}
Since acausality and achronality are conformally invariant notions, it will be often convenient to consider the metric $g_{\mathbb{S}^2}-dt^2$ on $\D\times\R$ introduced \todo{aggiustare.} which is conformal to the Poincaré model. We give now a first useful characterization of achronal and acausal sets:
\begin{lemma}
    A subset $X$ of $\AS^{2,1}\cup \partial\AS^{2,1}$ is achronal (respectively acausal) if and only if it is the graph of a function $f:D\to\R$ which is 1-Lipschitz, (resp. strictly 1-Lipschitz) with respect to the distance induced by the hemispherical metric $g_{\mathbb{S}^2}$.\\
    Where we have denoted $D=\pi_{\D}(X).$ 
\end{lemma}
\begin{proof}
     Let's assume $X$ is achronal. Now, since vertical lines in the Poincaré model are of timelike type, the restrictions of the projections $\pi_\D:\D\times\R\to\D$ to $X$ are injective. But then, $X$ can be interpreted as the graph of a function $f:D\to\R$. By imposing that $(x,f(x))$ and $(y,f(y))$ are not connected by a timelike curve we deduce that: 
     \begin{equation}\label{soloqua}
        |f(x)-f(y)|\leq d_{\mathbb{S}^2}(x,y)
     \end{equation}
     where $d_{\mathbb{S}^2}$ is the distance induced by the hemispherical metric. By the same reasonment we show that a 1-lipschit< graph over $\D$ is achronal. Moreover two points $(x,t)$ and $(y,s)$ are on the same lightlike geodesic if and only if $d_{\mathbb{S}^2}(x,y)=|t-s|$. Hence $X$ is acausal if and only if the inequality \refeq{soloqua} is strict. \
\end{proof}

Now a 1-Lipschitz function on a region $D\subset \D$ extends uniquely to the boundary of $D$. As a simple consequence to the previous lemma, we thus have: 

\begin{lemma}\label{achronalgraph}
    An achronal subset $X$ in $\AS^{2,1}$ is properly embedded if and only if it is a global graph over $\D$, and in this case it extends uniquely to the global graph of a 1-Lipschitz function over $\D\cup\partial\D$.
\end{lemma}

Because of \ref{achronalgraph} we will often refer to an \textit{achronal surface} as an achronal subset $X\subset\AS^{2,1}$ which is the graph of a 1-Lipschitz function defined in a domain in $\D.$ Before moving over to the study of propertis we shall remark how achronality and acasuality are global conditions.

\begin{definition} Given a surface $S$ and a Lorentzian manifold $(M,g),$ a $\mathcal{C}^1$ immersion $\sigma:S\to M$ is \textit{spacelike} if the pull-back metric $\sigma^*g$ is Riemannian. If $\sigma$ is an embedding, we refer to its image as a \textit{spacelike surface}.
\end{definition}

A spacelike surface $S$ is locally acasual, but there are examples of spacelike surfaces which are not achronal (hence a fortiori not acasual), a fact that highlights the global character of the achronality condition. On the other hand the following is true: 

\begin{lemma}
    Any properly embedded spacelike surface in $\AS^{2,1}$ is acasual. 
\end{lemma}
\begin{proof}
    By \ref{achronalgraph}, any properly embedded spacelike surfaces $S$ in $\AS^{2,1}$ disconnects the spaces in two regions $U,V$ whose common boundary is $S$, and we can assume that the outward pointing normal from $U$ (resp. $V$) is past-directed (resp. future directed). It then turns out that any future oriented causal path that meets $S$ passes from $V$ towards $U$. This implies that any causal path meets $S$ at most once. \todo{are we trying to model the fact that this is the present event?}
\end{proof}

\textcolor{red}{saltata un bel po' di roba}

\subsection{Domain of dependence}
We want to define and study properties of Cauchy surfaces and domains of dependence, a priori concept of Lorentianz geometry adapted to the geometry of $\AS^{2,1}$

\begin{definition}
    Given an achronal subset $X$ in a Lorentzian manifold $(M,g)$ the \textit{domain of dependence} of $X$ is the set: 
    \[
        \mathcal{D}(X)=\{p\in M |\;\text{every inextensibile causal curve through}\; p\; \text{meets} X \}.
    \]
    We say that $X$ is a \textit{Cauchy surface} of $M$ if $\mathcal{D}(X)=M$. A spacetime $M$ is said \textit{globally hyperbolic} if it admits a Cauchy surface.
\end{definition}

The theory of globally hyperbolic spacetimes is a well-developed topic in Lorentzian geometry, we will just state the facts that we will need in the following. 

\begin{theorem}
    Let M be a globally hyperbolic spacetime. Then:
    \begin{enumerate}
        \item Any two Cauchy surfaces in $M$ are diffeomorphic. 
        \item There exists a submersion $\tau:M\to\R$ whose fibers are Cauchy surfaces.
        \item M is diffeomorphic to $\Sigma\times\R$ where $\Sigma$ is any Cauchy surface in M.
    \end{enumerate}
\end{theorem}

\begin{observation}
    The spacetime $\AS^{2,1}$ is not globally hyperbolic. In fact if $X$ is achronal, it is contained in the graph of a 1-Lipschitz function \textcolor{red}{reference?} $f:(\D\cup\partial\D,g_{\mathbb{S}^2})\to\R.$ If $t_0>\text{sup}f$ and $\xi \in \partial\D$, then any lightlike ray with past end-point $(\xi,t_0)$ does not intersect $X.$ 
\end{observation}

\begin{observation}
    As causality notions are invariant under conformal change of metrics, we observe that causal paths in $\AS^{2,1}$ are the graphs of 1-Lipschitz functions from (intervals in) $\R$ to $\D$ with respect to the hemispherical metric in the image. Hence an inextesible causal curve in $\AS^{2,1}$ is either the graph of a global 1-Lipschitz function from $\R$, or it is defined on a proper interval and has endpoint(s) in $\partial\AS^{2,1}.$
\end{observation}

\begin{lemma}
    Given an achronal meridian $\Lambda$ in $\partial\AS^{2,1}$ any Cauchy surface in $\Omega(\Lambda)$ is properly embedded with boundary at infinity $\Lambda$. 
\end{lemma}
\begin{proof}
    
\end{proof}

\begin{proposition}
    Let $\Lambda$ be an achronal meridian in $\partial\AS^{2,1}$ different from the boundary of a lightlike plane. Let S be a properly embedded achronal surface in $\Omega(\Lambda)$. Then $\mathcal{D}(S)=\Omega(\Lambda)$. In particular $\Omega(\Lambda)$ is a globally hyperbolic spacetime.
\end{proposition}
\begin{proof}
    
\end{proof}


\textcolor{red}{Remark 4.4.7 saltato}

\begin{corollary}
    If $S$ and $S^{\prime}$ are properly embedded spacelike surfaces in $\AS^{2,1}$ , then $\mathcal{D}(S)=\mathcal{D}(S^{\prime})$ if and only if $\partial S=\partial S^{\prime}$.
\end{corollary}

\subsection{Properly achronal sets in $\A^{2,1}$}
For our interest will be important to consider the model $\A^{2,1}$. As $\A^{2,1}$ contains closed timelike lines, it does not contain any achronal subset. However if $P$ is a spacelike plane in $\A^{2,1}$ then $\A^{2,1}\setminus P$ does not contain closed causal curves. Indeed it is simply connected, so it admits an isometric embedding into $\AS^{2,1}$, given by a section of the covering map $\AS^{2,1}\to\A^{2,1},$ and whose image is a Dirichlet region \textcolor{red}{what is a dirichlet region?}

\begin{definition}
    A subset $X$ of $\A^{2,1}\cup\partial\A^{2,1}$ is a \textit{proper achronal subset} if there exists a spacelike plane $P$ such that $X$ is contained in $\A^{2,1}\cup\partial\A^{2,1}\setminus\overline{P}$ and is achronal as a subset of $\A^{2,1}\cup\partial\A^{2,1}\setminus\overline{P}.$  
\end{definition}

\textcolor{red}{Now they notice relations between dirichlet region and this proper achronal subsets}

We provide now an example that will be useful in the following:

\begin{lemma}
    Let $\varphi:\R\text{P}^1\to\R\text{P}^1$ be an orientation preserving homeomorphism. Then the graph of $\varphi$, say $\Gamma_\varphi\subset \R\text{P}^1\times\R\text{P}^1\simeq\partial\A^{2,1}$ is a proper achronal subset and any lift $\tilde{\Gamma_\varphi}$ is an achronal meridian in $\partial\AS^{2,1}.$  
\end{lemma}
\begin{proof}
\end{proof}

With what we have shown until now for achronal sets in $\AS^{2,1}$ can be rephrased for proper achronal sets of $\A^{2,1}$. \text{red}{saltato il lemma particolare che dice lui, praticamente saltata la pagina 34}\\

\begin{proposition}
    Let $\sigma: S\to\A^{2,1}$ be a proper spacelike immersion. Then 
    \begin{itemize}
        \item $\sigma$ is a proper embedding.
        \item $\sigma$ lifts to a proper embedding $\tilde{\sigma}:S\to\AS^{2,1}$.
        \item The boundary at infinity of $\sigma(S)$ is a proper achronal meridian $\Lambda$ in $\partial\A^{2,1}$.
        \item $\mathcal{D}(\sigma(S))=\Omega(\Lambda).$
    \end{itemize}
\end{proposition}

\begin{proof}
\end{proof}

\begin{observation}
    In the proof of the last proposition, once we proved that $\hat{S}$ is homeomorphic to $\R^2,$ we could have inferred immediately that $\hat{S}=S$ as $\Z/2\Z$ cannot act freely on $\R^2$ by diffeomorphism. 
\end{observation}

We therefore have an analogue version of \text{red}{citare lemma analogo} in $\A^{2,1}$.

\begin{corollary}
    If $S$ and $S^{\prime}$ are properly embedded spacelike surfaces in $\A^{2,1}$, then $\mathcal{D}(S)=\mathcal{D}(S^{\prime})$ if and only if $\partial S=\partial S^{\prime}$. 
\end{corollary}

\subsection{Convexity properties.}

\section{Globally Hyperbolic three-manifolds}
We want now to focus our attention on maximal globally hyperbolic (MGH) Anti-de Sitter spacetimes containing a compact Cauchy surfaces of genus $r$ (we will with a slight abuse of notation say that the spacetime has genus $r$.) We will show that there are no such manifolds when $r=0$. Furthermore, we will then give a rapid overview over the case $r=1$ and the focus on the most interesting case, $r\geq 2$, that will lead to a complete classification.

\subsection{General facts.} We will now state some general that will be useful in our classification.

\begin{lemma}\label{properembedding}
    Let $\sigma:S\to\A^{2,1}$ be a spacelike immersion. If $\sigma^*(g_{\A^{2,1}})$ is a complete Riemannian metric, then $\sigma$ is a proper embedding and $S$ is diffeomorphic to $\R^2$
\end{lemma}
\begin{proof}
    
\end{proof}

As an immediate consequence there cannot be any globally hyperbolic spacetime with genus $0$. In fact, suppose such a spacetime exists and denote by $\Sigma$ the Cauchy surfaces diffeomorphic to $\mathbb{S}^2$, the developing map restricted to $\Sigma$ would be a spacelike immersion, and the pull-back metric would be complete by compactness. But this contradicts \ref{properembedding}. Hence: 

\begin{corollary}
    There exists no globally hyperbolic Anti-de Sitter spacetime of genus 0. 
\end{corollary}

The following is a fundamental result on the structure of globally hyperbolic AdS spacetimes.

\begin{proposition}\label{holorep}
    Let $M$ be a globally hyperbolic Anti-de Sitter spacetimes of genus $r\geq 1$. Then: 
    \begin{enumerate}
        \item The developing map $\text{dev}:\tilde{M}  \to  \A^{2,1}$ is injective. 
        \item If $\Sigma$ is a Cauchy surface of $M,$ then the image of dev is contained in $\Omega(\Lambda)$, where $\Lambda$ is boundary at infinity of $\text{dev}(\tilde{\Sigma}).$
        \item If $\rho:\pi_1(M)\to\text{Isom}(\A^{2,1,})$ is the holonomy representation, $\rho(\pi_1(M))$ acts freely and properly discontinuosly on $\Omega(\Lambda)$, and $\Omega(\Lambda)$ is a globally hyperbolic spacetime containing $M$. 
    \end{enumerate}
\end{proposition}

\begin{proof}
    
\end{proof}

A remarkable difference between Lorentianz and Riemannian geometry is the that in Lorentianz geometry completeness is a very strong assumption, \textcolor{red}{reference} and in fact interesting classification results are obtained removing such condition. However, it is necessary to impose some maximality condition to compensate for non-completeness. Among several  approaches, one of the most common is the classification of a maximal globally hyperbolic spacetimes. Following \cite{bonsanteseppi} we restrict to a less general setting, but one could give similar definitions in the lager class of Einstein spacetimes.

\begin{definition}
    A globally hyperbolic Anti-de Sitter manifold $(M,g)$ is \textit{maximal} if any isometric embedding of $(M,g)$ into a globally hyperbolic Anti-de Sitter manifold $(M^{\prime},g^{\prime})$ which sends a Cauchy surface of $M$ to a Cauchy surface of $M^{\prime}$ is surjective.
\end{definition}

In eyesight of this definition and as a direct consequence of propostion \ref{holorep} we have:

\begin{corollary}
An Anti-de Sitter globally hyperbolic spacetime $M$ is maximal is and only if $\tilde{M}$ is isometric to the invisibile domain of a proper achronal meridian in $\partial\A^{2,1}.$ \textcolor{red}{importante da capire credo}
\end{corollary}


%genere uno 

\subsection{Examples of genus $r\geq 2$}
Let's $\Sigma_r$ be an oriented surface of genus $r\geq 2.$ We recall the definiton of Fuchsian representation. 

\begin{definition}
    A representaion $\rho:\pi_1(S)\to \text{PSL}(2,\R)$ is called positive \textit{Fuchsian} if there is a $\rho$-equivariant orientation preserving homeomorphism $\delta:\widetilde{\Sigma}_r\to\H^2$.
\end{definition}

The definition is invariant under conjugation in $\text{PSL}(2,\R)\simeq\text{Isom}_0(\H^2),$ but not under conjugation in $\text{Isom}(\H^2).$ It is a result of Goldman \textcolor{red}{Reference}, that a representation $\rho$ is positive Fuchsian if and only if the associated flat $\R\text{P}^1$ bundle $E_\rho$ constructed as the quotient of $\widetilde{\Sigma}_r\times\R\text{P}^1$ by the diagonal action of $\pi(S)$ (via deck transformation and the representation) has Euler class $2-2r.$ An equivalent condition is the existence of an orientation-preserving fiber bundle isomorphism between $E_\rho$ and the unit tangent bundle of $\Sigma_r.$ \\   

Another useful tool for the classification of genus $r\geq 2$ is the following fact in Teichm\"uller theory \textcolor{red}{reference}:

\begin{lemma}
    Given two positive Fuchsian representation $\rho_l,\rho_r:\pi_1(\Sigma_r)\to\text{PSL}(2,\R)$, any $(\rho_l,\rho_r)$ equivariant orientation-preserving homeomorphism of $\H^2,$ extends continuously to an orientation-preserving homeomorphism of $\H^2\cup\R\text{P}^1.$ Moreover, its extension $\varphi:\R\text{P}^1\to\R\text{P}^1$ is the unique $(\rho_l,\rho_r)-$equivariant orientation homeomorphism of $\R\text{P}^1$.     
\end{lemma}

Here by ($\rho_l,\rho_r$)-equivariant we mean that $\varphi\circ\rho_l=\rho_r\circ\varphi.$\\
Now let $\rho_l,\rho_r$ be two positive Fuchsian representation of $\Sigma_r.$ We will be interested in the representation:

\[
    \rho=(\rho_l,\rho_r):\pi_1(S)\to\text{Isom}_0(\A^{2,1})\simeq \text{PSL}(2,\R)\times\text{PSL}(2,\R).
\]

\begin{definition}
    Given a pair of positive Fuchsian representation $\rho_l,\rho_r:\pi_1(\Sigma_r)\to\text{PSL}(2,\R)$ we define $\Lambda(\rho)$ to be the graph in $\R\text{P}^1\times\R\text{P}^1$ of the unique $(\rho_l,\rho_r)-$equivariant orientation-preserving homeomorphism of $\R\text{P}^1,$ and $\Omega_\rho \coloneqq \Omega(\Lambda(\rho))$ its invisibile domain in $\A^{2,1}$
\end{definition}

Using the above construction, we can build examples of MGH spacetimes having holonomy any $\rho=(\rho_l,\rho_r)$ of this form. 

\begin{proposition}
    The domain $\Omega_\rho$ is invariant under the isometric action of $\pi_1(\Sigma_r)$ on $\A^{2,1}$ induced by $\rho.$ Moreover $\pi_1(\Sigma_r)$ acts freely and properly discontinuosly on $\Sigma_\rho$ and the quotient is a MGH spacetime of genus $r$ and holonomy $\rho.$ 
\end{proposition}
\begin{proof}
    \textcolor{red}{da adesso in poi bisogna capire per forza.}
\end{proof}

\subsection{Classification of genus $r\geq 2$.} We will now show that the examples of Proposition \textcolor{red}{citare} are \textit{all} the MGH spacetimes of genus $r$. 

\begin{lemma}
    Let $\rho:(\rho_l,\rho_r)$ be a pair of Fuchsian representations, and $\varphi:\R\text{P}^1\to\R\text{P}^1$ be the unique $(\rho_l,\rho_r)-$equivariant orientation-preserving homeomorphism of $\R\text{P}^1.$ Then $\Lambda(\rho)$ is the unique propre achronal meridian in $\partial\A^{2,1}$ invariant under the action of $\pi_1(\Sigma_r)$ induced by $\rho.$
\end{lemma}

\begin{corollary}
    For any pair $\rho=(\rho_l,\rho_r)$ of positive Fuchsian representations of $\pi_1(\Sigma_r)$, $M_\rho$ is the unique MGH spacetime with holonomy $\rho$.
\end{corollary}

We are only left with one last step for the classification result, we want to show that the left and right holonomies are necessarily positive Fuchsian. 

\begin{proposition}
    Let $M$ be an oriented, time-oriented, globally hyperbolic spacetime of genus $r\geq 2$ and let us endow a Cauchy surface $\Sigma$ with the orientation induced by the future normal vector. Then the left and right components of the holonomy $\rho=(\rho_l,\rho_r):\pi_1(\Sigma)\to\text{PSL}(2,\R)\times\text{PSL}(2,\R)$ are positive Fuchsian representation. 
\end{proposition}

\begin{observation}
We refer to the honolomy $\rho$ with respect to an orientation-preserving developing map. Therefore $\rho$ is well-defined up to conjugacy in $\text{PSL}(2,\R)\times\text{PSL}(2,\R)$
\end{observation}
\begin{proof}
    
\end{proof}

We can now state the classification result. We denote the \textit{deformation space} of MGH spacetimes of genus $r$ by:

\[
    \mathcal{MGH}(\Sigma_r)=\{g\;\text{MGH AdS metric on}\;\Sigma_r\times\R\}/\text{Diff}_0(\Sigma_r\times\R)
\]
where the group of diffeomorphism isotopic to the identity is acting by pill-back. The holonomy map takes value in the space of representation of $\pi_1(\Sigma_r)$ into $\PSL\times\PSL$ and is well-defined on the quotient $ \mathcal{MGH}(\Sigma_r).$
As a consequence of Proposition [...], the left and right components of the holonomy of elements of $ \mathcal{MGH}(\Sigma_r)$ are positive Fuchsian representations. The space of these rrepresenations up to conjugacy is idetified with the Teichm\"uller space of $\Sigma_r$ by the aforementioned work of Goldman []:
\[
    \mathcal{T}(\Sigma_r)\simeq\{\rho:\pi_1(\Sigma_r)\to\PSL\;\text{positive Fuchsian representation}\}/\PSL
\]

Therefore the holonomy map can be cosidered as a map from $ \mathcal{MGH}(\Sigma_r)$ with values in $\mathcal{T}(\Sigma_r)\times\mathcal{T}(\Sigma_r).$ Restating the classification in the original set of Mess \cite{Mess}: 
\begin{theorem}
    The holonomy map $\rho:\mathcal{MGH}(\Sigma_r)\to\mathcal{T}(\Sigma_r)\times\mathcal{T}(\Sigma_r)$ is a homeomorphism.
\end{theorem}
\section{Gauss map of spacelike surfaces}
    We will now introduce the \textit{Gauss Map} associated to a spacelike surface in Anti de Sitter space, this tool will reveal useful in the study of the relation of Anti-de Sitter geometry and Teichm\"uller theory and hyperbolic surfaces.

    We will start by fixing notation and recalling some general facts about immersed spacelike surfaces in Anti de Sitter space. We will assume that all our immersed surfaces are of class $C^1$. \\
    Given a regular immersion $\sigma: S\to\A^{2,1}$ we will say that it is \textit{spacelike} if the pull-back metric  


\textcolor{red}{maybe here something is repeated lol}

The tangent bundle $TS$ is naturally identified to a subbundle of $T\A^{2,1}$ by means of $d\sigma$. The normal bundle $N_\sigma$ is then defined as the $g_{\A}$-orthogonal complement of $TS$ in $T\A^{2,1}$, The restricion of $g_{\A}$ to $N_\sigma$ is negative definite. Using the $g_\A$ induced decomposition: 
\[
\sigma^*T\A^{2,1}=TS\oplus N_\sigma,    
\]

the pull-back of the ambient Levi-Civita connection $\nabla$, restricted to sections tangent to $S$ splits as the sum of the Levi-Civita connection $\nabla^I$ of the first fundamental form $I$, and a symmetric 2-form with value in $N_\sigma$. As a consequence of time-orientability of $\A^{2,1}$, the normal bundle admits a natural trivialization, which is the same as a choice of a continuos unit normal vector field for $\sigma.$ We will denote by $\nu:S\to N_\sigma$ the future-directed unit normal section, and consider the decomposition for all vector field $X,Y$ tangent to $S$: 
\[
    \nabla_V W=\nabla^I_V W+II(V,W)\nu,
\]
where the symmetric $(2,0)$ tensor $II$ is called \textit{second fundamental form}. It will reveal useful to consider the $I-$symmetric $(1,1)$-tensor $B\in (TS)^*\otimes TS$ defined by $II(V,W)=I(B(V),W)$ which is called the \textit{shaper operator} of $\sigma.$ As in the Riemannian case, it turns out that $\sigma_*(B(v))=\nabla_v\nu.$\\  
The first and second fundamental form of an immersion $\sigma$ satisfy contrainst equations, known as $\textit{Gauss-Codazzi equations}$, More preciselly the Gauss equation consists of the identity: 
\[K_I=-1-\det_I II\] where $K_I$ is the curvature of $I$ and $\det_I II$ is $\det B$ by definition. On the other hand the Codazzi equation states that $\nabla^III$ is a totally symmetric $3-$form. We have: 
\begin{equation} \nabla 
\end{equation}

\subsection{Germs of spacelike immersions in Ads manifolds.}

