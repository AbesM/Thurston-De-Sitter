\chapter{Mess Work}

In his 1990 paper "Lorentz Spacetimes of Constant Curvature" \cite{Mess}, Geoffrey Mess offered a completely new approach to the study of spacetimes in 2+1-dimension by employing tools and techniques from low-dimensional geometry and topology. The aim of this chapter is to give a brief introduction to Mess ideas, with a special attention to AdS geometry. Our treatment will follow the setting introduced by Bonsante and Seppi in \cite{bonsanteseppi}. We want to introduce general \textit{Lorentianz} sets such as \textit{achronal subset, invisibile domain, domain of dependence} and how they relate to graph of circle homeomorphism (and its convex hull) in our specialized Anti-de Sitter setting. We will show that those graphs are \textit{proper achronal sets} in the projective model and always lift to achronal sets in the Poincaré model.

\section{Causality and Convexity properties}
We begin by giving some definitions:
\begin{definition}
    A subset $X$ of $\AS^{2,1}\cup \partial\AS^{2,1}$ is achronal (respectively acausal) if no pair of points in $X$ is connected by timelike (resp. causal) lines in $\AS^{2,1}$.
\end{definition}
Since acausality and achronality are conformally invariant notions, it will be often convenient to consider the metric $g_{\mathbb{S}^2}-dt^2$ on $\D\times\R$ introduced in \refeq{emispherical} which is conformal to the Poincaré model. We give now a first useful characterization of achronal and acausal sets:
\begin{lemma}
    A subset $X$ of $\AS^{2,1}\cup \partial\AS^{2,1}$ is achronal (respectively acausal) if and only if it is the graph of a function $f:D\to\R$ which is 1-Lipschitz, (resp. strictly 1-Lipschitz) with respect to the distance induced by the hemispherical metric $g_{\mathbb{S}^2}$.\\
    Where we have denoted $D=\pi_{\D}(X).$ 
\end{lemma}
\begin{proof}
     Let's assume $X$ is achronal. Now, since vertical lines in the Poincaré model are of timelike type, the restrictions of the projections $\pi_\D:\D\times\R\to\D$ to $X$ are injective. But then, $X$ can be interpreted as the graph of a function $f:D\to\R$. By imposing that $(x,f(x))$ and $(y,f(y))$ are not connected by a timelike curve we deduce that: 
     \begin{equation}\label{soloqua}
        |f(x)-f(y)|\leq d_{\mathbb{S}^2}(x,y)
     \end{equation}
     where $d_{\mathbb{S}^2}$ is the distance induced by the hemispherical metric. By the same reasonment we show that a 1-lipschit< graph over $\D$ is achronal. Moreover two points $(x,t)$ and $(y,s)$ are on the same lightlike geodesic if and only if $d_{\mathbb{S}^2}(x,y)=|t-s|$. Hence $X$ is acausal if and only if the inequality \refeq{soloqua} is strict. \
\end{proof}

Now a 1-Lipschitz function on a region $D\subset \D$ extends uniquely to the boundary of $D$. As a simple consequence of the previous lemma, we thus have: 

\begin{lemma}\label{achronalgraph}
    An achronal subset $X$ in $\AS^{2,1}$ is properly embedded if and only if it is a global graph over $\D$, and in this case it extends uniquely to the global graph of a 1-Lipschitz function over $\D\cup\partial\D$.
\end{lemma}

Because of \ref{achronalgraph} we will often refer to an \textit{achronal surface} as an achronal subset $X\subset\AS^{2,1}$ which is the graph of a 1-Lipschitz function defined in a domain in $\D.$ Before moving over to the study of properties we shall remark how achronality and acasuality are global conditions. 

\begin{definition} Given a surface $S$ and a Lorentzian manifold $(M,g),$ a $\mathcal{C}^1$ immersion $\sigma:S\to M$ is \textit{spacelike} if the pull-back metric $\sigma^*g$ is Riemannian. If $\sigma$ is an embedding, we refer to its image as a \textit{spacelike surface}.
\end{definition}

A spacelike surface $S$ is locally acasual, but there are examples of spacelike surfaces which are not achronal (hence a fortiori not acasual), a fact that highlights the global character of the achronality condition. On the other hand the following is true: 

\begin{lemma}
    Any properly embedded spacelike surface in $\AS^{2,1}$ is acasual. 
\end{lemma}
\begin{proof}
    By \ref{achronalgraph}, any properly embedded spacelike surfaces $S$ in $\AS^{2,1}$ disconnects the spaces in two regions $U,V$ whose common boundary is $S$, and we can assume that the outward pointing normal from $U$ (resp. $V$) is past-directed (resp. future directed). It then turns out that any future oriented causal path that meets $S$ passes from $V$ towards $U$. This implies that any causal path meets $S$ at most once. \todo{are we trying to model the fact that this is the present event?}
\end{proof}

We have remarked in Theorem \ref{ConformalMetric} that unparametrized lighlike geodesic only depend on the conformal class of the Lorentzian metric, hence we will just refer to lighlike in $\AS^{2,1},$ even when we are considering it endowed with the hemispherical metric of \refeq{emispherical}. 


\begin{lemma}\label{containedgeo}
    Let $S$ be a properly embedded achronal surface of $\AS^{2,1}\cup\partial\AS^{2,1}$ and assume that a lightlike geodesics segment $\gamma$ joins two points of $S$. Then $\gamma$ is entirely contained in $S.$
\end{lemma}
\begin{proof}
    We want to exploit Lemma \ref{achronalgraph}, let $f^S:\overline{\D}\to\R$ be the function that definins $S$, which is $1$-Lipschitz with respect to the emispherical metric. Not if our segment $\gamma$ joins $(x,f^S(x))$ to $(y,f^S(y))$, then (up to switching the role of $x$ and $y$) it holds: $f(y)^S=f^(x)+d_{\mathbb{S}^2}(x,y).$ More is true; if $\gamma$ consists of points of the form $(z,f^S(x)+d_{\mathbb{S}^2}(x,z))$, for $z$ that lie on the $g_{\mathbb{S}^2}$ geodesic connecting $x$ to $y$. For such a point $z$ on the geodesic, because of achronality of $S$, it holds: 
        \[
            f^S(z)-f^S(x)\leq d_{\mathbb{S}^2}(x,z)\;\;\text{and}\;\;f^S(y)-f^S(z)\leq d_{\mathbb{S}^2}(z,y)=d_{\mathbb{S}^2}(x,y)-d_{\mathbb{S}^2}(x,z).
        \]
        Second inequality implies that $f^S(z)\geq f^S(x)+d_{\mathbb{S^2}}(x,z),$ it follows that $f^S(z)=f(x)+d_{\mathbb{S}^2}(x,z)$ and hence $\gamma$ is entirely contained in $S.$
\end{proof}

\subsection{Invisible domain}
Invisible domain where introduced by Barbot in \textcolor{red}{inserire}. We give general definition and properties for a generic $X$ subset of $\AS^{2,1}\cup\partial  \AS^{2,1}$ and then we will focus on $X$ entirely contained in the boundary. 

\begin{definition}
    Given an achronal domain $X$ in $\AS^{2,1}\cup\partial\AS^{2,1}$, the \textit{invisibile domain} of $X$ is the subset $\Omega(X)$ of $\AS^{2,1}$ defined as the set of points which are not connected to $X$ by no causal path.   
\end{definition}

Any 1-Lipschitz function on a subset of a metric space admits a $1-$Lipschitz extension everywhere (Mc Shane's theorem). In our setting allows us show that any achronal set $X$ is a subset of a properly embedded achronal surface.\\
We consider two particular extensions $f_\pm^X:\D\cup\partial\D,$ which we will refer to as \textit{extremal} extensions:
\[
    f_-^X(y)=\sup\{f^X(x)-d_{\mathbb{S}^2}(x,y)\;|\;x\in\pi_\D(X)\} \;\;     f_+^X(y)=\inf\{f^X(x)-d_{\mathbb{S}^2}(x,y)\;|\;x\in\pi_\D(X)\}.
\]

\begin{lemma}\label{422}
    Let X be any closed achronal subset $X$ of $\AS^{2,1}\cup\partial\AS^{2,1}$, and let $S_\pm(X)$ be the graphs of the extremal extensions $f_\pm^X$. 
    \begin{itemize}
        \item The properly embedded surfaces $S_(X)$ and $S_+(X)$ are achronal with $S_-(X)\subset \overline{I^-(S_+(X))}$, and $\Omega(X)=I^+(S_(X))\cap I^-(S_+(X)).$
        \item Every achronal subset containing $X$ is contained in $S_(X)\cup\Omega(X)\cup S_+(X).$
        \item Every point of $S_\pm(X)$ is connected to $X$ by at least one lightlike geodesic segment, which is contained in $S_\pm(X).$ Finally $S_+(X)\cap S_(X)$ is the union of $X$ and all lightlike geodesic segment joining points of $X$.
    \end{itemize}
\end{lemma}
\begin{proof}
    BS 4.2.2
\end{proof}

\begin{observation}\label{423}
    Given a point $(y,t)$ the set of points $(x,s)$ satisfying $\lvert s-t \rvert<d_{\mathbb{S}^2}(x,y)$ coincides with the region of $\AS^{2,1}$ which is connected to $(y,t)$ by a spacelike geodesic for the Anti-de Sitter metric. It coincides also with the region of points connected to $(y,t)$ by a spacelike geodesic for the conformal emispherical metric. Now, since $f_-^X(y)\leq t\leq f_+^X(y)$ is equivalent to the condition that $\lvert s-t \rvert<d_{\mathbb{S}^2}(x,y)$ for all $(x,t)\in X,$ the region 

    \[
        S_+(X)\cup\Omega(X)\cup S_-(X)=\{(y,t)\;|\; f_-^X(y)\leq t\leq f_+^X(y)\}
    \]
    consist of all the points that are connected to any point of $X$ by spacelike or lightlike geodesics. Moreover $\Omega(X)$ consist of points connected to any point of $X$ by a spacelike geodesic. \textcolor{blue}{quindi l'invisibile domain sarebbe una sezione del tempo? Come si lega alla Cauchy Surface?} We observe that $\Omega(X)$ could be empty, for instance when $X$ is a global graph then $S_-(X)=S_+(X)=X$ and $\Omega(X)$ is empty.
\end{observation}

\begin{observation}
    Since any point of $S_\pm(X)$ is connected to $X$ by a lightlike geodesic, it follows from Lemma \ref{containedgeo} that the intersection of any properly embedded achronal surface containing $X$ with $S_\pm(X)$ is a union of lightlike geodesic segments with an endpoint in $X$. In particular any properly embedded acausal surface containing $X$ is contained in $\Omega(X).$
\end{observation}

\subsection{Achronal meridian in $\partial\AS^{2,1}$}
We will be interested in the study of invisibile domains of achronal meridians $\Gamma$ in the boundary of $AS^{2,1},$ that are graphs of 1-Lipschitz functions $f:\partial\D\to\R.$

\begin{lemma}
    Let $\Lambda$ be an achronal meridian in $\partial\AS^{2,1}$. Then either $\Lambda$ is the boundary of a lightlike plane, or $S_+(\Lambda)\cap S_-(\Lambda)=\Lambda.$ In the latter case there is an achronal properly embedded surface in $\Omega(\Lambda)$ whose boundary in $\partial\AS^{2,1}$ is $\Lambda$.  
\end{lemma}

\begin{proof}
    Let $f:\pi\D\to\R$ be the function whose graph is $\Lambda.$ \textcolor{blue}{servono le oscillazioni}
\end{proof}

Now, given a point $x \in \AS^{2,1}$, the Dirichlet domain of $x$ is the region $R_x$ containing $x$ and bounded by two spacelike plane ''dual" to $x$. Namely the planes that we denote (with a slight abuse of notation) $P_x^+$ and $P_x^-$, consisting of points at timelike distance $\pi/2$ in the future (resp. past) along timelike geodesic with initial point $x$.

\begin{proposition}\ref{432}
    Let $\Lambda$ be an achronal meridian in $\partial\AS^{2,1}$ different from the boundary of a lightlike plane. Then: 
        \begin{itemize}
            \item A point $x\in\AS^{2,1}$ lies in $\Omega(\Lambda)$ if and only if $\Lambda$ is contained in the interior of the Dirichlet region $R_x$. 
            \item For any $z\in\Lambda$, let $L_-(z)$ and $L_+(z)$ be the two lightlike planes such that $z$ is the past vertex of $L_+(z)$ and the future vertex of $L_-(z)$. Then 
            \[
                \Omega(\Lambda)=\bigcap_{z\in\Lambda}I^+(L_-(z))\cap I^{-}(L_+(z)).
            \] 
            \item The length of the intersection of $\omega(\Lambda)$ with any timelike geodesic of $\AS^{2,1}$ is at most $\pi.$ Moreover, there exist a timelike geodesic whose intersection with $\Omega(\Lambda)$ has length $\pi$ if and only if $\Lambda$ is the boundary at infinity of a spacelike plane. 
        \end{itemize}
\end{proposition}
\begin{proof}
    By Observation \ref{423} a point $x$ lies in $\Omega(\Lambda)$ if and only if it is connected to any point of $\Lambda$ by a spacelike geodesic. The region of points connected to $x$ by a spacelike geodesic has boundary the lightcone from $x,$ whose intersection with $\partial\AS^{2,1}$ coincides with $p_x^\pm\cap \partial\AS^{2,1}.$\\
    Moving on the second item we observe that the region bounded by $L_+(z)$ and $L_-(z)$ contains exaclty points connected to $z$ by a spacelike geodesic. Using the characterization of $\Omega(\Lambda)$ as above, we have the second statement.   \\
    Lastly, if a timelike geodesic $\gamma$ meets $\Omega(\Lambda)$ at a point $x$, then $\Omega(\Lambda)\subset R_x$, so that the lenght of $\gamma\cap\Omega(\Lambda)$ is smaller than the lenght of $\gamma\cap R_x$. But the latter is $\pi.$ Now assume the existance of a geodesic $\gamma$ such that the lenght of $\gamma\cap R_x$ equals $\pi.$ Up to applying an isometry of $\AS^{2,1}$ we may assume that $\gamma$ is vertical in the Poincaré model of $\AS^{2,1}$ amd the mid-point of $\gamma\cap\Omega(\Lambda)$ is $(0,0)$. Thus $(0,-\pi/2)$ and $(0,\pi/2)$ lie on $S_-(\Lambda)$ and $S_+(\Lambda)$ respectively. \\
    Now, again by \ref{423} points of $\Lambda$ are connected to $(0,-\pi/2)$ by a spacelike or lightlike geodesic, hence $s\leq 0$ for all $(\xi,s)\in\Lambda$ Analogously using the point $(0,\pi/2)$ we deduce that $s\geq 0$ for all $(\xi,s)\in\Lambda,$ so that $\Lambda=\partial\D\times\{0\}$.
\end{proof}

Arguing in similar fashion, we obtain that the invisibile domain of an achronal meridian which is not the boundary of a lightlike plane is always contained in a Dirichlet region. 

\begin{proposition}\label{433}
    Given an achronal meridian $\Lambda$ in $\partial\AS^{2,1}$ different from the boundary of a lightlike plane, the invisibile domain $\Omega(\Lambda)$ is contained in a Dirichlet region unless $\Lambda$ is the boundary of a spacelike plane. 
\end{proposition}
\begin{proof}
    
\end{proof}
    


\textcolor{red}{magari inserire figura dell'invisibile domain}

\begin{observation}
    When $\Lambda$ is the boundary of a spacelike plane $P$, then tehre are two points $x_-$ and $x_+$ such that $P=P_{x_-}^+=P_{x_+}^-.$ The previous argument shows that $\Omega(\Lambda)$ is the union of all timelike geodesics joininx $x_-$ to $x_+$ In this case $S_-(\Lambda$) is the union of the future directed lightlike geodesic rays emanating from $x_-$, whereas $S+(\Lambda)$ is the union of future directed lighlike rays endin at $x_+$.  
\end{observation}

\subsection{Domain of dependence}
We want to define and study properties of Cauchy surfaces and domains of dependence, a priori concept of Lorentzian geometry adapted to the geometry of $\AS^{2,1}$

\begin{definition}
    Given an achronal subset $X$ in a Lorentzian manifold $(M,g)$ the \textit{domain of dependence} of $X$ is the set: 
    \[
        \mathcal{D}(X)=\{p\in M |\;\text{every inextensibile causal curve through}\; p\; \text{meets} X \}.
    \]
    We say that $X$ is a \textit{Cauchy surface} of $M$ if $\mathcal{D}(X)=M$. A spacetime $M$ is said \textit{globally hyperbolic} if it admits a Cauchy surface.
\end{definition}

\begin{observation}
    Equivalently the domain of dependence  $D(X)$ of $X$ is the set of points $p\in\A^{2,1}$ such that the plane $p^*,$ the projective dual of $p$ as introduced in \textcolor{red}{define} is disjoint from $X.$ Domain of dependence are always contained in an affine chart and admit only light-like support planes. \textcolor{red}{Ancora non ho davvero definito i support planes a questo punto}
\end{observation}

The theory of globally hyperbolic spacetimes is a well-developed topic in Lorentzian geometry, we will just state the facts that we will need in the following. 

\begin{theorem}\label{442}
    Let M be a globally hyperbolic spacetime. Then:
    \begin{enumerate}
        \item Any two Cauchy surfaces in $M$ are diffeomorphic. 
        \item There exists a submersion $\tau:M\to\R$ whose fibers are Cauchy surfaces.
        \item M is diffeomorphic to $\Sigma\times\R$ where $\Sigma$ is any Cauchy surface in M.
    \end{enumerate}
\end{theorem}

\begin{observation}
    The spacetime $\AS^{2,1}$ is not globally hyperbolic. In fact if $X$ is achronal, it is contained in the graph of a 1-Lipschitz function \textcolor{red}{reference?} $f:(\D\cup\partial\D,g_{\mathbb{S}^2})\to\R.$ If $t_0>\text{sup}f$ and $\xi \in \partial\D$, then any lightlike ray with past end-point $(\xi,t_0)$ does not intersect $X.$ 
\end{observation}

\begin{observation}
    As causality notions are invariant under conformal change of metrics, we observe that causal paths in $\AS^{2,1}$ are the graphs of 1-Lipschitz functions from (intervals in) $\R$ to $\D$ with respect to the hemispherical metric in the image. Hence an inextesible causal curve in $\AS^{2,1}$ is either the graph of a global 1-Lipschitz function from $\R$, or it is defined on a proper interval and has endpoint(s) in $\partial\AS^{2,1}.$
\end{observation}

\begin{lemma}\label{445}
    Given an achronal meridian $\Lambda$ in $\partial\AS^{2,1}$ any Cauchy surface in $\Omega(\Lambda)$ is properly embedded with boundary at infinity $\Lambda$. 
\end{lemma}
\begin{proof}
    Let $S$ be a Cauchy surface in $\Omega(\Lambda)$. For every $x\in \D$, the vertical line through $x$ in the Poincaré model meets $\Omega(\Lambda)$, and it interesection with $\Omega(\Lambda)$ must meet $S$ by definition of Cauchy surface. This shows that $S$ is a graph over $\D$, hence properly embedded, and clearly $\partial S=\Lambda.$
\end{proof}

\begin{proposition}\ref{446}
    Let $\Lambda$ be an achronal meridian in $\partial\AS^{2,1}$ different from the boundary of a lightlike plane. Let S be a properly embedded achronal surface in $\Omega(\Lambda)$. Then $\mathcal{D}(S)=\Omega(\Lambda)$. In particular $\Omega(\Lambda)$ is a globally hyperbolic spacetime.
\end{proposition}
\begin{proof}
    Let $x$ be any point in $\Omega(\Lambda)$ and take any inextensible causal path through $x$. A priori its future endpoint might be either in $S_+(\Lambda)$ or in $\Lambda$, but by definition of $\Omega(\Lambda$), $x$ cannot be connected by any causal path to $\Lambda$, hence the latter case is excluded. The same argument shows that the past endpoint is in $S_-(\Lambda$.) Since the inextensibile causal path meets both $S_+(\Lambda)$ and $S_-(\Lambda)$, it must meet $S$ by Lemma \ref{445}, hence $\xi\in\mathcal{D}(S)$. \\
    Conversely, consider a $x$ that is not in $\Omega(\Lambda)$, the one can find a causal path joining $x$ to $\Lambda$, which is necessarily inextensibile. Hence $x$ is not in $\mathcal{D}(S)$.  
\end{proof}

\begin{observation}
    As a direct consequence of Theorem \ref{442} and Proposition \ref{446} we have that $\Lambda$ is the boundary of a spacelike surface in $\Omega(\Lambda)$, namely a Cauchy surface in $\Omega(\Lambda)$. By Lemma \ref{445}, the surface is properly embedded, hence the graph of a global 1-Lipschitz function. This shows that any proper achronal meridian $\Lambda$ is the boundary at infinity of a properly embedded spacelike surface, we have improved Lemma 431.
\end{observation}

The most remarkable property of domain of dependence of a properly embedded surface in $\AS^{2,1}$, and it is a direct consequence of Proposition \ref{446}, is that it only depends on the boundary at infinity. In detail:

\begin{corollary}
    If $S$ and $S^{\prime}$ are properly embedded spacelike surfaces in $\AS^{2,1}$ , then $\mathcal{D}(S)=\mathcal{D}(S^{\prime})$ if and only if $\partial S=\partial S^{\prime}$.
\end{corollary}

\subsection{Properly achronal sets in $\A^{2,1}$}
For our interest will be important to consider the model $\A^{2,1}$. As $\A^{2,1}$ contains closed timelike lines, it does not contain any achronal subset. However if $P$ is a spacelike plane in $\A^{2,1}$ then $\A^{2,1}\setminus P$ does not contain closed causal curves. Indeed it is simply connected, so it admits an isometric embedding into $\AS^{2,1}$, given by a section of the covering map $\AS^{2,1}\to\A^{2,1},$ and whose image is a Dirichlet region.
\begin{definition}
    A subset $X$ of $\A^{2,1}\cup\partial\A^{2,1}$ is a \textit{proper achronal subset} if there exists a spacelike plane $P$ such that $X$ is contained in $\A^{2,1}\cup\partial\A^{2,1}\setminus\overline{P}$ and is achronal as a subset of $\A^{2,1}\cup\partial\A^{2,1}\setminus\overline{P}.$  
\end{definition}

It follows from the definition that if $X$ is a proper achronal subset of $\overline{\A^{2,1}}$, then it admits a section to $\AS^{2,1}\cup\partial\AS^{2,1}$, and the image remains achronal after the lifting. Conversely, let $\tilde{X}$ be an achronal subset of $\AS^{2,1}$ different from a lightlike plane, then it is contained in a Dirichlet region as a consequence of Lemma \textcolor{red}{417}, and the fact that any achronal subset of $\AS^{2,1}$ is contained in properly embedded one. As Dirichlet regions are projected in $\AS^{2,1}$ to the complement of a spacelike plane, the image of $\tilde{X}$ to $A^{2.1}$ is a proper achronal subset. The following lemma will be key in our \textit{path} to prove the earthquake theorem:\\

\begin{lemma}
    Let $\varphi:\R\text{P}^1\to\R\text{P}^1$ be an orientation preserving homeomorphism. Then the graph of $\varphi$, say $\Gamma_\varphi\subset \R\text{P}^1\times\R\text{P}^1\simeq\partial\A^{2,1}$ is a proper achronal subset and any lift $\tilde{\Gamma_\varphi}$ is an achronal meridian in $\partial\AS^{2,1}.$  
\end{lemma}
\begin{proof}
    We start by proving that $\Lambda_\varphi$ is locally achronal. Consider $U$ and $V$ intervals around $x$ and $@\varphi(x)$ and $\theta_1$ and $\theta_2$ are positive coordinates on $U$ and $V$ respectively, then timelike curves $\gamma(t)=(\gamma_1(t),\gamma_2(t))$ in $U\times V$ are characterized by the porpery that $\theta_1^{\prime} (t)\theta_2^{\prime} (t)<0,$ where $\theta_i(t)=\theta_i(\gamma_i(t)).$ \\
    In particular points on $\Lambda_\varphi\cap U\times V$ are not related by a timelike curve contained in $U\times V$, by the assumption that $\varphi$ is orientation-preserving. \\
    Let us prove the existance of a spacelike plane $P$ such that $\overline{P}\cap \Lambda_\varphi=\emptyset.$ Let us consider the identification $\S=\R\cup\{\infty\}$, and take $\varphi_0\in\PSL(2,\R)$ so that $\varphi_0^{-1}\varphi(0)=1, \varphi_0^{-1}\varphi(1)=\infty$ and $\varphi_0^{-1}\varphi(\infty)=0$ It follows that $\varphi_0^{-1}\varphi$ sends the intervals $(\infty,0), (0,1)$ and $(1,\infty$) respectively to $(0,1),(1,\infty),(\infty,0)$. Thus $\varphi_0^{-1}\varphi$ has no fixed points, that is, the graph of $\varphi$ does not intersect the graph of $\varphi_0$, which is the asymptotic boundary of the spacelike plane $P_{\varphi_0}.$\\
    Let us consider now the lift $\widetilde{\Lambda}_\varphi$ to the boundary of $\AS^{2,1}$. As $\Lambda_\varphi$ is contained in a simply connected region of $\overline{\A^{2,1}}$, its lift is a closed locally achronal curve contained in $\partial\AS^{2,1}$. In particular the projection $\widetilde{\Lambda}_\varphi\to\partial\D$ is locally injective. As $\widetilde{\Lambda}_\varphi$ is compact, the map is a covering. On the other hand, since $\Lambda_\varphi$ is homotopic to the boundary of a plane in $\partial\A^{2.1}$, it turns out that $\widetilde{\Lambda}_\varphi$ is homotopic to $\partial\D$ in $\partial\AS^{2,1}$, so that the projections $\widetilde{\Lambda}_\varphi\to\partial\D$ is bijective. It follows that $\widetilde{\Lambda}_\varphi$ is achronal.  
\end{proof}

With what we have shown until now for achronal sets in $\AS^{2,1}$ can be rephrased for proper achronal sets of $\A^{2,1}$. For example, any proper achronal set $X$ can be extended to a properly embedded proper achronal surface and there are two extremal extensions, just as in Lemma \ref{422}\\
We would like to focus now on proper achronal meridians in $\partial\A^{2,1}$. They lift to achronal meridians in $\partial\AS^{2,1}$ different form the boundary of lightlike planes. Indeed the boundary of a lightlike plane is not contained in a Dirichlet region. Conversely any achronal meridian on $\partial\AS^{2,1}$ different from the boundary of a lightlike plane projects to an achronal meridian of $\A^{2,1}$.

\begin{proposition}
    Let $\Lambda$ be a proper achronal meridian in $\partial\A^{2,1}$ and denote by $\widetilde{\Lambda}$ any lift to the universal covering. Then the universal covering map of $\A^{2,1}$ maps $\Omega(\widetilde{\Lambda})$ injectively to the domain: 
    \[
        \Omega(\Lambda)\coloneqq \{x\in\A^[2,1]\;|\;P_x\cap\Lambda=\emptyset\}.
    \]
\end{proposition}
    
\begin{proof}

\end{proof}

When $\Lambda$ is the graph of an orientation-preserving homomorphism $\varphi:\S\to\S$, there is a fairly simple characterization of $\Omega(\Lambda)$ exploiting the identificiation $\A^{2,1}\simeq \PSL$.


\begin{corollary}
    Let $\varphi:\S\to\S$ be an orientation-preserving homeomorphism. Then $x\in\A^{2,1}$ lies in $\Omega(\Lambda_\varphi)$ if and only if $x\circ\varphi$ has no fixed point as a homeomorphism of $\S$.
\end{corollary}
\begin{proof}
    The dual plane of $x,$ viewed as an element of $\PSL$, meets $\partial\A^{2,1}$ along the graph of $x^{-1}$, namely $\Lambda_{x^{-1}}$.\\
    With this remark in hand, we have that $x\in\Omega(\Lambda_\varphi)$ if and only if $\Lambda_{x^{-1}}\cap @\Lambda_\varphi=\emptyset$, that is, if and only if $x^{-1}\circ\varphi$ has no fixed point on $\S$.

\end{proof}

\begin{proposition}
    Let $\sigma: S\to\A^{2,1}$ be a proper spacelike immersion. Then 
    \begin{itemize}
        \item $\sigma$ is a proper embedding.
        \item $\sigma$ lifts to a proper embedding $\tilde{\sigma}:S\to\AS^{2,1}$.
        \item The boundary at infinity of $\sigma(S)$ is a proper achronal meridian $\Lambda$ in $\partial\A^{2,1}$.
        \item $\mathcal{D}(\sigma(S))=\Omega(\Lambda).$
    \end{itemize}
\end{proposition}

\begin{proof}
\end{proof}

\begin{observation}
    In the proof of the last proposition, once we proved that $\hat{S}$ is homeomorphic to $\R^2,$ we could have inferred immediately that $\hat{S}=S$ as $\Z/2\Z$ cannot act freely on $\R^2$ by diffeomorphism. 
\end{observation}

We therefore have an analogue version of \text{red}{citare lemma analogo} in $\A^{2,1}$.

\begin{corollary}
    If $S$ and $S^{\prime}$ are properly embedded spacelike surfaces in $\A^{2,1}$, then $\mathcal{D}(S)=\mathcal{D}(S^{\prime})$ if and only if $\partial S=\partial S^{\prime}$. 
\end{corollary}

\subsection{Convexity properties.}

Let $\Lambda$ be a proper achronal meridian in $\partial\A^{2,1}$. We would like to investigate convexity properties of $\Omega(\Lambda)$. We briefly recall that given $X \subset \R\text{P}^n$ is convex if it contained in an affine chart, and it is convex in the affine chart. The notion does not depend on the affine chart containing $X$. We would say that it is a proper convex set if it compactly contained in an affine chart. 

\begin{proposition}\label{461}
    Given a proper achronal meridian $\Lambda$ in $\partial\A^{2,1}$, the invisible domain $\Omega(\Lambda)$ is convex. If $\Lambda$ is different from the boundary of a spacelike plane then $\Omega(\Lambda)$ is a proper convex set.
\end{proposition}

\begin{proof}
    By Proposition \ref{433} there exists a spacelike plane $P$ such that $\Omega(\Lambda)$ is contained in the affine chart $V$ of $\R\text{P}^3$ obtained by removing the projective plane containing $P$. The domain $\A^{2,1}\cap V=\A^{2,1}\setminus P$ is isometric to a Dirichlet region $R$ of $\AS^{2,1}$, by an isometry that sends $\Lambda$ to a lifting $\widetilde{\Lambda}$ and $\Omega(\Lambda)$ to $\Omega(\widetilde{\Lambda})$. By te second point of Proposition \ref{432} we have 
    \[
        \Omega(\widetilde{\Lambda})=\bigcap_{\widetilde{z}\in\widetilde{\Lambda}}I^+(L_-(\widetilde{z}))\cap I^-(L_+(\widetilde{z})).
    \]

    Now if $\widetilde{z}$ projects to $z$, then the images of $L_-(\widetilde{z})$ and $L_+(\widetilde{z})$ in $V$ are the two components of $L(z)\cap \A^{2,1}$, where $L(z)$ is the affine tangent plane of $\partial\A^{2,1}\cap V$ at $z$. It turns out that te image of the region $I^+(L_-(\widetilde{z}))\cap I^-(L_+(\widetilde{z}))$ is the interesection of $\A^{2,1}$ with the half-space $U(z)$ bounded by $L(z)$ and whose closure contains $\Lambda$. This shows: 
\[
    \Omega(\Lambda)=\A^{2,1}\bigcap_{z\in\Lambda}U(z).
\] 

Actually we claim that:
\[
    \Omega(\Lambda)=\bigcap_{z\in\Lambda}U(z)\subset \A^{2,1},
\]
and this will conclude. As $\bigcap_{z\in\Lambda}U(z)$ is connected and meets $@A^{2,1}$, to show that is is contained in $@A^{2,1}$ it sufficies to show that it does not meet the boundary. For any $w\in\partial\A^{2,1}$ let us consider the leaf of the left ruling through $w$, which intersects $\Lambda$ at a point $z$. It turns out that $L(z)$ contains the leaf of the left ruling through $z$, hence $w\notin U(z)$. \\
Now assume that $\Lambda$ is not the boundary of a spacelike plane. Then by Proposition \ref{433} on the universal covering the compact set $\Omega(\widetilde{\Lambda})\cup S_+(\widetilde{\Lambda})\cup S_-(\widetilde{\Lambda})$ is contained in a Dirichlet domain, so its image is a compact set contained in an affine chart.
\end{proof}

As a consequence we have that $\Lambda$ is contained in an affine chart whose complement in $\R\text{P}^3$ is a projective plane containing a spacelike plane of $\A^{2,1}$ Hence it makes sense to give the following definition:

\begin{definition}
    Given a proper achronal meridian $\Lambda$ in $\partial\A^{2,1}$, we define $\mathcal{C}(\Lambda)$ to be the convex hull of $\Lambda$, which can be taken in an affine chart containing $\Lambda$
\end{definition}

What we have implicitly proved is that given $\Lambda$ an achronal meridian in $\partial\A^{2,1}$, then $\mathcal{C}(\Lambda)$ is contained in $\A^{2,1}$, which is not immediately obvious as $\A^{2,1}$ is not convex in $\R\text{P}^3$. \\

\begin{observation} Since $\Omega(\Lambda)$ is convex, $\mathcal{C}(\Lambda)$ is contained in $\Omega(\Lambda)$. Moreover, if $K$ is any convex set contained in $\overline{\A^{2,1}}$ and containing $\Lambda$, then $\mathcal{C}(\Lambda)\subset K\subset \overline{\Omega(\Lambda)}.$\\
    To see this, let $V$ be an affine chart such that $\Lambda\subset V$ is obtained removing a spacelike projective plane. Now, if $z\in \Lambda$ then for any $x\in\A^{2,1}\cap V$ the segment connecting $z$ and $x$ in $V$ is contained in $\A^{2,1}$ if and only if $x\in U(z)$, the half-space containing $\Lambda$ and bounded by the tangent space of $\Lambda$ at $z,$ as in the proof of Proposition \ref{461}. \\
    It follows, from the characterization of $\Omega(\Lambda)$ as the intersection of the $U(z)$ given in Proposition \ref{461}, that if $x$ is not in $\overline{\Omega(\Lambda)}$ it cannot be in $K$. Hence $\overline{\Omega(\Lambda)}$ is the biggest convex set of $\A^{2,1}$ containing $\Lambda$.
\end{observation}

Suppose now that $\Lambda$ is not the boundary of a spacelike plane. Now the topological frontiers in $\R\text{P}^3$ of $\Omega(\Lambda)$ and of $\mathcal{C}(\Lambda)$ are Lipschitz surfaces homeomorphic to a sphere. This sphere is disconneced by $\Lambda$ in two regions, homeomorphic to disks, which form the boundary of $\Omega(\Lambda)$ and of $\mathcal{C}(\Lambda)$ in $\A^{2,1}$. For $\Omega(\Lambda)$ those components are the image of $S_\pm(\widetilde{\Lambda})$ and will be denoted by $S_\pm(\Lambda$). \\
For $\mathcal{C}(\Lambda),$ let be $\mathcal{C}(\widetilde{\Lambda})$ a lifting, which is necessarily contained in a Dirichlet region, say $R$. Let $P$ be a support plane for $\mathcal{C}(\Lambda)$, which is necessarily either spacelike or lightlike, and let $\widetilde{P}$ be its lift which touches $\mathcal{C}(\widetilde{\Lambda}).$ Hence either $\widetilde{\Lambda}$ is in $I^+(\widetilde{P})\cup\widetilde{P}$ or in $I^-(\widetilde{P})\cup\widetilde{P}$. This permits to distinguish the componets of $\partial\mathcal{C}(\Lambda)\setminus\Lambda:$ the \textit{past boundary component} $\partial\mathcal{C}_-(\Lambda)$ has the property that $\widetilde{\Lambda}$ is contained in $I^+(\widetilde{P})\cup \widetilde{P}$ for all support planes which touch $\partial_-\mathcal{C}(\Lambda)$, and analogously for the \textit{future boundary component} $\partial_+\mathcal{C}(\Lambda)$.\\
We want to show a kind of duality between boundary components $\partial_\pm\mathcal{C}(\Lambda)$ and $S_\pm(\Lambda)$. 

\begin{proposition}
Let $\Lambda$ be a proper achronal meridian suality condition 
\end{proposition}
\begin{proof}
    
\end{proof}

\begin{observation}\label{465}
    It may happen that a boundary component of $\mathcal{C}(\Lambda)$ meets the boundary of $\Omega(\Lambda)$. This exactly happens when the curve $\Lambda$ contains a \textit{sawtooth}, namely two consecutive lightlike segments in $\A^{2,1},$ one past directed and the other future-directed. In this case the lighltike plane $L(z)$ tangent to $\partial\A^{2,1}$ at the vertex $z$ of the sawtooth contains a lightlike traingle contained in $L(z)$. This is however not contained in $\Omega(\Lambda)$. If the curve $\Lambda$ does nto contain any sawtooth, then $@\mathcal{\Lambda}\setminus\Lambda$ is entirely contained in $\Omega(\Lambda)$.\\
    The fundamental example is given in \textcolor{red}{inserire figura 8}, where the yellow refion represents at the same time the convex hull of the proper achronal meridian $\Lambda$ in $\partial\A^{2,1}$ composed of four lightlike segments, two past-directed and two future-directed, and the closure of $\Omega(\Lambda).$ \textcolor{red}{Vedi anche remark 8.1.3 e la figura 14.} 
\end{observation}

\begin{proposition}\label{466}
The past and future boundary components of $@\mathcal{C}(\Lambda)$ are achronal surfaces.
\end{proposition}
\begin{proof}
    
\end{proof}

\begin{observation}\label{467}
The past and future boundary components of $@\mathcal{C}(\Lambda)$ are not smooth, but only Lipschitz surfaces. Indeed the complement of $\Lambda$ and of the ightlike triangles (as described in \ref{465}) is locally connected by acausal Lipschitz arcs, and one can define a pseudo-distance, that in fact turns out to be a distance and makes $@\mathcal{C}(\Lambda)$ locally isometric to the hyperbolic plane. \\
The situation is very similar to the counterpart in hyperbolic three-space. The intersection of a spacelike support plane with $\mathcal{C}(\Lambda)$ is either a geodesic or a straight convex subset of $\H^2$, i.e. a subset bounded by geodesics. Thus $\partial\mathcal{C}(\Lambda)\setminus\Lambda$ is intrinsically a hyperbolic surface pleated along a measured geodesic lamination. A remarkable difference with respect to the hyperbolic case \textcolor{red}{si ma quale} is that in general those surfaces may be not complete, but they are always isometric to straight convex subset of $\H^2$. \textcolor{red}{Per espandere su questo consigliano Multi-black holes and earthquakes on Riemann surfaces with boundaries.}

\end{observation}

\section{Support planes.}
We still need to borrow some notions and notations from convex analysis. Given a convex body $K$ in affine space of dimension three, a \textit{support plane} of $K$ is an affine plane $Q$ such that $K$ is contained in a closed half-space bounded by $Q$, and $\partial K\cap Q\neq\emptyset.$ If $p$ is in the intersection of $K$ with $Q$ we will say that $Q$ is a \textit{support plane} at $p$. As a consequence of the Hahn-Banach theorem there exixst a support plane at every point $p\in\partial K.$ \\
We will adopt the terminology to the Anti-de Sitter setting, given a convex hull $\mathcal{C}(f)$ in $\A^{2,1}$, we say that a totally geodesic plane $P$ is a support plane of $\mathcal{C}(f)$ (at $p\in\partial\mathcal{C}(f))$ is $p\in\mathcal{C}(f)\cap \overline{P}\subset\overline{\A^{2,1}}$ and, in an affine chart containing $\text{graph}(f)$, $\mathcal{C}(f)$ lies in a closed half-space bounded by the affine plane that contains $P$. Even this definition does not depend on the choice of the affine chart as long this one contains $\text{graph}(f)$.\\ 

\textcolor{blue}{He goes for a remark that we might not need}\\

Recall \textcolor{orange}{from where?} that if $X$ is a set, $\mathcal{C}(X)$ its convex hull and $Q$ an affine support plane for $\mathcal{C}(X)$, then $Q\cap\mathcal{C}(X)=\mathcal{C}(Q\cap X).$ Applying this identity in our setting, we get that for any totally geodesic support plane $P$: 

\begin{equation}\label{suplane}
    P\cap\mathcal{C}(f)=\mathcal{C}(\partial P\cap\text{graph}(f)).
\end{equation}
    
We can characterize support planes by considering intersection with the convex hull as follows: 

\begin{proposition}\label{supportinho}
    Let $f:\S\to\S$ be an orientation-preserving homeomorphism, and let $P$ be a support plane of $\mathcal{C}(f)$ at a point $p\in\partial\mathcal{C}(f).$ Then:
    \begin{itemize}
        \item If $p\in\A^{2,1}$, then $P$ is a spacelike plane. 
        \item If $p\in\partial\A^{2,1}$ then $P$ is either spacelike or lightlike.
    \end{itemize}
\end{proposition}
\begin{proof}
    The key observation is that if $P$ is a spacelike plane, then $\partial P$ and $\text{graph}(f)=\mathcal{C}(f)\cap\partial\A^{2,1}$ do not intersect transversely. From \textcolor{blue}{add reference}, if $P$ is timelike then $\partial P$ is the graph of an orientating-reversing homeomorphism of $\S$, hence it intersects $\text{graph}(f)$ transversally. From Lemma \textcolor{blue}{add reference}, if $P$ is lightlike, then its boundary is the union of the two circles: $\{x\}\times\S$ and $\S\times{y}$. So if $p\in\partial P\cap\text{graph}(f)$ and $p$ is not the point $p_0=(x,y),$ then $\partial P$ and $\text{graph}(f)$ intersect transversally. Hence the only case where $P$ can be a lightlike support plane is when it intersects $\text{graph}(f)$ only at $p_0.$ We are left with the task to show that $P\cap\mathcal{C}(f)=\{p_0\}$ and it does not contain any point of $\A^{2,1}.$ By contradiction suppose the existance of a $q\in P\cap\mathcal{C}(f)$ with $q$ different from $p_0,$ then by \refeq{suplane} $\partial P\cap\text{graph}(f)$ would also contain another point different from $p_0$, a contradiction.
\end{proof}

Now, given a spacelike support plane $P$ of $\mathcal{C}(f)$ at a point $p$, we say that $P$ is a \textit{future} (resp. \textit{past}) \todo{Forse la definizione del survey è migliore?} support plane if in a small simply connected neighbourhood of
$p\in\overline{\A^{2,1}}, \mathcal{C}(f)$ is contained in the closure of the connected components of $U\setminus P$ which is the past (resp. future) of $P$. This means that there exists a future-oriented (resp past-oriented) timelike curves leaving $\mathcal{C}(f)\cap U$ and reaching $P\cap U$. \\
We observe that $\mathcal{C}(f)$ cannot have both a future and past support plane at $p$ unless $\CF$ has empty interior, a situation that happens exactly when $f$ is an element of $\PSL$, the example treated previously.\\
We can now state the following related to convergence of support planes: 
\begin{lemma}\label{49}
    Let $f:\S\to\S$ be an orientation-preserving homeomorphism which is not in $\PSL$, $p_n$ a sequence of points in $\partial\CF$, and $P_{\gamma_n}$ a sequence of future (resp. past) spacelike supports planes at $p_n$, for $\gamma_n\in\PSL$. Up to extracting a subsequence, we can assume $p_n\to p$ and $P_{\gamma_n}\to P.$ Then: 
    \begin{itemize}
        \item If $p\in\A^{2,1}$, then $P=P_\gamma$ is a future (resp. past) support plane of $\CF$, for $\gamma_n\to\gamma\in\PSL$.
        \item If $p\in\partial\A^{2,1}$, then either $P$ is a lighlike plane whose boundary is the union of two circle meeting at $p$ or the previous point holds.        
    \end{itemize}
\end{lemma}
\begin{proof}
    Diaf-Seppi
\end{proof}
\begin{corollary}\label{nametag}
    Let $f:\S\to\S$ be an orientation-preserving homeomorphism which is not in $\PSL$. Then $\partial\CF$ is the disjoint union of $\text{graph}(f)=\CF\cap\partial\A^{2,1}$ and of two topological discs, of which one admits future support plane, and the other only admits past support plane. 
\end{corollary}
\begin{proof}
    Diaf-Seppi
\end{proof}

According to Corollary \ref{nametag} we will call the connected component of $\partial\CF\setminus\text{graph}(f)$ that only admits future support planes the \textit{future boundary component}, and denote it by $\partial_+\CF$. Similarly, we will call the connected component of $\partial\CF\setminus\text{graph}(f)$ that only admits past support planes the \textit{past boundary component}, and denote it by $\partial_-\CF$.\\

\section{Globally Hyperbolic three-manifolds}
We want now to focus our attention on maximal globally hyperbolic (MGH) Anti-de Sitter spacetimes containing a compact Cauchy surfaces of genus $r$ (we will with a slight abuse of notation say that the spacetime has genus $r$.) We will show that there are no such manifolds when $r=0$. Furthermore, we will then give a rapid overview over the case $r=1$ and the focus on the most interesting case, $r\geq 2$, that will lead to a complete classification.

\subsection{General facts.} We will now state some general that will be useful in our classification.

\begin{lemma}\label{properembedding}
    Let $\sigma:S\to\A^{2,1}$ be a spacelike immersion. If $\sigma^*(g_{\A^{2,1}})$ is a complete Riemannian metric, then $\sigma$ is a proper embedding and $S$ is diffeomorphic to $\R^2$
\end{lemma}
\begin{proof}
    
\end{proof}

As an immediate consequence there cannot be any globally hyperbolic spacetime with genus $0$. In fact, suppose such a spacetime exists and denote by $\Sigma$ the Cauchy surfaces diffeomorphic to $\mathbb{S}^2$, the developing map restricted to $\Sigma$ would be a spacelike immersion, and the pull-back metric would be complete by compactness. But this contradicts \ref{properembedding}. Hence: 

\begin{corollary}
    There exists no globally hyperbolic Anti-de Sitter spacetime of genus 0. 
\end{corollary}

The following is a fundamental result on the structure of globally hyperbolic AdS spacetimes.

\begin{proposition}\label{holorep}
    Let $M$ be a globally hyperbolic Anti-de Sitter spacetimes of genus $r\geq 1$. Then: 
    \begin{enumerate}
        \item The developing map $\text{dev}:\tilde{M}  \to  \A^{2,1}$ is injective. 
        \item If $\Sigma$ is a Cauchy surface of $M,$ then the image of dev is contained in $\Omega(\Lambda)$, where $\Lambda$ is boundary at infinity of $\text{dev}(\tilde{\Sigma}).$
        \item If $\rho:\pi_1(M)\to\text{Isom}(\A^{2,1,})$ is the holonomy representation, $\rho(\pi_1(M))$ acts freely and properly discontinuosly on $\Omega(\Lambda)$, and $\Omega(\Lambda)$ is a globally hyperbolic spacetime containing $M$. 
    \end{enumerate}
\end{proposition}

\begin{proof}
    
\end{proof}

A remarkable difference between Lorentianz and Riemannian geometry is the that in Lorentianz geometry completeness is a very strong assumption, \textcolor{red}{reference} and in fact interesting classification results are obtained removing such condition. However, it is necessary to impose some maximality condition to compensate for non-completeness. Among several  approaches, one of the most common is the classification of a maximal globally hyperbolic spacetimes. Following \cite{bonsanteseppi} we restrict to a less general setting, but one could give similar definitions in the lager class of Einstein spacetimes.

\begin{definition}
    A globally hyperbolic Anti-de Sitter manifold $(M,g)$ is \textit{maximal} if any isometric embedding of $(M,g)$ into a globally hyperbolic Anti-de Sitter manifold $(M^{\prime},g^{\prime})$ which sends a Cauchy surface of $M$ to a Cauchy surface of $M^{\prime}$ is surjective.
\end{definition}

In eyesight of this definition and as a direct consequence of propostion \ref{holorep} we have:

\begin{corollary}
An Anti-de Sitter globally hyperbolic spacetime $M$ is maximal is and only if $\tilde{M}$ is isometric to the invisibile domain of a proper achronal meridian in $\partial\A^{2,1}.$ \textcolor{red}{importante da capire credo}
\end{corollary}


%genere uno 

\subsection{Examples of genus $r\geq 2$}
Let's $\Sigma_r$ be an oriented surface of genus $r\geq 2.$ We recall the definiton of Fuchsian representation. 

\begin{definition}
    A representaion $\rho:\pi_1(S)\to \text{PSL}(2,\R)$ is called positive \textit{Fuchsian} if there is a $\rho$-equivariant orientation preserving homeomorphism $\delta:\widetilde{\Sigma}_r\to\H^2$.
\end{definition}

The definition is invariant under conjugation in $\text{PSL}(2,\R)\simeq\text{Isom}_0(\H^2),$ but not under conjugation in $\text{Isom}(\H^2).$ It is a result of Goldman \textcolor{red}{Reference}, that a representation $\rho$ is positive Fuchsian if and only if the associated flat $\R\text{P}^1$ bundle $E_\rho$ constructed as the quotient of $\widetilde{\Sigma}_r\times\R\text{P}^1$ by the diagonal action of $\pi(S)$ (via deck transformation and the representation) has Euler class $2-2r.$ An equivalent condition is the existence of an orientation-preserving fiber bundle isomorphism between $E_\rho$ and the unit tangent bundle of $\Sigma_r.$ \\   

Another useful tool for the classification of genus $r\geq 2$ is the following fact in Teichm\"uller theory \textcolor{red}{reference}:

\begin{lemma}
    Given two positive Fuchsian representation $\rho_l,\rho_r:\pi_1(\Sigma_r)\to\text{PSL}(2,\R)$, any $(\rho_l,\rho_r)$ equivariant orientation-preserving homeomorphism of $\H^2,$ extends continuously to an orientation-preserving homeomorphism of $\H^2\cup\R\text{P}^1.$ Moreover, its extension $\varphi:\R\text{P}^1\to\R\text{P}^1$ is the unique $(\rho_l,\rho_r)-$equivariant orientation homeomorphism of $\R\text{P}^1$.     
\end{lemma}

Here by ($\rho_l,\rho_r$)-equivariant we mean that $\varphi\circ\rho_l=\rho_r\circ\varphi.$\\
Now let $\rho_l,\rho_r$ be two positive Fuchsian representation of $\Sigma_r.$ We will be interested in the representation:

\[
    \rho=(\rho_l,\rho_r):\pi_1(S)\to\text{Isom}_0(\A^{2,1})\simeq \text{PSL}(2,\R)\times\text{PSL}(2,\R).
\]

\begin{definition}
    Given a pair of positive Fuchsian representation $\rho_l,\rho_r:\pi_1(\Sigma_r)\to\text{PSL}(2,\R)$ we define $\Lambda(\rho)$ to be the graph in $\R\text{P}^1\times\R\text{P}^1$ of the unique $(\rho_l,\rho_r)-$equivariant orientation-preserving homeomorphism of $\R\text{P}^1,$ and $\Omega_\rho \coloneqq \Omega(\Lambda(\rho))$ its invisibile domain in $\A^{2,1}$
\end{definition}

Using the above construction, we can build examples of MGH spacetimes having holonomy any $\rho=(\rho_l,\rho_r)$ of this form. 

\begin{proposition}
    The domain $\Omega_\rho$ is invariant under the isometric action of $\pi_1(\Sigma_r)$ on $\A^{2,1}$ induced by $\rho.$ Moreover $\pi_1(\Sigma_r)$ acts freely and properly discontinuosly on $\Sigma_\rho$ and the quotient is a MGH spacetime of genus $r$ and holonomy $\rho.$ 
\end{proposition}
\begin{proof}
    \textcolor{red}{da adesso in poi bisogna capire per forza.}
\end{proof}

\subsection{Classification of genus $r\geq 2$.} We will now show that the examples of Proposition \textcolor{red}{citare} are \textit{all} the MGH spacetimes of genus $r$. 

\begin{lemma}
    Let $\rho:(\rho_l,\rho_r)$ be a pair of Fuchsian representations, and $\varphi:\R\text{P}^1\to\R\text{P}^1$ be the unique $(\rho_l,\rho_r)-$equivariant orientation-preserving homeomorphism of $\R\text{P}^1.$ Then $\Lambda(\rho)$ is the unique propre achronal meridian in $\partial\A^{2,1}$ invariant under the action of $\pi_1(\Sigma_r)$ induced by $\rho.$
\end{lemma}

\begin{corollary}
    For any pair $\rho=(\rho_l,\rho_r)$ of positive Fuchsian representations of $\pi_1(\Sigma_r)$, $M_\rho$ is the unique MGH spacetime with holonomy $\rho$.
\end{corollary}

We are only left with one last step for the classification result, we want to show that the left and right holonomies are necessarily positive Fuchsian. 

\begin{proposition}
    Let $M$ be an oriented, time-oriented, globally hyperbolic spacetime of genus $r\geq 2$ and let us endow a Cauchy surface $\Sigma$ with the orientation induced by the future normal vector. Then the left and right components of the holonomy $\rho=(\rho_l,\rho_r):\pi_1(\Sigma)\to\text{PSL}(2,\R)\times\text{PSL}(2,\R)$ are positive Fuchsian representation. 
\end{proposition}

\begin{observation}
We refer to the honolomy $\rho$ with respect to an orientation-preserving developing map. Therefore $\rho$ is well-defined up to conjugacy in $\text{PSL}(2,\R)\times\text{PSL}(2,\R)$
\end{observation}
\begin{proof}
    
\end{proof}

We can now state the classification result. We denote the \textit{deformation space} of MGH spacetimes of genus $r$ by:

\[
    \mathcal{MGH}(\Sigma_r)=\{g\;\text{MGH AdS metric on}\;\Sigma_r\times\R\}/\text{Diff}_0(\Sigma_r\times\R)
\]
where the group of diffeomorphism isotopic to the identity is acting by pill-back. The holonomy map takes value in the space of representation of $\pi_1(\Sigma_r)$ into $\PSL\times\PSL$ and is well-defined on the quotient $ \mathcal{MGH}(\Sigma_r).$
As a consequence of Proposition [...], the left and right components of the holonomy of elements of $ \mathcal{MGH}(\Sigma_r)$ are positive Fuchsian representations. The space of these rrepresenations up to conjugacy is idetified with the Teichm\"uller space of $\Sigma_r$ by the aforementioned work of Goldman []:
\[
    \mathcal{T}(\Sigma_r)\simeq\{\rho:\pi_1(\Sigma_r)\to\PSL\;\text{positive Fuchsian representation}\}/\PSL
\]

Therefore the holonomy map can be considered as a map from $ \mathcal{MGH}(\Sigma_r)$ with values in $\mathcal{T}(\Sigma_r)\times\mathcal{T}(\Sigma_r).$ Restating the classification in the original set of Mess \cite{Mess}: 
\begin{theorem}
    The holonomy map $\rho:\mathcal{MGH}(\Sigma_r)\to\mathcal{T}(\Sigma_r)\times\mathcal{T}(\Sigma_r)$ is a homeomorphism.
\end{theorem}
\section{Gauss map of spacelike surfaces}
    We will now introduce the \textit{Gauss Map} associated to a spacelike surface in Anti de Sitter space, this tool will be  useful in the study of the relation of Anti-de Sitter geometry and Teichm\"uller theory and hyperbolic surfaces.

    We will start by fixing notation and recalling some general facts about immersed spacelike surfaces in Anti de Sitter space. We will assume that all our immersed surfaces are of class $C^1$. \\
    Given a regular immersion $\sigma: S\to\A^{2,1}$ we will say that it is \textit{spacelike} if the pull-back metric $I\coloneqq\sigma^*g_{\A^{2,1}}$ is Riemannian, we call $I$ firs fundamental form of $\sigma$. 

The tangent bundle $TS$ is naturally identified to a subbundle of $T\A^{2,1}$ by means of $d\sigma$. The normal bundle $N_\sigma$ is then defined as the $g_{\A}$-orthogonal complement of $TS$ in $T\A^{2,1}$, The restriction of $g_{\A}$ to $N_\sigma$ is negative definite. Using the $g_\A$ induced decomposition: 
\[
\sigma^*T\A^{2,1}=TS\oplus N_\sigma,    
\]

the pull-back of the ambient Levi-Civita connection $\nabla$, restricted to sections tangent to $S$ splits as the sum of the Levi-Civita connection $\nabla^I$ of the first fundamental form $I$, and a symmetric 2-form with value in $N_\sigma$. As a consequence of time-orientability of $\A^{2,1}$, the normal bundle admits a natural trivialization, which is the same as a choice of a continuous unit normal vector field for $\sigma.$ We will denote by $\nu:S\to N_\sigma$ the future-directed unit normal section, and consider the decomposition for all vector field $X,Y$ tangent to $S$: 
\[
    \nabla_V W=\nabla^I_V W+II(V,W)\nu,
\]
where the symmetric $(2,0)$ tensor $II$ is called \textit{second fundamental form}. It will reveal useful to consider the $I-$symmetric $(1,1)$-tensor $B\in (TS)^*\otimes TS$ defined by $II(V,W)=I(B(V),W)$ which is called the \textit{shape operator} of $\sigma.$ As in the Riemannian case, it turns out that $\sigma_*(B(v))=\nabla_v\nu.$\\  
The first and second fundamental form of an immersion $\sigma$ satisfy constraint equations, known as \textit{Gauss-Codazzi equations}. More preciselly the Gauss equation consists of the identity: 
\begin{equation}\label{Gauss}
    K_I=-1-\det_I II
\end{equation}
where $K_I$ is the curvature of $I$ and $\det_I II$ is $\det B$ by definition. On the other hand the Codazzi equation states that $\nabla^III$ is a totally symmetric $3-$form. We have: 
\begin{equation} \nabla 
\end{equation}

\textcolor{red}{mancano eq. di codazzi}

What can be shown is that the Gauss-Codazzi equations relating first and second fundamental forms are \textit{necessary} but also \textit{sufficient}, more explicitly we mean:

\begin{theorem}\label{immcondition}
    Let $S$ e a simply connected surface, let $I$ be a Riemannian metric on $S$ and $II$  be a symmetric $(2,0)-$tensor on $S$. If $I$ and $II$ satisfy the Gauss-Codazzi equations \refeq{Gauss} and  \textcolor{red}{blablabla}
\end{theorem}

\subsection{Germs of spacelike immersions in Ads manifolds.}

Let us now consider the case of an oriented surface $\Sigma,$ not necessarily simply connected. Given a spacelike immersion $\sigma:\Sigma\to(M,g)$ where $(M,g)$ is an oriented Anti-de Sitter manifold, we can associate to $\sigma$ the pair $(I,II)$ as done in the previous section, where $II$ is computed with respect to the future unit normal vector $\nu$ of $\sigma$. We will always assume that the orientation of $\Sigma$ and $\nu$ are compatible with the orientation on $M.$\\
The pair $(I, II)$ only depends on the \textit{germ} of the immersion $\sigma.$\\
Given a pair $(I,II)$ on a surface $\Sigma$, we can perform the following construction. Let $\pi:\tilde{\Sigma}\to\Sigma$ be a universal cover, it follows that the pair $(\pi^*I,\pi^*II)$ satisfies the Gauss-Codazzi equations on $\tilde{\Sigma},$ hence by the existance part of Theorem \ref{immcondition} there exist a spacelike immersion $\tilde{\sigma}:\tilde{\Sigma}\to\A^{2,1}$ having \textit{immersion data} $(\pi^*I,\pi^*II)$. The uniqueness part of Theorem \ref{immcondition} has two main consequences: 

\begin{itemize}
    \item Any two such immersion differ by post-composition by a global isometry of $\A^{2,1}$. 
    \item Given any such $\tilde{\sigma},$ there exists a map $\rho:\pi_1(\Sigma)\to\text{Isom}_0(\A^{2,1})$ such that, for every $\gamma\in\pi_1(\Sigma),$ $f\circ\gamma=\rho(\gamma)\circ f$. 
\end{itemize}

More can be proved, $\rho$ is a group representation and changing $\tilde{\sigma}$ by post-composition with an isometry $f$ has the effect of conjugating $\rho$ by $f$. The immersion $\sigma$ can then be extended to an immersion of $U$, an open neighborhood of $\Sigma\times\{0\}$ in $\Sigma\times\R$, into $\A^{2,1}$, by mapping $(x,t)$ to the point $\gamma(t)$ on the timelike geodesic $\gamma$ such that $\gamma(0)=\sigma(p)$ and $\gamma^{\prime}(0)$ is the future normal vector of $\sigma$ at $x$. We want to explicit the expressions of the Anti-de Sitter metric in such a tubular neighborhood of $\sigma$ as it will be useful in the following: 

\begin{lemma}
    

\begin{equation}\label{64}
    4
\end{equation}


\end{lemma}

In short, given a pair $(I,II)$, the expression \refeq{64} provides a Lorentianz metric of constant curvature $-1$ on an open set $U$in $\Sigma\times\R$ containing the slice $\Sigma\times\{0\}$, and thus a germ of immersion of $\Sigma$ into an Anti-de Sitter three-manifold with immersion data $(I,II)$.\\
We want now to move on the case where $\Sigma$ is a closed surfaces. In such instance, the equivariant immersion $\tilde{\sigma}$ via the representation \textcolor{red}{scritto male qua} is necessarily an embedding, which can be extended to an embedding of $\tilde{\Sigma}\times\R$ onto a domain of dependence in $\A^{2,1}.$
The representation $\rho:\pi_1(\Sigma)\to\T$ coincides with the holonomy of a maximal globally hyperbolic Anti-de Sitter manifold $(M,g)$ (after the identification between $\pi_1(\Sigma)$ with $\pi_1(M)$ via embedding of $\Sigma\to M\simeq\Sigma\times\R)$, therefore $\rho$ consists of a pair of positive Fuchsian representation by Proposition \textcolor{red}{5.5.3}. \\ 

\subsection{Gauss map and projections}
\textcolor{red}{it's the fixed point of the projection}

\begin{observation}
    Under the identification given by Fix, the left and right projections can be interpreted in the following way. Given $p\in S$, $\Pi_l(p)$ is the parallel transport in Id of the future unit vector $\nu(p)$ at $\sigma(p)$ with respect to the right-invariant connection $D^r$. The right projection is then obtained using the left-invariant connection. 

    \textcolor{red}{In realtà Diaf fa il remark sotto 6.3.6}
\end{observation}

We want now to prove formulae which express the pull-back of the hyperbolic metrics by the left and right projections. When applying these formulae to the embedding data of a surface in an MGH Cauchy compact Anti-de Sitter spacetime $(M,g)$, we obtain a pair of hyperbolic metrics whose isotopy classes are the parameters of $(M,g)$
in $\mathcal{T}(S)\times\mathcal{T}(S).$

\begin{proposition}
    Let $\sigma:S\to\A^{2,1}$ be a spacelike immersion, let $\Pi_l,\Pi_r:S\to\H^2$ be the left and right projections and let $g_{\H^2}$ be the hyperbolic metric. Then 
    \begin{equation}
        \Pi_l^*
    \end{equation}
\end{proposition}

\begin{proof}
    
\end{proof}

\textit{Non-smooth surfaces.} The construction of the Gauss map can be extended (\textit{how?}) in the non-smooth setting, for instance for convex spacelike surfaces $S\in\A^{2,1}$. Then one define the set-valued Gauss map as the map sending each $x\in S$ to the set of future unit vectors in $T_x^{1,+}\A^{2,1}$ orthogonal to support planes of $S$ at $x$. Hence the image of $x$ is a convex subset of $T_x\A^{2,1}$, and it reduces to a single point if and only if $S$ is differentiable at $x$. The image of $G$ in $T^{1,+}_x\A^{2,1}$ is a $C^{1.1}$ surface. \\

