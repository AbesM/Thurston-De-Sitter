\chapter{Anti-de Sitter space in dimension (2+1)}
We want now to specialize to dimension 3 Anti-de Sitter geometry as it will be the specific model geometry of the manifolds of our next interest. 
\section{The {PSL}$(2,\R)$ model.} 

The fundamental observation is the existence of a special model in dimension 3 which endows Anti-de Sitter space with a Lie group structure. To construct such a model we observe that $q=-$det is a quadratic form with signature (2,2) over the real vector space $\mathcal{M}(2,\R)$. The associated bilinear form is expressed by the formula:
\begin{equation}\label{quadratic}
    \langle A,B\rangle=-\frac{1}{2}\text{tr}(A\cdot\text{adj}(B))
\end{equation}
for $A,B\in\mathcal{M}(2,\R)$, where adj denotes the adjugate matrix, namely: 
\[
    \text{adj}(\begin{bmatrix}
        a & b \\
        c & d
    \end{bmatrix}) = \begin{bmatrix}
        d & -b \\
        -c & a
    \end{bmatrix}
\]

hence there is an identification between $(\mathcal{M}(2,\R),q)$ and $(\R^{2,2}, q_{2,2}),$ unique up to composition by elements in $O(2,2)$. Under this isomorphism $\H^{2,1}$ is identified with the Lie group $\text{SL}(2,\R).$\\
Let us observe that $\text{SL}(2,\R)\times \text{SL}(2,\R)$ acts linearly on $\mathcal{M}(2,\R)$ by left and right multiplication:
$$(A,B)\cdot X=AXB^{-1}.$$

As a direct consequence of the Binet formula, the action preserves the quadratic form $q=$-det and thus induces a representation: 

\[ \rho:\text{SL}(2,\R)\times \text{SL}(2,\R)\to O(\mathcal{M}(2,\R),q). \]

Since the center of $\text{SL}(2,\R)$ is $\{\pm \text{Id}\},$ the kernel of $\rho$ is $K=\{(\text{Id},\text{Id}),(-\text{Id},-\text{Id})\},$ and by a dimensional argument (and connectedness of $\text{SL}(2,\R)$z) it turns out that the image of the representation is the connected component of the identity: 
\[
    \text{Isom}_0(\H^{2,1})=\text{SO}_0(\mathcal{M}(2,\R),q)\simeq (\text{SL}(2,\R)\times \text{SL}(2,\R))/K
\]
    
Using this model, one has natural identification between $\A^{n,1}$ and the Lie group $\text{PSL}(2,\R)$, in such a way that: 
\[
    \text{Isom}_0(\A^{n,1})\simeq \text{PSL}(2,\R)\times\text{PSL}(2,\R)
\]
acting by left and right multiplication on $\text{PSL}(2,\R)$.\\ We are mostly interested in orientation-preserving and time-preserving notions that do not depend on a chosen orientation, nevertheless we will fix here an orientation and a time-orientation of $\A^{2,1}\simeq\text{PSL}(2,\R).$ As we are dealing with a Lie group it is sufficient to define on orientation of the Lie algebra, namely the tangent at the identity Id. We declare as (positive) oriented basis of $\mathfrak{sl}(2,\R):$ 
\[
V=\begin{pmatrix}
  0 & 1 \\ 1 & 0
\end{pmatrix}\;\;
W=\begin{pmatrix}
  1 & 0 \\ 0 & -1
\end{pmatrix}
\;\;
U=\begin{pmatrix}
  0 & -1 \\ 1 & 0
\end{pmatrix}
\]

The first two vectors $V,W$ are spacelike, while $U$ is timelike. $U$ is the tangent vector to the one-parameter group of elliptic isometries of $\H^2$ facing $i\in\H^2$, parametrized by the angle of clockwise rotation; $V,W$ are vectors tangent to the one-parameter groups of hyperbolic isometries fixing the geodesic with endpoints $(-1,1)$ and $(0,\infty)$ respectively. Time-orientation can also be inherited by the Lie algebra, we declare that $U$ is future-pointing timelike vector. \\

The stabilizer of the identity in Isom$_0(\A^{n,1})$ is the diagonal subgroup $\Delta<\text{PSL}(2,\R)\times\text{PSL}(2,\R)$. Under the obvious identification of $\text{PSL}(2,\R)$ and $\Delta,$ the action of the identity stabilizer on the Lie algebra $\mathfrak{sl}(2,\R)=T_{\text{Id}}\text{PSL}(2,\R)$ is the adjoint action of $\text{PSL}(2,\R)$. A direct consequence of this construction is the bi-invariance of the quadratic form $q$. Indeed, denoting by $q_{\text{Id}}$ the restriction of $q$ to $T_{\text{Id}}\text{SL}(2,\R)$, a direct computation shows that $q_\text{Id}$ equals $(1/8)\kappa$ where $\kappa(X,Y)=4tr(XY)$ is the Killing form of $\mathfrak{sl}(2, \R)$.

\begin{observation}
    The Lie algebra $\mathfrak{sl}(2,\R)$ equipped with the quadratic form $q_{\text{Id}}$ is then a copy of the 3-dimensional Minkowski space, hence the adjoint action yelds a representation 
    \[
        \text{PSL}(2,\R)\to O(\mathfrak{sl}(2,\R),q_{\text{Id}})
    \]

which in turn induces the well-known isomorphism: 
\[
    \text{SO}_0(2,1)\simeq\text{SO}_0(\mathfrak{sl}(2,\R), q_\text{Id})\simeq \text{PSL}(2,\R),
\] which is nothing but the restriction of the isomorphism \textcolor{red}{metti link} to the stabilizer of the identity in the left-hand side $\text{Isom}(\A^{2,1}),$ and to the diagonal subgroup $\Delta$ in the right-hand side $\text{PSL}(2,\R)\times\text{PSL}(2,\R).$
\end{observation}

\begin{observation}
    The identification between $\H^{n,1}$ and $\text{SL}(2,\R)$ it's the analogue of the more famous identification between $S^3$ and $SU(2)$. Stuff about quaternions, we are kinda of changing the metric over quaternions (circa?)
\end{observation}

\section{The boundary of PSL(2, $\R$)} 
From the aforementioned identification we obtain a new one between $\partial\A^{2,1}$ with the boundary of $\text{PSL}(2,\R)$, namely $P(\mathcal{M}(2,\R))$, the projectivization of the cone of rank 1 matrices. Therefore from now on we shall consider 
\[
    \partial\A^{2,1}=\{[X]\in\text{P}(\mathcal{M}(2,\R))|\;\text{rank}(X)=1\}
\]
We endow $\overline{\A^{2,1}}=\A^{2,1}\cup\partial\A^{2,1}$ with the topology induced by seeing both as subsets of the real projective space $\text{P}(\mathcal{M}(2,\R)).$ We want to observe that we have the following homeomorphism: 
\begin{align*}\label{delta}
    \delta:\partial\A^{2,1}&\rightarrow\R\text{P}^1\times\R\text{P}^1\\
    [X]&\mapsto (\text{Im}(X),\text{Ker}(X))
\end{align*}


where we are considering $\R\text{P}^1$ as the space of one-dimensional subspaces of $\R^2$. Since we have that $\text{Im}(AXB^{-1})=A\cdot\text{Im}(X)$ and $\text{Ker} (AXB^{-1})=B\cdot\text{Ker}(X),$ the map $\delta$ is equivariant with respect to the action of $\PSL\times\PSL,$ acting on $\partial\A^{2,1}$ as the natural extension of the group of isometries of $\A^{2,1}$ and on $\R\text{P}^1\times\R\text{P}^1$ by the obvious product action.\\
In this setting our choice of a time-orientation \textit{can be modified} according to the following Lemma:
\begin{lemma}\label{invertime}
    The inversion map $\iota[X]=[X]^{-1}$ \todo{non è più bello con il -1 dentro?} is a time-reversing isometry of $\A^{2,1}$ which induces the homeomorphism $(x,y)\to(y,x)$ on the boundary $\partial\A^{2,1}.$
\end{lemma}
\begin{proof}
    It follows from definition that $\iota$ is equivariant with respect to the isomorphism of $\T$ which switches left and right factors. As $\text{d}_\text{Id}\iota=-\text{Id}$ is a linear isometry, $\iota$ is an isometry, the differential being minus the identity also shows time-reversal. \\
    The second part of the statement can be checked via the following. For a $2\times2$ matrix the Cayley-Hamilton theorem implies the equality $(\det X)X^{-1}=(\text{tr}X)(\text{Id}-X),$ so that projectively we have $[X^{-1}]=[\text{tr}X\text{Id}-X].$ Hence $\iota$ extends to the transformation $[X]\to[\text{tr}X\text{Id}-X]$ on $\partial\A^{2,1}.$ Now is $X$ is a rank $1$ matrix, it is traceless if and only if $X^2=0$, hence $\text{Ker}(X)=\text{Im}(X).$ If $\text{tr}(X)\neq 0$, then $X$ is diagonalizable with eigenvalues $0$ and $\text{tr}(X)$. Moreover $\text{Ker}(X)$ and $\text{Im}(X)$ are the corresponding eigenspaces. It follows then that $\text{Ker}(\text{tr}X\text{Id}-X)=\text{Im}(X)$ and $\text{Im}(\text{tr}X\text{Id}-X=\text{Ker}X$ 
\end{proof}


Consider the hyperbolic model of the upper half-plane $\H^2.$ $\R\text{P}^1$ correspond to the boundary at infinity $\partial\H^2$ and $\text{PSL}(2,\R)$ is identified to $\text{Isom}_0(\H^2$.) In this key we can consider $\partial\A^{2,1}$ as $\partial\H^2\times\partial\H^2$. We can then interpret the convergence to $\partial\A^{2,1}$ in this setting:

\begin{lemma}\label{convergenza}
    A sequence $[X_n]\in\A^{2,1}$ converges to $(x,y)\in\partial\A^{2,1}\simeq\T$ if and only if for every $p\in\H^2,$ $X_n(p)\to x$ and $X_n^{-1}(p)\to y.$ 
\end{lemma}
\begin{proof}
    $\PSL$ acts on $\H^2$ via isometry, hence if the conditions holds for some $p$ then it holds for all $p\in\H^2$. Without loss of generality we can assume $p=i$ in the upper halp-plane. Assuming $X_n$ converges projectively to a rank 1 matrix $X$, one check immediately that $X(p)$ in the projective class of $x=\text{Im}(X).$ The convergence of $X_n^{-1}(p)\to y$ is just a straightforward application of Lemma \ref{invertime}
\end{proof}

In dimension $3 \partial\A^{2,1}$ is a doubly ruled quadric, which in an affine chart looks like \textcolor{red}{inserire cono}. We shall describe such rulings in a more geometric way. Given any $(x_0,y_0)\in\partial\A^{2,1}$, 
\[
    \lambda_{y_0}\coloneqq\{(x,y_0)|x\in\R\text{P}^1\}
\]  
describe a projective line in $\R\text{P}^3$ which is contained in $\partial\A^{2,1}$, hence lightlike for the conformal Lorentzian structure of $\partial\A^{2,1}$ \textcolor{red}{according to a remark that i dont have,}. In fact $\lambda_{y_0}$ is the orbit of $(x_0,y_0)$ by the action of $\text{PSL}(2,\R)\times\{\text{Id}\},$ or by the (now free) action of $\text{PSO}(2,\R)\times\{\text{Id}\}.$ Here $\text{PSO}(2,\R)$ corresponds to a 1-paramater elliptic subgroup in $\text{PSL}(2,\R)$. In short: 
\[
    \lambda_{y_0}=\text{PSL}(2,\R)\cdot(x_0,y_0)=\text{PSO}(2,\R)\cdot(x_0,y_0).
\]

We refer to $\lambda_{y_0}$ as the \textit{left ruling} through $(x_0,y_0)$, and similarly the \textit{right ruling} is 
\[
    \mu_{x_0}\coloneqq\{(x_0,y)|y\in\R\text{P}^1\}
\]

One of the fundamental idea of the earthquake theorem is that to any given map $f:\partial\H^2\to\partial \H^2$ we can associate a subset of $\partial\A^{2,1}$, namely (via the map $\delta$ introduced in \refeq{delta}) the graph of $f$. It follows from the equivariance of $\delta$ that for any $(\alpha,\beta)\in\T$:
\begin{equation}
    (\alpha,\beta)\cdot\text{graph}(f)=\text{graph}(\beta f\alpha^{-1}).
\end{equation} 

We remark one last time that we will consider $\partial\A^{2,1}$ as always implicitly identified $\T$ via $\delta$.
\subsection{Geodesic in $\PSL$.} We have already seen geodesic in the general Anti-de Sitter space, we would like to specialize here using the model of $\PSL$ and tools from general Lie Group theory.  We would like to start considering geodesic through the identity. The Lie algebra of $\PSL$ is isometrically identified with Minkowski space, where under

\section{Spacelike planes}
We want to study now totally geodesic spacelike planes in $\A^{2,1}.$ They are all obtained as the intersection of $\A^{2,1}$ with a projective subspace in the projective space $\text{P}\mathcal{M}(2,\R).$ Hence they are all of the the following form: 
\begin{equation}\label{geoplanes}
    P_{[A]}=\{[X]\in\text{PSL}(2,\R)\;|\;\langle X,A\rangle=0\}
\end{equation}
for some non-zero matrix $A$. The notation is justified by the observation that the plane defined by $P_A$ depends only on the projective class of $A$. The totally geodesic plane is spacelike if and only if $q(A)=-\det A$ is negative. We will call such a plane \textit{dual plane} of $A$, in particular the dual plane $\text{P}_\gamma$  of an element $\gamma\in\text{PSL}(2,\R)$ is a spacelike totally geodesic plane. \\
\textit{Example}
Before the general treatment we want to focus now on a concrete example. Consider $\gamma=\text{Id}\in\text{PSL}(2,\R).$ Now because of \refeq{quadratic}, $\text{P}_{\text{Id}}$ is the subset of $\text{PSL}(2,\R)$ consisting of projective class of unit matrices $X$ with $\text{tr}(X)=0$. By the Cayley-Hamilton theorem, $X^2=-\text{Id},$ hence the elements of $\text{P}_{\text{Id}}$ are order-two isometries of $\H^2,$ that is, elliptic elements with rotation angle $\pi$.  We can also observe that $\text{P}_{\text{Id}}$ is invariant under the actions of $\PSL$ by conjugation, which corresponds to the diagonal in the isometry group $\PSL\times\PSL$ of $\A^{2,1}$. Using \ref{convergenza}, we can see that the boundary of $\text{P}_\gamma$ in $\partial\A^{2,1}\simeq \R\text{P}^1\times\R\text{P}^1$ is the diagonal; more precisely: 
\begin{equation}\label{graphino}
    \partial\text{P}_\gamma=\text{graph(Id)}\subset \T.
\end{equation}

Now consider a point $z\in\H^2$, and let us denote by $\mathcal{R}_z$ the order-two elliptic isometry with fixed point $z$. We claim that the map 
\[
    \iota:\H^2\to\text{P}_{\text{Id}},\;\;\;\ \iota(z)=\mathcal{R}_z
\]

is an isometry with respect to the hyperbolic metric of $\H^2$ and the induced metric on $\text{P}_{\text{Id}}.$ First, the inverse of $\iota$ is simply the fixed-point map $\text{Fix}:\text{P}_{\text{Id}}\to\H^2$ sending an elliptic isometry to its fixed point, which also shows that $\iota$ is equivariant with respect to the action of $\PSL on \H^2$ by homographies and on $\text{P}_\text{Id}$ by conjugation, since $\text{Fix}(\alpha\gamma\alpha^{-1})=\alpha\text{Fix}(\gamma).$ That is, the following holds: 
\begin{equation}
    \iota(\alpha\cdot p)=\alpha\circ\iota(p)\circ\alpha^{-1}.
\end{equation} 
A direct consequence is that $\iota$ is isometric, since the pull-back of the metric of $\text{P}_\text{Id}$ is necessarily $\PSL$-invariant and has constant curvature -1, hence it coincides with the standard hyperbolic metric on the upper half-plane.\\
This simple example is actually the key case to understand general spacelike totally geodesic planes as every spacelike totally geodesic plane is of the form $\text{P}_\gamma$ for some $\gamma\in\PSL.$ To see this, observe that the action of the isometry group of $\A^{2,1}$ on spacelike totally geodetic planes is transitive, and that $\text{P}_\gamma=(\gamma,\text{Id})\text{P}_{\text{Id}}$ as the isometry $(\gamma,Id)$ maps $\text{Id}\to\gamma$, and therefore the dual plane of $\text{Id}$ to the dual plane of $\gamma.$ In view of \textcolor{red}{missinho} and \refeq{graphino}, we can conclude the following: 
\begin{lemma}
    Every spacelike totally geodesic plane of $\A^{2,1}$ is of the form $\text{P}_\gamma$ for some orientation-preserving isometry $\gamma\in\PSL,$ and 
    \[
        \partial\text{P}_\gamma=\text{graph}(\gamma^{-1})\subset\T.
    \]
\end{lemma}
\section{Timelike planes}
Let's consider now a matrix $A\in\mathcal{M}(2,\R)$ such that $\det(A)=-1.$ The corresponding plane $P_A$ defined by equation \refeq{geoplanes} is a timelike totally geodesic plane. Associated with $[A]$ is an orientation-reversing isometry $\eta$ of $\H^2.$ Moreover, the action of $A$ by homography on $\mathbb{C}\text{P}^1$ preserves $\S$ and switches the two connected components of the complement. The matrix $A$ thus induces an orientation-reversing isometry, up to identifying the two components via $z\to\overline{z}$. We will thus denote $P_[A]$ by $P_\eta$. \\
The totally geodesic timelike plane $P_\eta$ can now be parametrized as follows. We have a map: 
\begin{equation}\label{refspa}
    \mathcal{I}\to\mathcal{I}_\eta
\end{equation}
from the spaces of reflections $\mathcal{I}$ along geodesic of $\H^2$, with values in $\PSL\simeq\A^{2,1}$. It's image it precisely $P_\eta.$ As seen in the proof of \ref{invertime} we have that an $X$ with determinant $-1$ is an inversion if and only if traceless. As $\det(A)=-1$ we have $\text{adj}(A)=-A^{-1}$, therefore $\langle XA,A\rangle=0$ if and only if $\text{tr}(X)=0$, that is $X$ is traceless hence an involution. This shows that the image of the map defined \refeq{refspa} is the entire plane $P_\eta$.\\  
In similar fashion to the spacelike case, using the transitivity of the group of isometries on timelike planes, every timelike planes is of the form above. We can show the following: 
\begin{lemma}
    Every timelike totally geodesic plane of $\A^{2,1}$ is of the form $P_\eta$ for some orientation-reversing isometry $\eta\in\PSL,$ and 
    \[
        \partial\text{P}_\eta=\text{graph}(\eta^{-1})\subset\T.
    \]
\end{lemma}
\begin{proof}
    Suppose we have a sequence $\mathcal{I}_n$
\end{proof}
\begin{proof}
    Since reflections of $\H^2$ are uniquely determined by (unoriented) geodesics, we can consider the map defined in \refeq{refspa} as a map from the space $\mathcal{G}(\H^2)$ of unoriented geodesics of $\H^2$ to $\PSL.$ This map is isometric with respect to the natural metric \textcolor{red}{quale} on $\mathcal{G}(\H^2)$ which makes it identified with the two-dimensional space $\A^{1,1}$ as treated more in detail in \textcolor{red}{Reference}
    \end{proof}
\section{Lightlike planes}
\section{affine charts MAYBE TO MOVE INTO MESS SECTION OR EITHER DELETE THE WHOLE MESS THING?}
