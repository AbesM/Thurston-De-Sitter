\chapter{Anti-de Sitter space in dimension (2+1)}
We want now to specialize to dimension 3 Anti-de Sitter geometry as it will be the specific model geometry of the manifolds of our next interest. 
\section{The {PSL}$(2,\R)$ model.} 

The fundamental observation is the existance of a special model in dimension 3 which endows Anti-de Sitter space with a Lie group structure. To construct such a model we observe that $q=-$det is a quadratic form with signature (2,2) over the real vector space $\mathcal{M}(2,\R)$. The associated bilinear form is expressed by the formula:
\begin{equation}\label{quadratic}
    \langle A,B\rangle=-\frac{1}{2}\text{tr}(A\cdot\text{adj}(B))
\end{equation}
for $A,B\in\mathcal{M}(2,\R)$, where adj denotes the adjugate matrix, namely: 
\[
    \text{adj}(\begin{bmatrix}
        a & b \\
        c & d
    \end{bmatrix}) = \begin{bmatrix}
        d & -b \\
        -c & a
    \end{bmatrix}
\]

hence there is an identification between $(\mathcal{M}(2,\R),q)$ and $(\R^{2,2}, q_{2,2}),$ unique up to composition by elements in $O(2,2)$. Under this isomorphism $\H^{2,1}$ is identified with the Lie group $\text{SL}(2,\R).$\\
Let us observe that $\text{SL}(2,\R)\times \text{SL}(2,\R)$ acts linearly on $\mathcal{M}(2,\R)$ by left and right multiplication:
$$(A,B)\cdot X=AXB^{-1}.$$

As a direct consequence of the Binet formula, the action preserves the quadratic form $q=$-det and thus induces a representation: 

\[ \rho:SL(2,\R)\times SL(2,\R)\to O(\mathcal{M}(2,\R),q). \]

Since the center of $SL(2,\R)$ is $\{\pm \text{Id}\},$ the kernel of $\rho$ is $K=\{(\text{Id},1),(-\1,-\1)\},$ and by a dimensional argument it turns out that the image of the representation is the connected component of the identity: \todo{Perché è immagine di un connesso e contiene l'identità. Per le dimensioni sono le stesse ma sto divendo per due per il ker? sus}

\[
    \text{Isom}_0(\H^{2,1})=\text{SO}_0(\mathcal{M}(2,\R),q)\simeq (\text{SL}(2,\R)\times \text{SL}(2,\R))/K
\]
    
Using this model, one has natural identification between $\A^{n,1}$ and the Lie group $\text{PSL}(2,\R)$, in such a way that: 
\[
    \text{Isom}_0(\A^{n,1})\simeq \text{PSL}(2,\R)\times\text{PSL}(2,\R)
\]
acting by left and right multiplication on $\text{PSL}(2,\R)$.\\ We are mostly interested in orientation-preserving and time-preserving notions that do not depend on a chosen orientation, nevertheless we will fix here an orientation and a time-orientation of $\A^{2,1}\simeq\text{PSL}(2,\R).$ As we are dealing with a Lie group it is sufficient to define on orientation of the Lie algebra, namely the tangent at the identity Id. We declare as (positive) oriented basis of $\mathfrak{sl}(2,\R):$ 
\[
    \text{Inserire V,W,U}
\]  

The first two vectors $V,W$ are spacelike, while $U$ is timelike. $U$ is the tangent vector to the one-parameter group of elliptic isometries of $\H^2$ ficing $i\in\H^2$, parametrized by the angle of clockwise rotation; $V,W$ are vectors tangent to the one-parameter groups of hyperbolic isometries fixing the geodesic with endpoints $(-1,1)$ and $(0,\infty)$ respectively. Time-orientation can also be inherited by the Lie algebra, we declare that $U$ is future-pointing timelike vector. \\

The stabilizer of the identity in Isom$_0(\A^{n,1})$ is the diagonal subgroup $\Delta<\text{PSL}(2,\R)\times\text{PSL}(2,\R)$. Under the obvious identification of $\text{PSL}(2,\R)$ and $\Delta,$ the action of the identity stabilizer on the Lie algebra $\mathfrak{sl}(2,\R)=T_{\text{Id}}\text{PSL}(2,\R)$ is the adjoint action of $\text{PSL}(2,\R)$. A direct consequence of this construction is the bi-invariance of the quadratic form $q$. Indeed, denoting by $q_{\text{Id}}$ the restriction of $q$ to $T_{\text{Id}}\text{SL}(2,\R)$, a direct computation shows that $q_\text{Id}$ equals $(1/8)\kappa$ where $\kappa(X,Y)=4tr(XY)$ is the Killing form of $\mathfrak{sl}(2, \R)$.

\begin{observation}
    The Lie algebra $\mathfrak{sl}(2,\R)$ equipped with the quadratic form $q_{\text{Id}}$ is then a copy of the 3-dimensional Minkowski space, hence the adjoint action yelds a representation 
    \[
        \text{PSL}(2,\R)\to O(\mathfrak{sl}(2,\R),q_{\text{Id}})
    \]

which in turn induces the well-known isomorphism: 
\[
    \text{SO}_0(2,1)\simeq\text{SO}_0(\mathfrak{sl}(2,\R), q_\text{Id})\simeq \text{PSL}(2,\R),
\] which is nothing but the restriction of the isomorphism \textcolor{red}{metti link} to the stabilizer of the identity in the left-hand side $\text{Isom}(\A^{2,1}),$ and to the diagonal subgroup $\Delta$ in the right-hand side $\text{PSL}(2,\R)\times\text{PSL}(2,\R).$
\end{observation}

\begin{observation}
    The identification between $\H^{n,1}$ and $\text{SL}(2,\R)$ it's the analogue of the more famous identification between $S^3$ and $SU(2)$. Stuff about quaternions, we are kinda of changing the metric over quaternions (circa?)
\end{observation}

\section{The boundary of PSL(2, $\R$)} 
From the aforementioned identification we obtain a new one between $\partial\A^{2,1}$ with the boundary of $\text{PSL}(2,\R)$, namely $P(\mathcal{M}(2,\R))$, the projectivization of the cone of rank 1 matrices. Therefore from now on we shall consider 
\[
    \partial\A^{2,1}=\{[X]\in\text{P}(\mathcal{M}(2,\R))|\;\text{rank}(X)=1\}
\]
We endow $\overline{\A^{2,1}}=\A^{2,1}\cup\partial\A^{2,1}$ with the topology induced by seeing both as subsets of the real projective space $\text{P}(\mathcal{M}(2,\R)).$ We want to observe that we have the following homeomorphism: 
\[
    \delta:\partial\A^{2,1}\to\R\text{P}^1\times\R\text{P}^1
\] \todo{definire la funzione con latex }

where we are considering $\R\text{P}^1$ as the space of one-dimensional subspaces of $\R^2$. Since we have that $\text{Im}(AXB^{-1})=A\cdot\text{Im}(X)$ and $\text{Ker} (AXB^{-1})=B\cdot\text{Ker}(X),$ the map $\delta$ is equivariant with respect to the action of $\PSL\times\PSL,$ acting on $\partial\A^{2,1}$ as the natural extension of the group of isometries of $\A^{2,1}$ and on $\R\text{P}^1\times\R\text{P}^1$ by the obvious product action.\\
Consider the hyperbolic model of the upper half-plane $\H^2.$ $\R\text{P}^1$ correspond to the boundary at infinity $\partial\H^2$ and $\text{PSL}(2,\R)$ is identified to $\text{Isom}_0(\H^2$.) In this key we can consider $\partial\A^{2,1}$ as $\partial\H^2\times\partial\H^2$. We can then interpret the convergence to $\partial\A^{2,1}$ in this setting:

\begin{lemma}
    A sequence $[X_n]\in\A^{2,1}$ converges to $(x,y)\in\partial\A^{2,1}\simeq\T$ if and only if for every $p\in\H^2,$ $X_n(p)\to x$ and $X_n^{-1}(p)\to y.$ 
\end{lemma}

In dimension $3 \partial\A^{2,1}$ is a doubly ruled quadric, which in an affine chart looks like \textcolor{red}{inserire cono}. We shall describe such rulings in a more geometric way. Given any $(x_0,y_0)\in\partial\A^{2,1}$, 
\[
    \lambda_{y_0}\coloneqq\{(x,y_0)|x\in\R\text{P}^1\}
\]  
describe a projective line in $\R\text{P}^3$ which is contained in $\partial\A^{2,1}$, hence lightlike for the conformal Lorentzian structure of $\partial\A^{2,1}$ \textcolor{red}{according to a remark that i dont have,}. In fact $\lambda_{y_0}$ is the orbit of $(x_0,y_0)$ by the action of $\text{PSL}(2,\R)\times\{\text{Id}\},$ or by the (now free) action of $\text{PSO}(2,\R)\times\{\text{Id}\}.$ Here $\text{PSO}(2,\R)$ corresponds to a 1-paramater elliptic subgroup in $\text{PSL}(2,\R)$. In short: 
\[
    \lambda_{y_0}=\text{PSL}(2,\R)\cdot(x_0,y_0)=\text{PSO}(2,\R)\cdot(x_0,y_0).
\]

We refer to $\lambda_{y_0}$ as the \textit{left ruling} through $(x_0,y_0)$, and similarly the \textit{righ ruling} is 
\[
    \mu_{x_0}\coloneqq\{(x_0,y)|y\in\R\text{P}^1\}
\]

\section{Spacelike planes}
We want to study now totally geodesic spacelike planes in $\A^{2,1}.$ They are all obtained as the intersection of $\A^{2,1}$ with a projective subspace in the projective space $\text{P}\mathcal{M}(2,\R).$ Hence they are all of the the following form: 
\[
    P_{[A]}=\{[X]\in\text{PSL}(2,\R)\;|\;\langle X,A\rangle=0\} 
\] for some non-zero matrix $A$. The notation is justified by the observation that the plane defined by $P_A$ depends only on the projective class of $A$. \todo{How so?} The totally geodesic plane is spacelike if and only if $q(A)=-\det A$ is negative. We will call such a plane \textit{dual plane} of $A$, in particular the dual plane $\text{P}_\gamma$  of an element $\gamma\in\text{PSL}(2,\R)$ is a spacelike totally geodesic plane. \\
\textit{Example}
Before the general treatment we want to focus now on a concrete example. Consider $\gamma=\text{Id}\in\text{PSL}(2,\R).$ Now because of \refeq{quadratic}, $\text{P}_{\text{Id}}$ is the subset of $\text{PSL}(2,\R)$ consisting of projective class of unit matrices $X$ with $\text{tr}(X)=0$. By the Cayley-Hamilton theorem, $X^2=-\text{Id},$ hence the elements of $\text{P}_{\text{Id}}$ are order-two isometries of $\H^2,$ that is, elliptic elements with rotation angle $\pi$.  We can also observe that $\text{P}_{\text{Id}}$ is invariant under the actions of $\PSL$ by conjugation, which corresponds to the diagonal in the isometry group $\PSL\times\PSL$ of $\A^{2,1}$. 

\section{affine charts}
The initial step in our proof is to consider the graph of an orientation-preserving homeomorphism $f:\R\text{P}^1\to\R\text{P}^1$ as a subset of $\partial\A^{2,1}$, and taking its convex hull. However, the convex hull of a set in projective space can be defined in affine chart, but $\overline{\A^{2,1}}$ is not contained in any affine chart. The following lemma has the purpose to show that the convex hull of the graph of $f$ is well-defined: 
\begin{lemma}
    Let $f:\R\text{P}^1\to\R\text{P}^1$ be an orientation-preserving homeomorphism. Then: 
    \begin{enumerate}
        \item There exists a spacelike plane $\text{P}_\gamma$ in $\A^{2,1}$ such that $\partial\A^{2,1}\cap\text{graph}(f)=\emptyset.$
        \item Given any point $(x_0,y_0)\notin\text{graph}(f)$, there exists a spacelike plane $\text{P}_\gamma$ such that $\partial\text{P}_\gamma=\emptyset$ and $(x_0,y_0)\in\partial\text{P}_\gamma.$ 
    \end{enumerate}
\end{lemma}

\begin{proof}
    \begin{enumerate}
        \item We recall that $\PSL$ acts transitively on pairs of distinct points of $\R\text{P}^1\simeq\R\cup\{\infty\}$ (actually more is true, as it acts simply transitively on \textit{positively oriented triples}). Hence we may assume, up to the action of 
    \end{enumerate}
\end{proof}